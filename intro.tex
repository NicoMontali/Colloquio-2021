Per capire appieno la profondità del dibattito che si svolse a metà del XX secolo tra Claude Levi-Strauss e Jean-Paul Sartre occorrerebbe molto più di un lavoro di colloquio, dal momento che si tratta di due delle più influenti e originali personalità intellettuali in Francia dell'epoca. Per svolgere nella maniera più chiara possibile una ricostruzione storico-filosofica di tale dibattito, pertanto, è mia intenzione utilizzare le opere centrali dello scambio tra i due intellettuali - si perdoni il paragone botanico - come il tronco di un albero, dal quale discendere a ritroso nel tempo fino alle radici e, parimenti, procedere in avanti nella storia della filosofia fino a concentrarsi su alcuni rami. Ovviamente, per circoscrivere il lavoro da svolgere, l’attenzione dell’autore si concentrerà sulle opere di pensatori e intellettuali più direttamente influenti sugli autori analizzati ed influenzati da questi, ovvero, per tornare a fare uso del paragone botanico, sui rami principali e più prossimi al tronco, e sulle radici meno in profondità nel terreno.
Il lavoro, pertanto, si articolerà in tre capitoli per comodità cronologica, anche se il primo e il terzo capitolo costituiranno due metà di un unico discorso.
Ricalcando il saggio di Bruno Karsenti all'interno del volume \textit{Simposio L\'evi-Strauss. Uno sguardo dall'oggi}\footnote{\textit{Le proprietà di uno specchio deformante} pp.XXXX in \cite{simposio2013}}, il compito di Jean-Paul Sartre e della sua opera, in particolare i due volumi componenti la monumentale \textit{Critica della Ragion dialettica} sarà quello di fungere da specchio deformante, ossia evidenziare le peculiarità della posizione di Lévi-Strauss attraverso l'adozione di una concezione della ragione diametralmente differente, anche se non per questo incompatibile. 

Claude Lévi-Strauss, come si vedrà approfonditamente in seguito, si può a buon diritto collocare tra quei pochi e grandi pensatori situati al bivio tra due o più discipline, in particolare la filosofia, declinando e reimpiegando le intuizioni filosofiche nelle scienze umane.
Questi eclettici, di cui la Francia può vantare un cospicuo numero, si pensi a Pierre Bourdieu così come ad Émile Durkheim e Henri Bergson, ma anche a Henri Poincaré e Gaston Bachelard in materia più propriamente \textit{scientifica}, hanno non solo il grande merito di rinnovare e rivitalizzare la loro disciplina, ma anche il gravoso compito di ridefinire il lessico e gli strumenti di tale disciplina, aprendo un dibattito di carattere epistemologico che va a ricollocare le scienze umane all'interno del paradigma scientifico dominante.
Il ruolo di Lévi-Strauss, com'é stato ampiamente sottolineato\footnote{La bibliografia di e su Lévi-Strauss è sterminata; per un elenco di opere si rimanda alla sitografia stilata dagli studiosi francesi e pubblicata rispettivamente sul sito della Biblioteca Nazionale di Parigi e della EHESS: 
\cite{levibibliographie} e anche 
\cite{levibibliographie2}.} è stato capitale nel ridefinire le possibilità interpretative dell'antropologia attraverso l'uso rigoroso del concetto di struttura. Mentre alcuni suoi contemporanei, in seguito indicati come \textit{strutturalisti}, ossia come facenti uso della nozione di "struttura" nel loro campo d'indagine, sono stati guardati con crescente sospetto nel passare degli anni, il lavoro di Lévi-Strauss è ancora oggi oggetto di discussione, da cui è facile dedurre quale sia la portata delle intuizioni dell'antropologo francese\footnote{Per una ricostruzione storica e un contributo esegetico dettagliato delle declinazioni che la nozione di "struttura" ha ricevuto nel corso dei decenni si rinvia alle opere di Sergio Moravia, Raymond Boudon e Jean Piaget: \cite{moravia1975strutturalismo}, \cite{boudon2020strutturalismo} e ancora \cite{jean1968strutturalismo}. Nonostante appaiano oggigiorno poco aggiornate costituiscono un importante documento circa la ricezione da parte dei contemporanei.}.

Il presente lavoro, tuttavia, non si concentrerà tanto sulla nozione di struttura, che rimane imprescindibile per comprendere l'opera dell'antropologo, quanto invece sulla concezione levistraussiana di ragione, in particolare quanto elaborato nell'ultimo capitolo de \textit{La pensée sauvage}. Come ricorda giustamente Pierre Guenancia, l'ultimo capitolo di questa maestosa opera costituisce un discorso autonomo, in gran parte svincolato dai capitoli precedenti, ma non per questo privo di continuità con il pensiero levistraussiano.
\textit{Il pensiero selvaggio} si può a buon diritto considerare come il prodotto compiuto di un pensatore che già si è confrontato con un'umanità altra rispetto alla civiltà occidentale, ha fatto suo un enorme bagaglio culturale etnografico osservandolo in prima persona, conosce il pensiero, le orme degli antropologi ed etnologi che prima di lui hanno percorso la sua stessa strada, e solo ora, finalmente, si concede una sintesi originale sul sistema di pensiero che costituisce l'unità minima dell'attività intellettuale umana, desunta dalle ricerche esposte nel monumentale \textit{Le strutture elementari della parentela}, il pensiero selvaggio, appunto.
D'altronde, già Lucien Lévy-Bruhl, eminente filosofo e antropologo all'Università Sorbona di Parigi si era concentrato sulla definizione di una mentalità \textit{primitiva}, profondamente differente da quella occidentale a causa dell'assenza del principio di non contraddizione, ma dominata invece dal principio di \textit{partecipazione}; ed anche Émile Durkheim, nel suo ultimo grande lavoro, \textit{Le forme elementari della vita religiosa}, si dedica da una parte allo studio del totemismo, da lui identificato come la forma più primitiva di religione, ma tenta anche, partendo da questa specie di rappresentazioni collettive, di pervenire ad una mentalità primitiva che non solo permette, ma presiede le rappresentazioni collettive di carattere religioso.
Ciò che lega le opere dell'ultimo Durkheim, di Lévy-Bruhl e di Lévi-Strauss è l'intenzione di indagare le istituzioni sociali per andare a rinvenire un sostrato celato sotto il velo dei contenuti: uno stadio del pensiero non ancora addomesticato, non ancora ammansito attraverso le leggi della logica, il pensiero allo stato selvaggio.
Questo è esattamente l'ambizioso progetto di Claude Lévi-Strauss, per quanto viene sviluppato soprattutto ne \textit{Il pensiero selvaggio}: partire dall'antropologia per giungere ad una teoria della mente dell'uomo.
Nella Francia del secondo XX secolo, tuttavia, Claude Lévi-Strauss e la sua antropologia debbono contendersi la scena intellettuale con le filosofie "tradizionali", meno legate alle scienze umane e alle riflessioni recenti di queste; tra gli esponenti di tali filosofie, un posto di rilievo spetta sicuramente al poliedrico Jean-Paul Sartre.
Quest'ultimo, all'epoca dell'uscita delle opere più influenti di Lévi-Strauss, era un intellettuale di spicco, dichiaratamente di sinistra, tuttavia critico nei confronti della sinistra internazionale, in quel periodo sconvolta dall'emergere dei gesti di Josif Stalin in Unione Sovietica.
Sartre, scrittore poligrafo estremamente prolifico, per quanto in polemica con la figura del \textit{philosophe} allievo dell'Ècole Normale, si poteva definire a tutti gli effetti un filosofo, celebre in particolare per \textit{L'essere e il nulla}, opera di carattere esistenzialista, nella quale confluivano le idee già elaborate nei romanzi \textit{Il muro} e \textit{La nausea}, e maturata in seguito al fecondo incontro con i tedeschi Husserl e Heidegger, dai quali aveva mutuato un'impostazione fenomenologica. Dopo questo volume il genio sartriano, non solo celebrato ma anche criticato per il carattere pessimista delle sue posizioni, difese la sua concezione di esistenzialismo connotandola di maggior ottimismo e propositività all'interno della conferenza \textit{L'esistenzialismo è un umanismo}, edito in seguito come volume autonomo.
Il pensiero sartriano, per utilizzare una categoria schematica e per questo imprecisa, si può far rientrare tra le filosofie della riflessione, dove con questo termine si intende un modo di fare filosofia che predilige il sé come soggetto ed oggetto dell'indagine. Il soggetto sartriano, per come viene definito all'interno dell'\textit{Essere e il nulla} si costituisce come 

Come si avrà modo di vedere, infatti, entrambi gli autori, "campioni" a loro modo della loro disciplina in terra francese, si dedicano entrambi alla descrizione e all'analisi critica della razionalità dell'uomo, specialmente per come essa si manifesta all'interno della storia e dei gruppi sociali.
All'interno dell'opera di Sartre confluiscono diverse componenti: il marxismo, la filosofia della storia, l'esistenzialismo, per indicare le principali; mentre in Lévi-Strauss vi è una forma assai differente di marxismo, un'avversione ostentata nei confronti della filosofia accademica, l'antropologia e l'etnologia, la formazione filosofica "ripudiata". Ciò è sufficiente ad intuire quanto vi fosse in gioco quando i due attori 