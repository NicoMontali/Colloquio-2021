Per capire appieno la profondità del dibattito che si svolse a metà del XX secolo tra Claude Levi-Strauss e Jean-Paul Sartre occorrerebbe molto più di un lavoro di colloquio, dal momento che si tratta di due delle più influenti e originali personalità intellettuali francesi del loro tempo. Per operare nel modo più chiaro possibile una ricostruzione storico-filosofica di tale dibattito, pertanto, è mia intenzione trattare le opere centrali dello scambio tra i due intellettuali - per fare uso di un paragone botanico - come il tronco di un albero, dal quale discendere a ritroso nel tempo fino alle radici e, parimenti, procedere cronologicamente nella storia della filosofia fino a concentrarsi su alcuni rami. Ovviamente, per circoscrivere il lavoro da svolgere, l’attenzione si concentrerà sulle opere e sugli intellettuali maggiormente influenti ed influenzati da Sartre e Lévi-Strauss, ovvero, per tornare a fare uso del paragone botanico, sui rami principali e più prossimi al tronco, e sulle radici appena sotto la superficie. Il lavoro, pertanto, si articolerà in tre capitoli, seguendo una scansione cronologica, anche se il primo e il terzo capitolo, di fatto, costituiscono due metà di un unico discorso.\par
Seguendo l'esempio di Bruno Karsenti\footnote{\cite{proprietà2013karsenti}}, il compito di Jean-Paul Sartre e della sua opera, in particolare i due volumi componenti la monumentale \textit{Critica della Ragion dialettica} sarà quello di fungere da specchio deformante, ossia evidenziare le peculiarità della posizione di Lévi-Strauss attraverso l'adozione di una concezione della ragione diametralmente differente, ma paradossalmente complementare. \par \vspace{0.25cm}
Claude Lévi-Strauss, come si vedrà più approfonditamente in seguito, si può collocare tra quei rari quanto grandi pensatori situati all'intersezione tra due o più discipline, in particolare la filosofia, l'etnologia, l'antropologia e le scienze umane in generale.\par
Questi eclettici, di cui la Francia può vantare un discreto numero, si pensi a Denis Diderot, Pierre Bourdieu, Émile Durkheim\footnote{Non si cada nell'errore di considerare Durkheim esclusivamente un sociologo, per quanto pioniere: animato dallo stesso spirito di Auguste Comte, nel suo lavoro egli intrattiene con le altre scienze umane un rapporto di reciproco arricchimento, riunisce attorno a sé una rete di collaboratori dalle competenze varie per tentare di svolgere un vasto e completo studio della realtà sociale nelle sue molteplici dimensioni (morali, religiosi, giuridici, economici).}, Henri Bergson, ma anche a Henri Poincaré e Gaston Bachelard sul versante scientifico, hanno non solo il grande merito di rinnovare e rivitalizzare la loro disciplina, ma anche l'oneroso compito di ridefinire il lessico e gli strumenti di tale disciplina, aprendo un dibattito di carattere epistemologico che va a ricollocare le scienze umane all'interno del paradigma scientifico dominante.\par \vspace{0.25cm}
Il ruolo di Lévi-Strauss, com'é stato ampiamente sottolineato\footnote{L'opera di Claude Lévi-Strauss e la bibliografia che lo riguarda è sterminata; per una lista aggiornata si rimanda a \cite{levibibliographie}; \cite{abeles2004bibliographie}; \cite{levibibliografia}} è stato capitale nel ridefinire le possibilità interpretative dell'antropologia attraverso l'uso rigoroso del concetto di \enquote{struttura}. Mentre alcuni suoi contemporanei, aderenti anche loro al movimento \textit{strutturalista}, ossia facenti uso della nozione di \enquote{struttura} nel loro campo d'indagine, sono stati guardati con crescente sospetto con il passare degli anni, il lavoro di Lévi-Strauss è stato sicuramente il più foriero di innovazioni nel suo campo di applicazione\footnote{Per una ricostruzione storica e un contributo esegetico dettagliato delle declinazioni che la nozione di \enquote{struttura} ha ricevuto nel corso dei decenni si rinvia a: \cite{moravia1975strutturalismo}, \cite{boudon2020strutturalismo} e ancora \cite{jean1968strutturalismo}.}.\par 
Il presente lavoro, tuttavia, non si concentrerà tanto sulla nozione di \textit{struttura}, che rimane imprescindibile per comprendere l'opera dell'antropologo, quanto invece sulla concezione levistraussiana di ragione, in particolare quanto elaborato ne \textit{La pensée sauvage}. Come ricorda giustamente Pierre Guenancia, l'ultimo capitolo di quest'opera merita un posto a sé, in gran parte svincolato dai capitoli precedenti, ma non per questo privo di continuità con il pensiero levistraussiano.\par
\textit{Il pensiero selvaggio} si può a buon diritto considerare come il prodotto maturo di un pensatore che già si è confrontato con un'umanità altra rispetto alla civiltà occidentale, ha fatto suo un enorme bagaglio culturale etnografico osservandolo in prima persona, conosce il pensiero, il percorso degli antropologi ed etnologi che prima di lui hanno percorso la sua stessa strada, e solo ora, finalmente, si concede una sintesi originale sul sistema di pensiero che costituisce l'unità minima dell'attività intellettuale umana, desunta dalle ricerche esposte nel monumentale \textit{Le strutture elementari della parentela}, il pensiero allo stato selvaggio.\par
D'altronde, già Lucien Lévy-Bruhl, eminente filosofo e antropologo all'Università Sorbona di Parigi, si era concentrato sulla definizione di una mentalità \enquote{primitiva}, profondamente differente da quella occidentale a causa dell'assenza del principio di non contraddizione, ma dominata invece dal principio di \enquote{partecipazione}; ed anche Émile Durkheim, nel suo ultimo grande lavoro, \textit{Le forme elementari della vita religiosa}, si dedica da una parte allo studio del totemismo, da lui identificato come la forma più primitiva di istituzione religiosa, ma tenta anche, partendo da questa specie di rappresentazioni collettive, di pervenire ad una mentalità primitiva che non solo permette, ma è sottesa alle rappresentazioni collettive di carattere religioso.\par \vspace{0.25cm}
Ciò che lega le opere dell'ultimo Durkheim, di Lévy-Bruhl e di Lévi-Strauss è l'intenzione di indagare le istituzioni sociali per andare a rinvenire un sostrato celato sotto il velo dei contenuti: uno stadio del pensiero non ancora addomesticato, non ancora ammansito attraverso le leggi della logica, il pensiero \enquote{allo stato selvaggio}. Questo è esattamente l'ambizioso progetto di Claude Lévi-Strauss, per quanto viene sviluppato soprattutto ne \textit{Il pensiero selvaggio}: partire dall'antropologia per giungere ad una teoria della mente dell'uomo.\par
Nella Francia del secondo XX secolo, tuttavia, Claude Lévi-Strauss e la sua antropologia debbono contendersi la scena intellettuale con le filosofie \enquote{tradizionali}, meno legate alle scienze umane e alle riflessioni recenti di queste; tra gli esponenti di tali filosofie, un posto di rilievo spetta sicuramente al poliedrico Jean-Paul Sartre.\par
Quest'ultimo, all'epoca dell'uscita delle opere più influenti di Lévi-Strauss, era un intellettuale di spicco, dichiaratamente di sinistra, militante ma critico verso la sinistra internazionale, in quel periodo sconvolta dalla rivelazione dei gesti di Josif Stalin in Unione Sovietica.\par \vspace{0.25cm}
Sartre, scrittore poligrafo estremamente prolifico, si può definire a tutti gli effetti un filosofo (nonostante la sua nota avversione i \textit{philosophes} allievi dell'École Normale): assai celebre e celebrato durante gli anni della Seconda Guerra Mondiale in particolare per \textit{L'Essere e il Nulla}, opera di carattere esistenzialista, nella quale confluivano le idee già elaborate nei romanzi, e maturata in seguito al fecondo incontro con i tedeschi Husserl e Heidegger, dai quali aveva mutuato un'impostazione fenomenologica\footnote{Per un'analisi della fenomenologia sartriana si vedano le sezioni dedicate nei testi introduttivi: \cite{costa2002franzini}, pp.246-250, e anche \cite{macann1993four}, pp.111-158.}.\par
Dopo il capolavoro del '43, celebrato ma anche criticato per il suo carattere pessimista, Sartre si dedica all'elaborazione teorica di una filosofia aperta all'incontro con l'Altro e alla dimensione sociale. L'intenzione sfocia nella pubblicazione della conferenza \textit{L'esistenzialismo è un umanismo} (1943), edito in seguito come volume autonomo, in cui sono ripresentate le tesi de \textit{L'Essere e il Nulla} in una chiave più ottimista e propositiva; e nei \textit{Quaderni per una morale} (annotazioni stese nel 1947-48 ed edite postume), in cui si riconosce l'esigenza di affiancare un'antropologia all'ontologia già descritta in precedenza. Nella \textit{Critica della ragion dialettica} queste intenzioni vanno a concretizzarsi in un'antropologia marxista, e come tale \textit{éngagée}: impegnata nella liberazione dell'individuo dal processo di alienazione messo in atto dai processi di produzione; ma non solo: il progetto di Sartre è teorizzare una fenomenologia della storia, individuandone le strutture fondamentali.\par
Il pensiero sartriano, per utilizzare una categoria schematica e per questo imprecisa, si può far rientrare tra le \enquote{filosofie della riflessione}, dove con questo termine si intende un modo di fare filosofia che predilige il Sé come soggetto indagatore ed oggetto dell'indagine. Allontanandosi dai dati sensibili, limitati per quanto utili, la ragione adotta categorie concettuali all'interno delle quali vigono leggi e forze che possono essere studiate solo attraverso la ragione, per cui questa si erge a soggetto giudice ed oggetto della sua indagine. All'interno della \textit{Critica}, la ragione dialettica, che si manifesta nella Storia, analizza il suo oggetto, la Storia.\par \vspace{0.25cm}
Compito di questo lavoro è confrontare la concezione sartriana di ragione dialettica, concezione radicalmente marxista, con il concetto di ragione analitica elaborato da Lévi-Strauss nella sua antropologia. Come si avrà modo di vedere, entrambi gli autori, campioni a loro modo della loro disciplina in terra francese, si dedicano alla descrizione e all'analisi critica della razionalità dell'uomo, specialmente per come essa si manifesta all'interno della storia e nei gruppi sociali.\par
All'interno dell'opera di Sartre confluiscono diverse componenti: il marxismo, la sua particolare filosofia della storia, l'esistenzialismo, per indicarne alcune; mentre in Lévi-Strauss vi è un marxismo assai differente e caratterizzato, un'avversione ostentata nei confronti della filosofia accademica, le scienze umane \textit{tout court}, la formazione filosofica \textit{ripudiata}.\par
Ciò è sufficiente a suggerire quanto la discussione tra i due intellettuali sia densa di rilievi e zone d'ombra, e quanto, ancora oggi, possa dire sull'intersecarsi tra filosofia e scienze umane. L'opposizione sartriana, che in sé manifesta problematiche ampiamente avvertite nel marxismo \textit{tradizionale}, mette alla prova il paradigma scientifico elaborato dall'antropologia strutturale levistraussiana, e in questo modo le permette di affinare i suoi strumenti e risolvere il rapporto problematico che essa intrattiene con la Storia.\par
Parafrasando Eduardo Viveiros de Castro, e facendo uso di un'immagine suggerita da Michel Izard, le filosofie/antropologie di Sartre e Lévi-Strauss sono \enquote{cannibali}: si contendono lo stesso campo d'indagine, intendendo l'una inglobare l'altra. \par