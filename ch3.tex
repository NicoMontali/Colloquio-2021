\chapter{Alcuni eredi}
\section{Lucien Sebag (1933-1965)}
\epigraph{Lucien Sebag si è tolto la vita il 9 gennaio 1965. Aveva trentun anni. Tentare di individuare oggi le lacune e i contrasti che contrassegnarono la sua esistenza non porterebbe ad altro che a ricostruire la sua vita in funzione della sua fine. [...]\par Lo strutturalismo, nella misura in cui si ricongiungeva con la corrente generale della scienza, apparve a Sebag come la sola alternativa al pensiero marxista; la psicoanalisi e l'etnologia fornirono alla sua riflessione degli argomenti privilegiati. E \textit{Marxismo e strutturalismo}, l'unica opera che egli abbia pubblicato nel corso della sua vita, riflette in termini filosofici questa evoluzione.}{\textsc{Jean-Paul} e \textsc{Marie-Claire Boons}, dalla presentazione dell'articolo \textit{Le mythe, code et message} di \textsc{Lucien Sebag}, in \enquote{Les Temps Modernes}, n.226, pp.1607-1623.}
Allievo di Claude Lévi-Strauss, antropologo e filosofo, Lucien Sebag merita sicuramente un posto di rilievo nel dibattito tra marxismo e strutturalismo, nonostante le sue pubblicazioni siano numericamente piuttosto esigue. Il suo lascito intellettuale è costituito principalmente da due volumi, entrambi fortunatamente tradotti in italiano: \textit{Marxisme et structuralisme}\footnote{\cite{sebag1972marxismo}} e \textit{L'invention du monde chez les Indiens Pueblos}\footnote{Tradotto in italiano come \cite{sebag1979invenzione}}, edito postumo nel 1971. Come indica chiaramente il titolo del primo volume, il pensiero di Sebag si contraddistingue per l'ambizioso tentativo di coniugare due correnti di pensiero assai in voga nella Francia di metà secolo. Del resto è opportuno ricordare che Sebag frequentò la figlia di Jacques Lacan, Judith, ed era paziente del padre di quest'ultima, psicanalista e psichiatra, e quando questa lo rifiutò egli si tolse la vita, suscitando sgomento tra gli intellettuali suoi contemporanei e delusione verso lo psichiatra, colpevole di aver sottovalutato la situazione clinica del paziente, accettandolo nel suo studio senza dedicargli la dovuta cautela e attenzione\footnote{\cite{dosse1998structuralism2}, p.89.}.\par
François Dosse, storico culturale autore di una \textit{Histoire du structuralisme} in due volumi\footnote{\cite{dosse1997structuralism1}; \cite{dosse1998structuralism2}}, dedica al pensiero di Sebag relavitamente poca attenzione, non essendo quest'ultimo uno dei grandi protagonisti del movimento, e cade inoltre nella tentazione di leggere l'opera di Sebag in funzione della sua tragica fine\footnote{\enquote{Many were hopeful that Lucien Sebag the theoretician could modernize Marxism transformed by its relationship with all forms of structuralism. But the book proclaiming the union between Marxism and structuralism also aspired to set the seal on another union, between its author and the woman to whom the book was dedicated: Judith, Lacan's daughter.}, in \cite{dosse1998structuralism2}, p.89. È evidente che questa prospettiva legge l'incompiutezza dell'opera di Sebag in rapporto alla sua vicenda biografica, senza privilegiarne la specificità concettuale.}.\par
Riguardo al progetto filosofico di Sebag, Dosse scrive:
\begin{quote}
    [...] he [Sebag] criticized Marxism for having somewhat fetishized its privileged object and for having underestimated the underlying, immanent principles that organized economic reality, particularly those making it possible to transcend the differences between societies, this "creation of language that defines the very being of culture."! Sebag defended the humanist positions that led him to view structuralism as anthropology and to be skeptical of certain speculative extensions. "Man produces everything that is human, and this tautology prevents us from making structuralism into an extra-anthropological theory about the origin of meaning."\footnote{\cite{dosse1998structuralism2}, p.89.}
\end{quote}
La posizione di Sebag, infatti, che si esemplifica al meglio nella sua opera \textit{Marxismo e strutturalismo} parte proprio dalla constatazione che la \textit{praxis} 



%----------------------------------------------------------------------------------------
\section{Jean Pouillon (1916-2002)}
\epigraph{Lévi-Strauss non è certo il primo, né il solo a sottolineare il carattere strutturale dei fenomeni sociali, ma la sua originalità sta nel prenderlo sul serio e nel trarne imperturbabilmente tutte le conseguenze.}{\textsc{Jean Pouillon}, da \textit{L'oeuvre de Claude Lévi-Strauss} in \enquote{Les Temps Modernes}, 1956, vol. \textsc{xii}, 1956, n. 126, p. 158.}

Lo storico della filosofia François Dosse nel suo \textit{History of the structuralism} in due volumi\footnote{\cite{dosse1997structuralism1} e \cite{dosse1998structuralism2}} definisce Jean Pouillon come \enquote{the man of the middle ground}\footnote{\cite{dosse1997structuralism1}, pp.5-8.}. La definizione, nonostante non sia in grado di riconoscere la specificità del pensiero di Pouillon, rende l'idea del ruolo che quest'ultimo ha assunto in vita: un filosofo passato all'etnologia, un pensatore autonomo, ma anche un intermediario tra Sartre e Lévi-Strauss. Già parte dell'équipe di \textit{Les Temps Modernes} e della redazione della \textit{Nouvelle Revue de psychanalyse}, Pouillon ha ricoperto il ruolo di segretario generale della rivista \textit{L'Homme. Revue française d'anthropologie}, fondata da Lévi-Strauss nel 1961, fino al 1996.\par
Di formazione filosofo, l'avvicinamento di Pouillon all'etnologia avviene in seguito alla lettura di \textit{Tristi Tropici}, di cui è incaricato da Sartre di scrivere una recensione per \textit{Les Temps Modernes}. La richiesta viene accolta entusiasticamente da Pouillon, che si interessa al lavoro dell'etnologo fino a pubblicare un articolo intitolato \enquote{L'œuvre de Claude Lévi-Strauss}.\par
Per Pouillon la conversione dalla filosofia all'etnologia segue un percorso analogo alla conversione vissuta da Lévi-Strauss: la ricerca filosofica attraverso la riflessione si rivela insufficiente, l'insegnamento sartriano, in altri termini, non soddisfa la sensibilità di Pouillon, così l'indagine deve compiere \enquote{il giro lungo}, osservare e confrontarsi con l'infinitamente altro da sé.\par
Per questo Pouillon, su richiesta di Lévi-Strauss, entra a far parte dell'équipe de \textit{L'Homme}, frequentando inoltre il seminario che quest'ultimo presiedeva all'École Pratique des Hautes Études. Lì nel 1960, in occasione della pubblicazione della \textit{Critique de la raison dialectique}, Pouillon tiene una lezione sul pensiero di Sartre, visto il suo trascorso nell'entourage del \textit{philosophe}.\par
Il chiasmo che si viene a creare, la presentazione alla redazione de \textit{Les Temps Modernes} dell'opera di Lévi-Strauss da una parte e, dall'altra, la lezione sul pensiero di Sartre all'École Pratique des Hautes Études, sembra fare di Pouillon il candidato perfetto per sintetizzare il pensiero del \textit{philosophe} da una parte e dell'antropologo dall'altra. Eppure, quando viene chiesto a Pouillon di confrontare la \textit{Critica della ragion dialettica} e \textit{Il pensiero selvaggio} egli definisce i due libri complementari ma incomparabili. Come riporta infatti François Dosse, riprendendo la definizione data da Pouillon nell'intervista con Michel Izàrd:
\begin{quote}
    It was clearly not possible to bring together "these two cannibals" - Sartre and Claude Lévi-Strauss - without running the risk that one would devour the other.
\end{quote}
Dal momento che non può darsi una sintesi tra queste due opere, nonché tra questi due pensatori, come si inserisce in questa prospettiva il pensiero di Jean Pouillon?


nel suo Fetiches sans fetichisme c'è un capitolo dedicato al dibattito


