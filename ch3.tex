\section{Raymond Aron (1905-1983)}
Intellettuale poliedrico, celebre non solo in Francia, già compagno di studi di Jean-Paul Sartre, con il quale intratterrà non senza dissapori un rapporto di amicizia per tutta la vita, e Paul Nizan (1905-1940), i volumi più celebri ed influenti di Aron sono \textit{L'oppio degli intellettuali}\footnote{\cite{aron1958oppio}} e \textit{Le tappe del pensiero sociologico}\footnote{\cite{aron1998tappe}}, ma l'opera che in questa sede sarà analizzata più attentamente è \textit{Histoire et dialectique de la violence}\footnote{\cite{aron1973histoire}}. Il volume, pubblicato nel 1973, costituisce una corposa e polemica risposta alla \textit{Critique} sartriana.

\section{Lucien Sebag (1933-1965)}
\epigraph{Lucien Sebag si è tolto la vita il 9 gennaio 1965. Aveva trentun anni. Tentare di individuare oggi le lacune e i contrasti che contrassegnarono la sua esistenza non porterebbe ad altro che a ricostruire la sua vita in funzione della sua fine. [...]\\Lo strutturalismo, nella misura in cui si ricongiungeva con la corrente generale della scienza, apparve a Sebag come la sola alternativa al pensiero marxista; la psicoanalisi e l'etnologia fornirono alla sua riflessione degli argomenti privilegiati. E \textit{Marxismo e strutturalismo}, l'unica opera che egli abbia pubblicato nel corso della sua vita, riflette in termini filosofici questa evoluzione.}{\textsc{Jean-Paul} e \textsc{Marie-Claire Boons}, dalla presentazione dell'articolo \textit{Le mythe, code et message} di \textsc{Lucien Sebag}, in \enquote{Les Temps Modernes}, n.226, pp. 1607-1623.}

Allievo di Claude Lévi-Strauss, antropologo e filosofo, Lucien Sebag merita sicuramente un posto di rilievo nel dibattito tra marxismo e strutturalismo, nonostante le sue pubblicazioni siano numericamente piuttosto esigue. Il suo lascito intellettuale è costituito principalmente da due volumi, entrambi fortunosamente tradotti in italiano: \textit{Marxisme et structuralisme}\footnote{\cite{sebag1972marxismo}} e \textit{L'invention du monde chez les Indiens Pueblos}\footnote{Tradotto in italiano come \cite{sebag1979invenzione}}, edito postumo nel 1971. Come indica chiaramente il titolo del primo volume, il pensiero di Sebag si contraddistingue per l'ambizioso tentativo di coniugare due correnti di pensiero assai in voga nella Francia di metà secolo. Del resto è opportuno ricordare che Sebag frequentò la figlia di Jacques Lacan, Judith, ed era seguito dal padre in quanto psicanalista e psichiatra.

\section{Jean Pouillon (1916-2002)}
\epigraph{Lévi-Strauss non è certo il primo, né il solo a sottolineare il carattere strutturale dei fenomeni sociali, ma la sua originalità sta nel prenderlo sul serio e nel trarne imperturbabilmente tutte le conseguenze. }{\textsc{Jean Pouillon}, da \textit{L'oeuvre de Claude Lévi-Strauss} in \enquote{Les Temps Modernes}, 1956, vol. \textsc{xii}, 1956, n. 126, p. 158.}

Un altro intellettuale assai interessante nel panorama filosofico francese di metà secolo è Jean Pouillon. Già parte dell'équipe di \textit{Les Temps Modernes} e della redazione della \textit{Nouvelle Revue de psychanalyse}, Pouillon ha ricoperto il ruolo di segretario generale della rivista \textit{L'Homme. Revue française d'anthropologie}, fondata da Lévi-Strauss nel 1961, fino al 1996.

nel suo Fetiches sans fetichisme c'è un capitolo dedicato al dibattito


