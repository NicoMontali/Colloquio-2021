\chapter{Ragione analitica e ragion dialettica}
Come si vedrà ancora più chiaramente nel capitolo successivo, il dibattito tra la filosofia di Jean-Paul Sartre e lo strutturalismo di Claude Lévi-Strauss si configura come una lotta tra cannibali.\par
Perché cannibali? L'immagine, già utilizzata da Michel Izard\footnote{\cite{dosse1997structuralism1}, p.8.}, vuole accentuare da una parte l'opposizione a tratti feroce che si viene a creare tra questi due campioni del pensiero francese, caratterizzati per la predilezione di un metodo d'indagine dialettico contrapposto ad un metodo strutturalista, l'attenzione per la storia contro l'attenzione per le strutture che con variazioni minimali si ripetono all'interno della storia sopravvivendole; dall'altra come si configura tale lotta. L'antinomia sopra descritta, infatti, per essere compresa appieno necessita di un'ulteriore contestualizzazione: sia Sartre che Lévi-Strauss sono intellettuali marxisti, ma se il primo si ispira a Marx per ciò che concerne la concezione dialettica della storia, il suo progetto filosofico trae origini dall'esistenzialismo, e come tale mira a riconoscere l'irriducibile libertà dell'individuo, nonostante quest'ultimo sia calato in una realtà storica che lo condiziona. Sartre, infatti, riconosce l'esistenza di strutture all'interno della storia, ma rivendica la sopravvivenza della libertà del singolo agli eventi, mentre Lévi-Strauss, nonostante riconosca l'importanza della storia e dell'individuo, si concentra sull'individuare strutture che presiedono alla vita collettiva ma anche alle manifestazioni intellettuali del pensiero individuale. Per utilizzare una rappresentazione schematica, e in quanto tale limitata, è come se l'analisi di Sartre partisse dall'interiorità individuale, dotata di assoluta e angosciante libertà, si aprisse alla storia, dominata dalle strutture e dai ricorsi, e poi approdasse di nuovo ad una assoluta libertà consistente nell'intravedere nella storia stessa la libertà dell'individuo, anzi, consistente proprio nella necessità di integrare questa assoluta e angosciante libertà nella visione strutturale della storia; mentre Lévi-Strauss utilizza una prospettiva inversa: una volta riconosciuta all'interno della produzione individuale l'esistenza di strutture che la dominano, strutture rilevabili attraverso un metodo etnologico che integra psicanalisi e strumenti matematici, l'analisi approdasse al riconoscimento di una realtà storica a sua volta da comprendere e superare, per riconoscere finalmente le strutture che sussistono al di sotto della apparentemente arbitraria realtà storica.\par
In questo si potrebbe spiegare la definizione di \enquote{cannibali} attribuita da Izard a Sartre e Lévi-Strauss: due giganti del pensiero che, contendendosi il campo delle scienze umane, cercano di inglobare il metodo dell'altro all'interno della propria proposta metodologica.\par
Del resto, bisogna ricordare, sia Sartre che Lévi-Strauss sono stati, rispettivamente per la generazione del primissimo dopoguerra e per i movimenti della contestazione avvenuti a cavallo tra gli anni '70 e '80, protagonisti di una rivoluzione culturale, che ha fornito alle due generazioni sopracitate gli strumenti per comprendere la propria epoca, nonché per riappropriarsi della possibilità di capire il mondo contemporaneo e riappropriarsene.\par
Subito dopo la fine della Seconda Guerra Mondiale, Jean-Paul Sartre, inizialmente refrattario agli eventi storici, cerca a suo modo di portare la filosofia \enquote{nelle strade}, sia impegnandosi politicamente nel Partito Comunista Francese, sia dedicando opere e \textit{pamphlet} all'emancipazione della classe operaia e al riconoscimento di una sua dignità intellettuale; Lévi-Strauss, con gli altri intellettuali celebrati dal movimento strutturalista, costituisce un attacco alla filosofia degli accademici, ormai colpevolmente distante dalla realtà storica circostante e ridotta ad un vuoto gioco di termini\footnote{\enquote{Ho cominciato allora a capire che tutti i problemi, gravi o futili, possono essere liquidati applicando un metodo sempre identico, che consiste nel contrapporre due punti di vista tradizionali sulla questione; introdurre cioè il primo con le giustificazioni del senso comune, per distruggerlo poi con il secondo; infine rigettarli uno da una parte e uno dall'altra, adottando invece un terzo punto di vista che riveli il carattere ugualmente parziale dei due altri, ricondotti con artifici di vocabolario agli aspetti complementari di una stessa realtà: forma e sostanza, contenente e contenuto, essere e parere, continuo e discontinuo, essenza ed esistenza ecc.} [\cite{levi1960tristi}, p.49]}.\par
Come si è già detto, sia Sartre che Lévi-Strauss sono intellettuali di formazione marxista, ma l'assiduo impegno critico di Sartre all'interno del PCF e sulle pagine della rivista da lui fondata e diretta, \textit{Les temps modernes}, è assai differente dall'impegno marxista levistraussiano, più intensamente vissuto nei gesti e nelle opere di alcuni suoi allievi quali, come si vedrà, Lucien Sebag.\par
%-------------------------------------------------------------------
\section{Jean-Paul Sartre e l'antropologia filosofica}
%-------------------------------------------------------------------
\subsection{Alcune questioni editoriali}Prima di avvicinarsi alla \textit{Critica della Ragion dialettica} occorre chiarire una questione editoriale di non minore importanza: l'opera, estremamente corposa, si articola in due tomi sia nell'edizione francese che in quella italiana. Il primo tomo, \textit{Teoria degli insiemi pratici}, viene edito in Francia da Gallimard nel 1960\footnote{\cite{sartre1960critique}} e in Italia da Il Saggiatore\footnote{\cite{sartre1963critica}} in due volumi separati nel 1963, in un'edizione che vede anteposte alla \textit{Critica} vera e propria le \textit{Questioni di metodo}. Il secondo tomo, \textit{L'intelligibilità della storia}, viene pubblicato nel 1985 a Parigi sempre da Gallimard\footnote{\cite{sartre1985critique}}, mentre in Italia, per la traduzione pubblicata da Marinotti occorre attendere fino al 2006.\par
Il primo volume, edito mentre Sartre è ancora vivente, è un lavoro compiuto nonostante la sua complessità e le difficoltà interpretative che ancora oggi suscita; il secondo volume, invece, è postumo, e come tale non ha potuto ricevere la revisione che gli sarebbe spettata.\par
\subsection{Un'accoglienza problematica}
La \textit{Critique} è accolta con recensioni assai sfavorevoli in terra francese, critiche e aperte prese di posizione polemiche nei confronti del linguaggio e dei contenuti, giudicati oscuri, ridondanti, sospettati di misticismo. Una simile ostilità può essere spiegata a partire dal fatto che il libro di Sartre aveva come bersaglio polemico sia gli intellettuali marxisti, colpevoli di dogmatismo (cfr. \textit{infra}) e mancanza di \textit{engagement}, sia gli intellettuali non marxisti, al servizio di un sistema capitalistico colpevole di trasformare gli uomini in \textit{cose} attraverso un processo di alienazione.\par
Lo scritto è salutato con freddezza a sinistra, dove sarebbe stato logico attendersi un'accoglienza più calorosa, vista la tematica marxista sul quale era incentrato, ma anche a destra, si pensi ad un liberale quale Raymond Aron\footnote{Per un'analisi più attenta delle posizioni di Aron si veda \textit{infra}, \S 3.2}, che a sinistra, nell'ambiente marxista.\par
Dopo un'iniziale diffidenza, però, e soprattutto dopo la morte dell'autore e l'edizione della seconda parte dell'opera, la \textit{Critique} ha ricevuto maggior interesse e interesse, sicuramente dal punto di vista storico-filosofico: oltreoceano Joseph Catalano, già autore di diversi contributi sul pensiero del \textit{philosophe}\footnote{\cite{catalano1985commentary}; \cite{catalano2010body}; \cite{catalano2010reading}; \cite{catalano2021saint}.} parigino, ha dedicato un volume al commento dell'opera\footnote{\cite{catalano1986commentary}}; e anche in Europa si sono moltiplicati i contributi dedicati al marxismo sartriano, come si vedrà in seguito\footnote{Per una lista aggiornata si veda la bibliografia curata da Gabriella Farina in \cite{wormser2005sartre}, pp.XXXXX}.
%---------------------------------------------------------
\subsection{Rinnovare il marxismo.} La \textit{Critica} e le \textit{Questioni di metodo}, come sottolinea nella sua \textit{Introduzione a Sartre} Sergio Moravia\footnote{\cite{moravia1973introduzione}, pp.102-128. Nonostante il lavoro di Moravia sia datato, il suo contributo è prezioso poiché, come si può notare dalla bibliografia, lo studioso si è occupato approfonditamente sia di Sartre che di Lévi-Strauss.}, si pongono l'obiettivo di \enquote{ridestare} il marxismo contemporaneo dal sonno dogmatico in cui si era sopito, sonno provocato da decenni di stalinismo.\par
Lo stalinismo aveva infatti causato la necessità di ripensare gli strumenti concettuali forniti da Marx, \enquote{superare - per utilizzare le parole di Catalano - il pensiero di Marx facendolo incontrare con la storia}\footnote{\enquote{The thought of Marx is not a Platonic form that can be recaptured by a reading of his works.  Indeed, a "personal" attempt to understand the "real Marx" would presuppose a  type of alienation that the writings of Marx could not exhibit, namely, the alienation of a private life encountering history. }, \cite{catalano1986commentary}, p.74.}.\par
D'altra parte, l'originale obiettivo di Marx e del marxismo era recuperare una connessione diretta con il concreto, connessione che l'idealismo tedesco aveva perso riducendo il reale alla sua componente idealistica, e portando avanti il ragionamento sulla realtà ipostatizzata come idea, e non sulla realtà conosciuta attraverso un processo dialettico di reciproco adeguamento tra soggetto percepente e oggetto percepito.\par
L'obiettivo di Sartre è rivitalizzare il marxismo attraverso una attenta disamina critica, svolta applicando il metodo critico agli stessi strumenti critici di cui fa uso il marxismo.\par
Al centro di tale disamina, com'è intuibile, sta la nozione di ragion dialettica, nozione controversa, guardata con sospetto dai filosofi di corrente analitica, e vissuta come problematica anche dalla tradizione continentale.\par
%--------------------------------------------------------------------
\paragraph{La dialettica}
Sarebbe ingenuo pensare di esaurire in una breve definizione la nozione - probabilmente - più problematica del pensiero di Sartre: la dialettica è un oggetto e un metodo (e come tale si presta ad una disamina critica come quella sartriana) che muta continuamente in rapporto al soggetto che ne compie l'analisi. È, in ultima analisi, un processo e la struttura del processo che si manifesta nella sua forma più compiuta nell'azione umana calata nella storia, nella \textit{praxis}.\par
Hervé Vautrelle, all'interno del commentario edito da Ellipses\footnote{\cite{vautrelle2001critique}}, chiarisce perché sia necessario distinguere tra ragione analitica e ragion dialettica.\par
\begin{quote}
    La ragion dialettica rischiara il passato e il presente alla luce dell'avvenire, attraverso \enquote{l'intelligibilità assoluta di una novità irriducibile}, mentre la ragione analitica rapporta il presente e il futuro al passato, dissolvendo ciò che è sconosciuto in ciò che è conosciuto. La ragione. La ragione analitica scopre i legami di esteriorità riguardo ad elementi che sono giustapposti gli uni di fianco agli altri, senza compenetrazione [reciproca]. Per la ragione dialettica, tutti i momenti di un processo prendono senso in rapporto al primo momento, e quest'ultimo si comprende a partire dai momenti che seguono.\footnote{\cite{vautrelle2001critique}, p.54. [traduzione mia]}\par
\end{quote}
Il soggetto, per Sartre, fa esperienza della dialettica all'interno del gruppo: l'azione del singolo è definita \enquote{prassi costituente} mentre l'azione comune del gruppo è detta \enquote{prassi costituita}.\par
Lo scopo di Sartre, del resto, è restituire al soggetto la sua possibilità di intervenire sul corpo sociale, e così facendo produrre la storia, e la dialettica è l'unica metodologia adatta a cogliere la continuità tra l'Essere dell'uomo e il mondo in cui è dato, la compenetrazione tra l'individuo e il sociale, l'unione indissolubile che vi è nel reale tra soggetto ed oggetto.\par
L'urgenza di analizzare l'uomo all'interno di un corpo sociale, e quindi l'esigenza di uscire dall'ontologia pessimista de \textit{L'Essere e il Nulla}, si rintraccia già nei \textit{Quaderni per una morale}, una serie di appunti stesi tra il 1947 e il 1948 e pubblicati postumi nel 1983 dalla figlia Arlette.\par
\begin{quote}
    Quando c’è una pluralità di Altri c’è società. La Società è la prima concrezione che spinge a passare dall’ontologia all’antropologia. Supporre che vi siano stati degli uomini senza società è tanto assurdo quanto supporre che vi siano stati uomini senza linguaggio. La realtà umana sorge in mezzo agli altri. Questo si traduce antropologicamente con: l’uomo esiste in società. E il suo rapporto originario con la società consiste in questo, che non può né fondervisi del tutto né superarla.\footnote{\cite{sartre1983cahiers}, p.124; tr. it. in \cite{sartre2019quaderni}}
\end{quote}
L'interesse di Sartre verso questioni di ordine sociologico e antropologico dipende dal ruolo che egli accorda alla filosofia: il compito di questa è sovrintendere e dirigere le altre scienze umane stabilendone i fini. E il fine ultimo di una filosofia marxista non può che essere quello di liberare l'individuo, riconoscere l'effetto della \textit{praxis} sul mondo, restituendole il suo posto all'interno di una realtà sociale dominata e determinata dai rapporti di produzione, da un sistema capitalistico che annulla l'umanità dell'uomo alienandolo, rendendolo \enquote{cosa}. In questo orizzonte la filosofia non può che aprirsi alle scienze sociali, rifiutando di osservare passivamente la realtà sociale come un'ontologia ed elaborando invece un'antropologia, una concezione dell'uomo a partire dalla sua possibilità d'azione.\par
Diversamente da quanto si ritiene comunemente, l'intento di Sartre non è elaborare una filosofia della storia, ma, per usare le parole di Juliette Simont, proporre una \enquote{assai ampia fenomenologia della storia, della società}\footnote{\cite{simont2000siecles}}. Si può parlare di fenomenologia dal momento che la descrizione del processo storico avviene dal suo interno, dal punto di vista di un soggetto consapevole di essere tale e che, in quanto tale, non pretende di calare dall'alto categorie da lui elaborate perché è consapevole del fatto che gli strumenti concettuali di un singolo, in quanto idee, non possono pretendere di aver maggiore realtà del concreto.\par
E tale fenomenologia non intende cogliere la storia nella sua staticità, ma nel suo farsi, nell'azione umana.\par
%------------------------------------------------------------------
\paragraph{La \textit{penuria}}
Il motore della storia, il movente di tutte le azioni umane, però, non è la dialettica stessa: la dialettica non è che il processo, l'ordine e lo svolgimento degli eventi, che come tale richiede un metodo dialettico e non analitico per essere studiata, e una ragione dialettica e non analitica per applicare tale metodo. Ma il vero motore della storia, ciò che mette in atto gli eventi, ponendo gli uomini nella condizione di confliggere gli uni con gli altri o di organizzarsi in gruppi, ciò che, in altri termini, fornisce i fini alle loro azioni è la \enquote{penuria} - (\textit{raréfaction} in francese.\par
L'individuo, soggetto portatore di un'interiorità soggettiva, si rapporta alla materia, disumana e inorganica, sotto forma di \textit{praxis} e non come \textit{in sé}: la materia si rivela all'esperienza come governata dalle leggi dell'esteriorità, a differenza dell'interiorità del soggetto, l'\textit{in sé}. La penuria è la scarsezza delle risorse messe a disposizione dell'individuo dall'ambiente, motivo per cui è legittimo definire la penuria \enquote{il motore passivo della Storia}.\par
Nonostante l'intenzione di Sartre sia ritrovare un principio di intelligibilità nella Storia, un principio necessario e immutabile, che si possa applicare a qualunque tipo di organismo, egli non riesce ad immaginare un altro tipo di rapporto con l'ambiente, pertanto ammette che\par
\begin{quote}
    [...]malgrado la sua contingenza, la penuria è una relazione umana fondamentale (con la Natura e con gli uomini). In tal senso, bisogna dire che è la penuria a fare di noi \textit{questi} individui producenti \textit{questa} Storia e autodefinentisi come uomini.\footnote{\cite{sartre1963critica}, p.249. Corsivi originali.}\par
\end{quote}
La penuria, in altri termini, è il motivo per cui il soggetto è costretto a uscire dalla sua interiorità soggettiva, irriducibile all'oggettività delle leggi esteriori che dominano l'inumano e l'inorganico, e, così facendo, confrontarsi con il mondo a lui esterno.\par %XXXXXXXXXXXXXXXXXXXXXXXXXXXXXXXXXXXX
La penuria è ovviamente anche il motivo per cui nasce e si impone a tutta la civiltà il mercato: la scarsità di risorse disponibili, l'avidità di alcuni potenti che corrono ad accaparrarsi tali risorse spinge l'uomo a incontrare l'Altro attraverso il commercio. L'analisi sartriana, nonostante
Il contatto con popolazioni extraeuropee, ad esempio, è causato dalla scarsità di risorse, economiche e non. Anche il sorgere della
%-----------------------------------------------------------------
\paragraph{Il metodo passivo-regressivo}

La proposta metodologica elaborata da Sartre all'interno delle \textit{Questioni di metodo} risponde al ruolo che il \textit{philosophe} accorda alla filosofia: sovrintendere e definire il fine delle altre scienze umane, a partire dall'antropologia levistraussiana. 



Non si pensi, però, che Sartre non nutrisse scarsa stima o ammirazione nei confronti dello strutturalismo di Lévi-Strauss: per Sartre il metodo strutturalista merita riconoscimento, a patto che se ne accettino i limiti: \cite{howells1992cambridge}.
In che cosa consiste il \textit{marxismo dogmatico} di cui sono accusati i marxisti? 

Scrive Sartre circa la deriva del marxismo sotto la dittatura stalinista:
\begin{quote}
    Stranamente, il marxismo stalinizzato assume un carattere d'immobilismo tale che un operaio non è più un essere reale che cambia con il mondo, ma un'Idea platonica. Difatti in Platone le Idee sono l'eterno, l'Universale e il Vero. La variazione e l'avvenimento, riflessi confusi di queste \underline{forme statiche}, sono ai margini della Verità. Platone mira a coglierli attraverso i miti. [...] Così, come gli individui e le imprese, il \underline{vissuto} cade nella sfera dell'irrazionale, dell'inutilizzabile, e il teorico lo considera come un \textit{non-significante}.
    L'esistenzialismo reagisce affermando la specificità dell'avvenimento storico che rifiuta di concepire come l'assurda giustapposizione d'un residuo contingente e d'un significato \textit{a priori}. Si tratta di trovare una dialettica docile e paziente che sposi i processi nella \underline{loro} verità e rifiuti la tesi \textit{a priori} per cui tutti i conflitti vissuti oppongono termini contraddittori o anche solo contrari [...].\footnote{\cite{sartre1963critica}, pp.96-97; corsivi originali, sottolineature mie.} %xproblemx di spaziatura
\end{quote} \vspace{-0.4cm}
In questo passaggio Sartre lamenta un'opposizione tra le categorie del marxismo stalinizzato, rigide e incapaci di riflettere la ricchezza e le possibilità d'azione dell'individuo, nonché l'interiorità costituita dal vissuto di quest'ultimo, come il soggetto si inserisce nella contingenza. Il marxismo contemporaneo è colpevole di platonismo in quanto all'operaio, al lavoratore \textit{esistente} si sostituisce l'\textit{essenza} dell'operaio stesso. Ciò è per Sartre imperdonabile: già ne \textit{L'Essere e il Nulla} il \textit{philosophe} parigino aveva sostenuto la priorità dell'esistenza sull'essenza per quanto concerne il formarsi della coscienza:
\begin{quote}
    Il che significa che la coscienza non viene prodotta come esemplare particolare di una possibilità astratta, ma che, invece, scaturendo dal seno dell'essere, crea e sostiene la sua essenza, cioè l'ordinamento sintetico delle sue possibilità.\par
    Ciò vuol dire anche che il tipo di essere della coscienza è l'opposto di quello che ci rivela la prova ontologica: poiché la coscienza non è un possibile prima dell'essere, ma invece il suo essere è la sorgente e la condizione di ogni possibilità, è la sua esistenza che ne implica l'essenza.\footnote{\cite{sartre1943essere}, p.19.}
\end{quote}
e ancor di più, all'interno de \textit{L'esistenzialismo è un umanismo} ne aveva sottolineato le conseguenze morali.
%------------------------------------------------------------------------------------------
\paragraph{\normalfont{Essenza} ed \normalfont{esistenza}}
Per comprendere appieno il significato dei paragrafi precedenti forse conviene fare un passo indietro riferendosi alla conferenza del 1946 \textit{L'esistenzialismo è un umanismo}, in cui le nozioni di soggettività ed esistenza come contrapposta ad essenza hanno chiara definizione: l'antropologia filosofica che si è vista nella sua ricchezza nei paragrafi precedenti prende le sue mosse dal progetto esistenzialista ed umanista per come viene tratteggiato già nella conferenza del 1946. \textit{L'esistenzialismo è un umanismo}, infatti, tratteggia un'antropologia filosofica in quanto punta a definire l'uomo nelle sue possibilità d'azione. Se infatti ne \textit{L'Essere e il Nulla} Sartre aveva elaborato la sua peculiare visione dell'esistenzialismo soprattutto dal punto di vista ontologico e fenomenologico, nel volumetto successivo l'enfasi viene posta sulle conseguenze in sede morale della concezione antropologica già elaborata ne \textit{L'Essere e il Nulla}. Una volta stabilito, infatti, che l'esistenza non segue ma precede l'essenza dell'uomo, la conseguenza etica più rilevante da trarre consiste nella non esistenza di una natura umana come \textit{data}, ma solo in quanto \textit{costruita}.
\begin{quote}
    Così non c'è una natura umana, poiché non c'è un Dio che la concepisca. L'uomo è soltanto, non solo quale si concepisce, ma quale si vuole, e precisamente quale si concepisce dopo l'esistenza e quale si vuole dopo questo slancio verso l'esistere: l'uomo non è altro che ciò che si fa. Questo è il principio primo dell'esistenzialismo. Ed è anche quello che si chiama la soggettività e che ci vien rimproverata con questo stesso termine. Ma che cosa vogliamo dire noi, con questo, se non che l'uomo ha una dignità più grande che non la pietra o il tavolo? Perché noi vogliamo dire che l'uomo in primo luogo esiste, ossia che egli è in primo luogo ciò che si slancia verso un avvenire e ciò che ha coscienza di progettarsi verso l'avvenire.\par
    L'uomo è, dapprima, un progetto che vive se stesso soggettivamente, invece di essere muschio, putridume o cavolfiore; niente esiste prima di questo progetto; niente esiste nel cielo intelligibile; l'uomo sarà anzitutto quello che avrà progettato di essere. Non quello che vorrà essere. Poiché quello che intendiamo di solito con il verbo \enquote{volere} è una decisione cosciente, posteriore, per la maggior parte di noi, a ciò che noi stessi ci siamo fatti.\footnote{\cite{sartre1946esistenzialismo}, pp.51-52.}
\end{quote}
Di qui a dire che l'uomo è assolutamente libero nei confronti delle sue scelte e che in quanto tale ne è l'unico responsabile, il passo è breve. Ciò che però merita maggior attenzione in questo lungo passo è lo statuto di assoluta alterità dell'uomo rispetto agli oggetti che lo circondano. Mentre questi ultimi, infatti, godono di un'esistenza puramente \textit{tecnica}\footnote{\cite{sartre1946esistenzialismo}, pp.47-50.}, e in quanto tale riconducibile ad un'essenza, l'uomo è dotato di una coscienza che lo rende da una parte autore e unico responsabile delle sue azioni e decisioni, dall'altra parte in grado di rendersi conto dell'assoluta libertà con la quale ha compiuto le sue decisioni ed azioni. In questo consiste il duplice valore della soggettività.\par
Ora, dal momento che la scienza sociale levistraussiana, seguendo la lezione di Durkheim, intende trattare i fatti sociali alla stregua delle \textit{cose}, l'obiettivo di Sartre è invece riconoscere la fondamentale ed assoluta libertà che sta alla base della condizione umana e la caratterizza ad ogni suo livello.
%------------------------------------------------------------------------------------
\section{La ragion dialettica}
Concludendo, in che cosa consiste la ragion dialettica? È evidente che essa non consiste \textit{unicamente} nel riconoscere la presenza della dialettica alla base di ogni manifestazione umana, ma anche nella consapevolezza della presenza della dialettica. 

\begin{quote}
    La dialettica come logica vivente dell'azione non può apparire a una ragione contemplativa; essa si rivela in corso di \textit{praxis} e come momento necessario di questa o, se si preferisce, si crea di nuovo in tutte le azioni (benché queste ci appaiano solo sulla base di un mondo interamente costituito dalla \textit{praxis} dialettica del passato) e diventa metodo teorico e pratico quando l’azione in corso di svolgimento si dà i propri lumi. Nel corso dell'azione, l'individuo scopre la dialettica come trasparenza razionale in quanto la fa e come necessità assoluta in quanto gli sfugge, ossia semplicemente in quanto gli altri la fanno; per concludere, proprio nella misura in cui si riconosce nel superamento dei suoi bisogni, l'individuo riconosce la legge che gli altri gli impongono superando i loro bisogni (la riconosce: ciò non vuol dire che vi si sottometta), riconosce la propria autonomia (in quanto può venir utilizzata dall'altro, e lo viene di continuo, mediante finte, manovre, ecc.) come potenza estranea e l’autonomia degli altri come la legge inesorabile che permette di costringerli. Ma, appunto per la reciprocità delle costrizioni e delle autonomie, la legge finisce per sfuggire a tutti e solo nel movimento vorticoso della totalizzazione appare come Ragione dialettica, cioè esterna a tutti perché interna a ciascuno e totalizzazione in corso ma senza totalizzatore di tutte le totalizzazioni totalizzate e di tutte le totalità detotalizzate.\footnote{\cite{sartre1963critica}, p.164.}
\end{quote}




%-----------------------------------------------------------------
\section{Claude Lévi-Strauss: struttura e storia}
\subsection{Il marxismo in antropologia}
A questo punto del percorso non è solo opportuno ma necessario specificare il ruolo che il marxismo svolge nel pensiero di Lévi-Strauss. Confrontando quest'ultimo con Sartre, infatti, ci si accorge che il marxismo sartriano, nonostante l'importante apporto dell'esistenzialismo, è assai più aderente alla tradizione marxista, inoltre il \textit{philosophe} parigino è intenzionato a impegnarsi politicamente e attivamente secondo una prospettiva marxista, mentre in Lévi-Strauss il marxismo ha un apporto tutto sommato tangenziale, tanto che Sergio Moravia ha dubitato della genuinità del marxismo levistraussiano\footnote{\cite{moravia1969ragionenascosta}; per un'analisi più dettagliata vd. anche \cite{mckeon1981marxism}.}.\par
All'inizio di \textit{Tristi Tropici} Lévi-Strauss dichiara da quali discipline abbia tratto ispirazione: la geologia, la psicanalisi e il marxismo\footnote{\cite{levi1960tristi}, pp.55-58.}. Secondo l'antropologo, sia in geologia che in psicanalisi lo studioso si trova \enquote{davanti a fenomeni in apparenza impenetrabili}, per analizzare i quali deve applicare \enquote{qualità raffinate: intuito, sensibilità e gusto}. La storia dello storico, scrive Lévi-Strauss, differisce assai da quella del geologo e dello psicanalista: questi ultimi traggono le loro osservazioni proiettando sul passato modelli validi per la natura del soggetto cui sono applicati, \enquote{certe proprietà fondamentali dell'universo fisico o psichico} validi in quanto \enquote{l'ordine che si stabilisce in un insieme [...] non è né contingente né arbitrario}\footnote{\cite{levi1960tristi}, pp.55.}.\par
Il modello epistemologico che si prospetta differisce profondamente dalle scienze filosofiche per come erano state concepite fino ad allora: contrariamente allo storicismo tradizionale, che vede il filosofo e la ragione storica apparire al termine degli eventi, per lo strutturalismo si tratta di individuare ciò che costituisce il \textit{proprium} dell'uomo, che in quanto tale è rintracciabile in altri fenomeni umani, sia collettivi che individuali, ossia le strutture fondamentali che presiedono ai processi cognitivi. L'ambizione scientifica di fondo, come sottolineato da Francesco Remotti\footnote{\cite{simposio2013}.}, è restituire alle scienze umane lo statuto di scienza tra le altre scienze: in grado, cioè, di elaborare un modello replicabile valido anche per l'uomo.\par
\textit{Elementi di autocritica} di Louis Althusser\footnote{\cite{althusser1975elementi}, pp.23-43.} riconosce alla base di gran parte degli \textit{strutturalisti} una tendenza spinozista. La cifra filosofica comune allo spinozismo consiste proprio nel materialismo radicale con cui si riconosce la natura dell'uomo come \textit{cosa}. È evidente che la parentela tra questo materialismo e il marxismo levistraussiano sia nei fatti superficiale: Lévi-Strauss condivide con il marxismo la forte critica verso il colonialismo, celebra la decolonizzazione come una manifestazione della lotta di classe, ma non prende posizione pubblicamente in favore della classe operaia. Nelle opere della maturità, prima tra tutte \textit{Tristi Tropici}, vi è una forte critica nei confronti della civiltà occidentale, delle sue contraddizioni e delle sue profonde ingiustizie, causa profonda della nascita della disciplina etnologica: 
\begin{quote}
    [...] se l'Occidente ha prodotto degli etnografi è perché un cocente rimorso doveva tormentarlo, obbligandolo a confrontare la sua immagine con quella delle società differenti, nella speranza di vedervi riflesse le stesse tare, o di averne un aiuto per spiegarsi come le proprie si fossero sviluppate.\footnote{\cite{levi1960tristi}, p.377.}
\end{quote}
Per utilizzare una schematizzazione, il debito di Lévi-Strauss nei confronti del marxismo è maggiore del lascito a quest'ultimo: Lévi-Strauss utilizza concetti marxiani senza però curarsi troppo di rimanere all'interno del tracciato di Marx ed Engels. Nel passo che segue, ad esempio, viene rivendicata una certa indipendenza nei confronti del lascito di Marx:
\begin{quote}
    Il marxismo - se non proprio Marx . ha ragionato troppo spesso come se le pratiche dipendessero immediatamente dalla \textit{praxis}. Senza mettere in causa l'incontestabile primato delle infrastrutture, noi crediamo che tra \textit{praxis} e pratiche si inserisca sempre un mediatore, che è lo schema concettuale per opera del quale una materia e una forma, prive entrambe di esistenza indipendente, si adempiono come strutture, ossia come esseri al tempo stesso empirici e intelligibili. Ciò che noi desideriamo è proprio dare il nostro contributo a quella teoria delle sovrastrutture, appena abbozzata da Marx, che riserva alla storia [...] la cura di sviluppare lo studio delle infrastrutture propriamente dette, che non può essere il nostro in modo specifico, dato che l'etnologia è prima di tutto psicologia.\footnote{\cite{levi2010pensiero}, pp.142-143.}
\end{quote}
Lévi-Strauss nutre un profondo rispetto nei confronti dell'opera di Marx, ma la sua intenzione è elaborare il suo pensiero in autonomia, libero dal peso dalla tradizione da cui, nonostante tutto, trae i suoi strumenti e le sue intuizioni.\par

%----------------------------------------------------------------
\subsection{\normalfont{Storia e dialettica}}
Come ricorda Pierre Guenancia\footnote{\cite{guenancia2013fourmis}}, nell'opera di Lévi-Strauss l'ultimo capitolo de \textit{Il pensiero selvaggio} merita un posto a parte. É evidente la distanza concettuale che lo separa dal resto del libro, dedicato ad un'analisi delle forme di pensiero delle popolazioni indigene. Questo scarto è da un lato dovuto al fatto che si tratta delle conclusioni, dall'altro l'autore sta tirando le somme non solo sul volume ma su una buona parte della sua produzione: Il pensiero selvaggio è un'opera della maturità, e l'ultimo capitolo corona una lunga e dettagliata analisi che in corso da diversi anni, che solo ora può esprimersi esplicitamente su che cosa sia il concetto di ragione emergente dall'elenco sterminato di modi del pensiero.\par

LA RAGIONE ANALITICA È QUELLA CHE NON SI LASCIA INFLUENZARE DAL SUO OGGETTO DI ANALISI. PERCHÉ? PERCHÉ COGLIE ALL'INTERNO DELL'OGGETTO CIÒ CHE SFUGGE ALLA DIALETTICA, CIÒ CHE VI È DI STABILE E NON VARIA ALL'APPARIRE DELL'OSSERVATORE

INTRODUZIONE - RAPPORTO TRASVERSALE DI LS CON LA DIALETTICA E LA STORIA

\subsection{\normalfont{Struttura e dialettica}}
Il primo saggio di Antropologia strutturale
vd. anche i saggi in \textit{Antropologia strutturale}, ce ne sono di esplicitamente dedicati a struttura e dialettica

Oggetto di questo paragrafo sarà l'opera \textit{Il pensiero selvaggio}\footnote{\cite{levi2010pensiero}.}, e in particolare l'ultimo capitolo 
Vediamo di fare notevole riferimento all'opera di F. Remotti\footnote{\cite{remotti1971levi}.}

\subsection{Antropologia e marxismo}
Vd. il libro di Wiktor stockowski

Definizione levistraussiana di uomo: vd. \cite{wiseman2009cambridge}, p.20, 23, 42 et segg., 153. vd. Douglas., 
Recupera le conclusioni delle Questioni di metodo: si spiega il rapporto tra antropologia e esistenzialismo.

vd. Dosse: vol.I per quasi tutto c'è un capitolo; vol.3 p.88 e segg., p.112 e segg.