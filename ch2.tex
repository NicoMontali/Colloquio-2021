\section{Jean-Paul Sartre e l'antropologia filosofica}
\subsection{Questioni editoriali}Prima di avvicinarsi alla \textit{Critica della ragion dialettica} occorre chiarire una questione editoriale di non minore importanza: l'opera, estremamente corposa, si articola in due tomi sia nell'edizione francese che in quella italiana. Il primo tomo, \textit{Teoria degli insiemi pratici}, viene edito in Francia da Gallimard nel 1960\footnote{\cite{sartre1960critique}} e in Italia da Il Saggiatore\footnote{\cite{sartre1963critica}} in due volumi separati nel 1963, in un'edizione che vede anteposte alla \textit{Critica} vera e propria le \textit{Questioni di metodo}. Il secondo tomo, \textit{L'intelligibilità della storia}, viene pubblicato nel 1985 a Parigi sempre da Gallimard\footnote{\cite{sartre1985critique}}, mentre in Italia, per la traduzione pubblicata da Marinotti occorre attendere fino al 2006. Perché questa attesa?
Il primo volume, edito mentre Sartre è ancora vivente, è un volume compiuto nonostante la sua complessità e le difficoltà interpretative che ancora oggi suscita; il secondo volume, invece, è postumo, e come tale non ha potuto ricevere la revisione che gli sarebbe spettata.
\subsection{Rinnovare il marxismo.} La \textit{Critica} e le \textit{Questioni di metodo}, come sottolinea nella sua \textit{Introduzione a Sartre} Sergio Moravia, si pone l'obiettivo di "ridestare" il marxismo contemporaneo dal sonno dogmatico in cui si era sopito. L'originale obiettivo di Marx e del marxismo era recuperare una connessione diretta con il concreto, connessione che l'idealismo tedesco aveva perso riducendo il reale alla sua componente idealistica, questo contatto con il concreto, tuttavia, era totalmente assente nel marxismo contemporaneo a Sartre, più impegnato a piegare i fatti per farli rientrare nelle categorie marxiste che ad esaminare criticamente le categorie al fine di aggiornarle.
L'opera di Sartre nasce per operare una vera e propria revisione critica del concetto di dialettica, al fine di riabilitarla. Come evidenzia nel suo agile saggio Nicolas Tertulian\footnote{\cite{tertulian2007ragione}}, la dialettica porta in sè la negazione della negazione, fenomeno che, sottolinea Sartre a più riprese, non può che verificarsi in presenza della componente umana.
È la natura umana, con la libertà e la scelta che in sè comporta, 


\section{Claude Lévi-Strauss: struttura e storia}
Oggetto di questo paragrafo sarà l'opera \textit{Il pensiero selvaggio}\footnote{\cite{levi2010pensiero}.}, e in particolare l'ultimo capitolo 
Vediamo di fare notevole riferimento all'opera di F. Remotti\footnote{\cite{remotti1971levi}.}