\chapter{Ragione analitica e\\ ragion dialettica}
\begin{wrapfigure}{r}{0.30\textwidth}
    \vspace{-0.9cm}
  \begin{center}
    \includegraphics[width=0.28\textwidth]{images/stjohn.png}
  \end{center}
    \vspace{-0.5cm}
  \caption{\textsc{Albrecht Dürer}, \textit{San Giovanni divora il libro}, da l'\textit{Apocalisse}, 1496-98, Staatliche Kunsthalle, Karlsruhe; particolare.}
    \vspace{-0.3cm}
\end{wrapfigure}
\paragraph{Il pensiero cannibale} Nonostante l'evidente violenza dell'immagine, non è esagerato dire che il dibattito tra la filosofia di Jean-Paul Sartre e lo strutturalismo di Claude Lévi-Strauss si configura come una lotta tra cannibali.\par
Perché cannibali? L'immagine, già utilizzata da Michel Izard\footnote{\cite{dosse1997structuralism1}, p.8.}, vuole accentuare da una parte l'opposizione a tratti feroce che si viene a creare tra questi due campioni del pensiero francese, caratterizzati per la predilezione di un metodo d'indagine dialettico contrapposto ad un metodo strutturalista, l'attenzione per la storia contro l'attenzione per le strutture che con variazioni minimali si ripetono all'interno della storia sopravvivendole; dall'altra come si configura tale lotta. L'antinomia sopra descritta, infatti, per essere compresa appieno necessita di un'ulteriore contestualizzazione: sia Sartre che Lévi-Strauss sono intellettuali marxisti, ma se il primo si ispira a Marx per ciò che concerne la concezione dialettica della storia, il suo progetto filosofico trae origini dall'esistenzialismo, e come tale mira a riconoscere l'irriducibile libertà dell'individuo, nonostante quest'ultimo sia calato in una realtà storica che lo condiziona. Sartre, infatti, riconosce l'esistenza di strutture all'interno della storia, ma rivendica la sopravvivenza della libertà del singolo agli eventi, mentre Lévi-Strauss, nonostante riconosca l'importanza della storia e dell'individuo, si concentra sull'individuare strutture che presiedono alla vita collettiva ma anche alle manifestazioni intellettuali del pensiero individuale. Per utilizzare una rappresentazione schematica, e in quanto tale limitata, è come se l'analisi di Sartre partisse dall'interiorità individuale, dotata di assoluta e angosciante libertà, si aprisse alla storia, dominata dalle strutture e dai ricorsi, e poi approdasse di nuovo ad una assoluta libertà consistente nell'intravedere nella storia stessa la libertà dell'individuo, anzi, consistente proprio nella necessità di integrare questa assoluta e angosciante libertà nella visione strutturale della storia; mentre Lévi-Strauss utilizza una prospettiva inversa: una volta riconosciuta all'interno della produzione individuale l'esistenza di strutture che la dominano, strutture rilevabili attraverso un metodo etnologico che integra psicanalisi e strumenti matematici, l'analisi approdasse al riconoscimento di una realtà storica a sua volta da comprendere e superare, per riconoscere finalmente le strutture che sussistono al di sotto della apparentemente arbitraria realtà storica.\par
In questo si potrebbe spiegare la definizione di \enquote{cannibali} attribuita da Izard a Sartre e Lévi-Strauss: due giganti del pensiero che, contendendosi il campo delle scienze umane, cercano di inglobare il metodo dell'altro all'interno della propria proposta metodologica.\par
Del resto, bisogna ricordare, sia Sartre che Lévi-Strauss sono stati, rispettivamente per la generazione del primissimo dopoguerra e per i movimenti della contestazione avvenuti a cavallo tra gli anni '70 e '80, protagonisti di una rivoluzione culturale, che ha fornito alle due generazioni sopracitate gli strumenti per comprendere la propria epoca, nonché per riappropriarsi della possibilità di capire il mondo contemporaneo e riappropriarsene.\par
Subito dopo la fine della Seconda Guerra Mondiale, Jean-Paul Sartre, inizialmente refrattario agli eventi storici, cerca a suo modo di portare la filosofia \enquote{nelle strade}, sia impegnandosi politicamente nel Partito Comunista Francese, sia dedicando opere e \textit{pamphlet} all'emancipazione della classe operaia e al riconoscimento di una sua dignità intellettuale; Lévi-Strauss, con gli altri intellettuali celebrati dal movimento strutturalista, costituisce un attacco alla filosofia degli accademici, ormai colpevolmente distante dalla realtà storica circostante e ridotta ad un vuoto gioco di termini\footnote{\enquote{Ho cominciato allora a capire che tutti i problemi, gravi o futili, possono essere liquidati applicando un metodo sempre identico, che consiste nel contrapporre due punti di vista tradizionali sulla questione; introdurre cioè il primo con le giustificazioni del senso comune, per distruggerlo poi con il secondo; infine rigettarli uno da una parte e uno dall'altra, adottando invece un terzo punto di vista che riveli il carattere ugualmente parziale dei due altri, ricondotti con artifici di vocabolario agli aspetti complementari di una stessa realtà: forma e sostanza, contenente e contenuto, essere e parere, continuo e discontinuo, essenza ed esistenza ecc.} [\cite{levi1960tristi}, p.49]}.\par
Come si è già detto, sia Sartre che Lévi-Strauss sono intellettuali di formazione marxista, ma l'assiduo impegno critico di Sartre all'interno del PCF e sulle pagine della rivista da lui fondata e diretta, \textit{Les temps modernes}, è assai differente dall'ispirazione marxista di Lévi-Strauss, più intensamente sentita da alcuni suoi allievi quali, come si vedrà, Lucien Sebag.\par
%-------------------------------------------------------------------
\section{Jean-Paul Sartre e l'antropologia filosofica}
%-------------------------------------------------------------------
\paragraph{Alcune questioni editoriali}Prima di avvicinarsi alla \textit{Critica della Ragion dialettica} occorre chiarire una questione editoriale di non minore importanza: l'opera, estremamente corposa, si articola in due tomi sia nell'edizione francese che in quella italiana. Il primo tomo, \textit{Teoria degli insiemi pratici}, viene edito in Francia da Gallimard nel 1960\footnote{\cite{sartre1960critique}.} e in Italia da Il Saggiatore\footnote{\cite{sartre1963critica}.} in due volumi separati nel 1963, in un'edizione che vede anteposte alla \textit{Critica} vera e propria le \textit{Questioni di metodo}. Il secondo tomo, \textit{L'intelligibilità della storia}, viene pubblicato nel 1985 a Parigi sempre da Gallimard\footnote{\cite{sartre1985critique}.}, mentre in Italia, per la traduzione pubblicata da Marinotti occorre attendere fino al 2006.\par
Il primo volume, edito mentre Sartre è ancora vivente, è un lavoro compiuto nonostante la sua complessità e le difficoltà interpretative che ancora oggi suscita; il secondo volume, invece, è postumo, e come tale non ha potuto ricevere la revisione che gli sarebbe spettata.\par
\subsection{Un'accoglienza problematica}
La \textit{Critique} è accolta con recensioni assai sfavorevoli in terra francese, critiche e aperte prese di posizione polemiche nei confronti del linguaggio e dei contenuti, giudicati oscuri, ridondanti, sospettati di misticismo. Una simile ostilità può essere spiegata a partire dal fatto che il libro di Sartre aveva come bersaglio polemico sia gli intellettuali marxisti, colpevoli di dogmatismo (cfr. \textit{infra}) e mancanza di \textit{engagement}, sia gli intellettuali non marxisti, al servizio di un sistema capitalistico colpevole di trasformare gli uomini in \textit{cose} attraverso un processo di alienazione.\par
Lo scritto è salutato con freddezza a sinistra, dove sarebbe stato logico attendersi un'accoglienza più calorosa, vista la tematica marxista sul quale era incentrato, ma anche a destra, si pensi ad un liberale quale Raymond Aron\footnote{Per un'analisi più attenta delle posizioni di Aron si veda \textit{infra}, \S 3.2}, che a sinistra, nell'ambiente marxista.\par
Dopo un'iniziale diffidenza, però, e soprattutto dopo la morte dell'autore e l'edizione della seconda parte dell'opera, la \textit{Critique} ha ricevuto maggior interesse e interesse, sicuramente dal punto di vista storico-filosofico: oltreoceano Joseph Catalano, già autore di diversi contributi sul pensiero del \textit{philosophe}\footnote{\cite{catalano1985commentary}; \cite{catalano2010body}; \cite{catalano2010reading}; \cite{catalano2021saint}.} parigino, ha dedicato un volume al commento dell'opera\footnote{\cite{catalano1986commentary}}; e anche in Europa si sono moltiplicati i contributi dedicati al marxismo sartriano, come si vedrà in seguito\footnote{Per una lista aggiornata si veda la bibliografia curata da Gabriella Farina in \cite{wormser2005sartre}, pp.185-204.}.
%---------------------------------------------------------
\subsection{Rinnovare il marxismo.} La \textit{Critica} e le \textit{Questioni di metodo}, come sottolinea nella sua \textit{Introduzione a Sartre} Sergio Moravia\footnote{\cite{moravia1973introduzione}, pp.102-128. Nonostante il lavoro di Moravia sia datato, il suo contributo è prezioso poiché, come si può notare dalla bibliografia, lo studioso si è occupato approfonditamente sia di Sartre che di Lévi-Strauss.}, si pongono l'obiettivo di \enquote{ridestare} il marxismo contemporaneo dal sonno dogmatico in cui si era sopito, sonno provocato da decenni di stalinismo.\par
Lo stalinismo aveva infatti causato la necessità di ripensare gli strumenti concettuali forniti da Marx, \enquote{superare - per utilizzare le parole di Catalano - il pensiero di Marx facendolo incontrare con la storia}\footnote{\enquote{The thought of Marx is not a Platonic form that can be recaptured by a reading of his works.  Indeed, a "personal" attempt to understand the "real Marx" would presuppose a  type of alienation that the writings of Marx could not exhibit, namely, the alienation of a private life encountering history. }, \cite{catalano1986commentary}, p.74.}.\par
D'altra parte, l'originale obiettivo di Marx e del marxismo era recuperare una connessione diretta con il concreto, connessione che l'idealismo tedesco aveva perso riducendo il reale alla sua componente idealistica, e portando avanti il ragionamento sulla realtà ipostatizzata come idea, e non sulla realtà conosciuta attraverso un processo dialettico di reciproco adeguamento tra soggetto che percepisce e oggetto percepito.\par
In che cosa consiste il \enquote{marxismo dogmatico} di cui sono accusati i marxisti? Scrive Sartre circa la deriva del marxismo sotto la dittatura stalinista:
\begin{quote}
    Stranamente, il marxismo stalinizzato assume un carattere d'immobilismo tale che un operaio non è più un essere reale che cambia con il mondo, ma un'Idea platonica. Difatti in Platone le Idee sono l'eterno, l'Universale e il Vero. La variazione e l'avvenimento, riflessi confusi di queste \underline{forme statiche}, sono ai margini della Verità. Platone mira a coglierli attraverso i miti. [...] Così, come gli individui e le imprese, il \underline{vissuto} cade nella sfera dell'irrazionale, dell'inutilizzabile, e il teorico lo considera come un \textit{non-significante}.
    L'esistenzialismo reagisce affermando la specificità dell'avvenimento storico che rifiuta di concepire come l'assurda giustapposizione d'un residuo contingente e d'un significato \textit{a priori}. Si tratta di trovare una dialettica docile e paziente che sposi i processi nella \underline{loro} verità e rifiuti la tesi \textit{a priori} per cui tutti i conflitti vissuti oppongono termini contraddittori o anche solo contrari [...].\footnote{\cite{sartre1963critica}, pp.96-97; corsivi originali, sottolineature mie.}
\end{quote} %\vspace{-0.4cm}
In questo passaggio Sartre lamenta un'opposizione tra le categorie del marxismo stalinizzato, rigide e incapaci di riflettere la ricchezza e le possibilità d'azione dell'individuo, e la realtà storica all'interno di cui il soggetto si inserisce. Il marxismo contemporaneo è colpevole di platonismo in quanto all'operaio, al lavoratore \textit{esistente} si sostituisce l'\textit{essenza} dell'operaio stesso. Ciò è per Sartre imperdonabile: già ne \textit{L'Essere e il Nulla} il \textit{philosophe} parigino aveva sostenuto la priorità dell'esistenza sull'essenza per quanto concerne il formarsi della coscienza:
\begin{quote}
    Il che significa che la coscienza non viene prodotta come esemplare particolare di una possibilità astratta, ma che, invece, scaturendo dal seno dell'essere, crea e sostiene la sua essenza, cioè l'ordinamento sintetico delle sue possibilità.\par
    Ciò vuol dire anche che il tipo di essere della coscienza è l'opposto di quello che ci rivela la prova ontologica: poiché la coscienza non è un possibile prima dell'essere, ma invece il suo essere è la sorgente e la condizione di ogni possibilità, è la sua esistenza che ne implica l'essenza.\footnote{\cite{sartre1943essere}, p.19.}
\end{quote}
e ancor di più, all'interno de \textit{L'esistenzialismo è un umanismo} ne aveva sottolineato le conseguenze morali.
%------------------------------------------------------------------------------------------
\paragraph{Essenza ed esistenza}
Per comprendere appieno il significato dei paragrafi precedenti conviene forse fare un passo indietro riferendosi alla conferenza del 1946, dal titolo \textit{L'esistenzialismo è un umanismo}, in cui le nozioni di soggettività ed esistenza come contrapposta ad essenza hanno chiara definizione: l'antropologia filosofica che si è vista nella sua ricchezza nei paragrafi precedenti prende le sue mosse dal progetto esistenzialista ed umanista per come viene tratteggiato già nella conferenza del 1946. \textit{L'esistenzialismo è un umanismo}, infatti, tratteggia un'antropologia filosofica in quanto punta a definire l'uomo nelle sue possibilità d'azione. Se infatti ne \textit{L'Essere e il Nulla} Sartre aveva elaborato la sua peculiare visione dell'esistenzialismo soprattutto dal punto di vista ontologico e fenomenologico, nel volumetto successivo l'enfasi viene posta sulle conseguenze in sede morale della concezione antropologica già elaborata ne \textit{L'Essere e il Nulla}. Una volta stabilito, infatti, che l'esistenza non segue ma precede l'essenza dell'uomo, la conseguenza etica più rilevante da trarre consiste nella non esistenza di una natura umana come \textit{data}, ma solo in quanto \textit{costruita}.
\begin{quote}
    Così non c'è una natura umana, poiché non c'è un Dio che la concepisca. L'uomo è soltanto, non solo quale si concepisce, ma quale si vuole, e precisamente quale si concepisce dopo l'esistenza e quale si vuole dopo questo slancio verso l'esistere: l'uomo non è altro che ciò che si fa. Questo è il principio primo dell'esistenzialismo. Ed è anche quello che si chiama la soggettività e che ci vien rimproverata con questo stesso termine. Ma che cosa vogliamo dire noi, con questo, se non che l'uomo ha una dignità più grande che non la pietra o il tavolo? Perché noi vogliamo dire che l'uomo in primo luogo esiste, ossia che egli è in primo luogo ciò che si slancia verso un avvenire e ciò che ha coscienza di progettarsi verso l'avvenire.\par
    L'uomo è, dapprima, un progetto che vive se stesso soggettivamente, invece di essere muschio, putridume o cavolfiore; niente esiste prima di questo progetto; niente esiste nel cielo intelligibile; l'uomo sarà anzitutto quello che avrà progettato di essere. Non quello che vorrà essere. Poiché quello che intendiamo di solito con il verbo \enquote{volere} è una decisione cosciente, posteriore, per la maggior parte di noi, a ciò che noi stessi ci siamo fatti.\footnote{\cite{sartre1946esistenzialismo}, pp.51-52.}
\end{quote}
Di qui a dire che l'uomo è assolutamente libero nei confronti delle sue scelte e che in quanto tale ne è l'unico responsabile, il passo è breve. Ciò che però merita maggior attenzione in questo lungo passo è lo statuto di assoluta alterità dell'uomo rispetto agli oggetti che lo circondano. Mentre questi ultimi, infatti, godono di un'esistenza puramente \textit{tecnica}\footnote{\cite{sartre1946esistenzialismo}, pp.47-50.}, e in quanto tale riconducibile ad un'essenza, l'uomo è dotato di una coscienza che lo rende da una parte autore e unico responsabile delle sue azioni e decisioni, dall'altra parte in grado di rendersi conto dell'assoluta libertà con la quale ha compiuto le sue decisioni ed azioni. In questo consiste il duplice valore della soggettività.\par
Ora, dal momento che la scienza sociale levistraussiana, seguendo la lezione di Durkheim, intende trattare i fatti sociali alla stregua delle \textit{cose}, l'obiettivo di Sartre è invece riconoscere la fondamentale ed assoluta libertà che sta alla base della condizione umana e la caratterizza ad ogni suo livello. Una scienza sociale che non parta da questo presupposto, da tali premesse esistenzialiste, è per Sartre contestabile.\par
Sartre è sicuramente a conoscenza della tradizione sociologica francese precedente a Lévi-Strauss, e la sua critica sembra infatti più rivolta a Durkheim che a Lévi-Strauss, in quanto il primo 1) aveva definito i fatti sociali come \enquote{delle cose la cui natura, per quanto malleabile, non è tuttavia modificabile a volontà}\footnote{\cite{durkheim1973breviario}, pp.29.}; 2) aveva attribuito alle rappresentazioni collettive una \enquote{potenza imperativa e coercitiva in virtù della quale s'impongono a [l'individuo]}\footnote{\cite{durkheim1973breviario}, p.52.}; 3) aveva riconosciuto alla base dell'uomo una profonda dicotomia tra il soddisfacimento dei bisogni individuali, legato al lato istintuale ed animale dell'uomo, e la formazione di regole morali, passibili di universalizzazione ed opera di un processo di riflessione emancipato dall'immediata risposta agli stimoli esterni. Durkheim aveva risposto alle critiche più ovvie già a partire dalla seconda edizione, ma probabilmente Sartre intende compiere un'operazione più radicale: riportare la filosofia al centro della speculazione delle scienze sociali, e la filosofia più adatta a restituire al soggetto il suo statuto è l'esistenzialismo.\par
%--------------------------------------------------------------------
\paragraph{La dialettica}
Il concetto di dialettica affonda le sue radici nella tradizione dell'idealismo tedesco, motivo per cui sarebbe ingenuo pretendere di fornirne una definizione\footnote{Un prezioso contributo per lo studio della dialettica è offerto dal volume antologico \cite{burgio2007dialettica}.}. Il compito della presente sezione sarà limitato alla definizione della dialettica in maniera circoscritta alla trattazione nella \textit{Critique} sartriana.\par
Per Sartre la dialettica è sia un metodo che un oggetto: un metodo in quanto contribuisce a concepire il mondo nella maniera più corretta, ossia come formato da una rete di relazioni tra oggetti che in tal modo si compenetrano e si definiscono a vicenda; un oggetto in quanto è la struttura di base del reale, l'ossatura che regge gli eventi e gli oggetti, che viene scoperta quando viene spogliata dai caratteri contingenti degli eventi.\par
Per evitare di cadere nell'idealismo hegeliano, Sartre precisa che la dialettica non è \enquote{una legge celeste che s'impone all'Universo, una forza metafisica che genera da sé il processo storico}, bensì la \enquote{\textit{risultante} dell'affrontarsi dei progetti}: pertanto essa non è una forza trascendente all'uomo che si dà indipendentemente ad esso, all'interno della natura, bensì è una \textit{risultante}, la cui comparsa avviene solo in presenza di un \enquote{progetto}. La dialettica è appannaggio dell'uomo soltanto; non vi è una dialettica nella Natura extraumana\footnote{\cite{tertulian2007ragione}, p.239}.\par
La dialettica è, in ultima analisi, il processo e la forma del processo che si manifesta nella sua forma più compiuta nell'azione umana calata nella storia, nella \textit{praxis}.\par
Hervé Vautrelle, all'interno del proprio commentario al primo volume della critica\footnote{\cite{vautrelle2001critique}}, fornisce una prima distinzione tra ragione analitica e ragion dialettica.\par
\begin{quote}
    La ragion dialettica rischiara il passato e il presente alla luce dell'avvenire, attraverso \enquote{l'intelligibilità assoluta di una novità irriducibile}, mentre la ragione analitica rapporta il presente e il futuro al passato, dissolvendo ciò che è sconosciuto in ciò che è conosciuto. La ragione. La ragione analitica scopre i legami di esteriorità riguardo ad elementi che sono giustapposti gli uni di fianco agli altri, senza compenetrazione [reciproca]. Per la ragione dialettica, tutti i momenti di un processo prendono senso in rapporto al primo momento, e quest'ultimo si comprende a partire dai momenti che seguono.\footnote{\cite{vautrelle2001critique}, p.54. [traduzione mia]}\par
\end{quote}
Il soggetto, per Sartre, fa esperienza della dialettica all'interno del gruppo: l'azione del singolo è definita \enquote{prassi costituente} mentre l'azione comune del gruppo è detta \enquote{prassi costituita}.\par
Lo scopo di Sartre, del resto, è restituire al soggetto la sua possibilità di intervenire sul corpo sociale, e così facendo produrre la storia, e la dialettica è l'unica metodologia adatta a cogliere la continuità tra l'Essere dell'uomo e il mondo in cui è dato, la compenetrazione tra l'individuo e il sociale, l'unione indissolubile che vi è nel reale tra soggetto ed oggetto.\par
L'urgenza di analizzare l'uomo all'interno di un corpo sociale, e quindi l'esigenza di uscire dall'ontologia pessimista de \textit{L'Essere e il Nulla}, si rintraccia già nei \textit{Quaderni per una morale}, una serie di appunti stesi tra il 1947 e il 1948 e pubblicati postumi nel 1983 dalla figlia Arlette.\par
\begin{quote}
    Quando c’è una pluralità di Altri c’è società. La Società è la prima concrezione che spinge a passare dall’ontologia all’antropologia. Supporre che vi siano stati degli uomini senza società è tanto assurdo quanto supporre che vi siano stati uomini senza linguaggio. La realtà umana sorge in mezzo agli altri. Questo si traduce antropologicamente con: l’uomo esiste in società. E il suo rapporto originario con la società consiste in questo, che non può né fondervisi del tutto né superarla.\footnote{\cite{sartre1983cahiers}, p.124; tr. it. in \cite{sartre2019quaderni}}
\end{quote}
L'interesse di Sartre verso questioni di ordine sociologico e antropologico dipende dal ruolo che egli accorda alla filosofia: il compito di questa è sovrintendere e dirigere le altre scienze umane stabilendone i fini. E il fine ultimo di una filosofia marxista non può che essere quello di liberare l'individuo, riconoscere l'effetto della \textit{praxis} sul mondo, restituendole il suo posto all'interno di una realtà sociale dominata e determinata dai rapporti di produzione, da un sistema capitalistico che annulla l'umanità dell'uomo alienandolo, rendendolo \enquote{cosa}. In questo orizzonte la filosofia non può che aprirsi alle scienze sociali, rifiutando di osservare passivamente la realtà sociale come un'ontologia ed elaborando invece un'antropologia, una concezione dell'uomo a partire dalla sua possibilità d'azione.\par
Diversamente da quanto si ritiene comunemente, l'intento di Sartre non è elaborare una filosofia della storia, ma, per usare le parole di Juliette Simont, proporre una \enquote{assai ampia fenomenologia della storia, della società}\footnote{\cite{simont2000siecles}}. Si può parlare di fenomenologia dal momento che la descrizione del processo storico avviene dal suo interno, dal punto di vista di un soggetto consapevole di essere tale e che, in quanto tale, non pretende di calare dall'alto categorie da lui elaborate perché è consapevole del fatto che gli strumenti concettuali di un singolo, in quanto idee, non possono pretendere di aver maggiore realtà del concreto.\par
E tale fenomenologia non intende cogliere la storia nella sua staticità, ma nel suo farsi, nell'azione umana.\par
%---------------------------------------------------------------------------------------------
\paragraph{La \textit{penuria}}
Il motore della storia, il movente di tutte le azioni umane, però, non è la dialettica stessa: la dialettica non è che il processo, l'ordine e lo svolgimento degli eventi, che come tale richiede un metodo dialettico e non analitico per essere studiata, e una ragione dialettica e non analitica per applicare tale metodo. Ma il motore materiale della storia, ciò che mette in atto gli eventi \textit{de facto}, ponendo gli uomini nella condizione di confliggere gli uni con gli altri o di organizzarsi in gruppi, ciò che, in altri termini, fornisce i fini alle loro azioni è la \enquote{penuria} - (\textit{raréfaction} in francese.\par
La condizione di penuria rivela la non-centralità dell'esistenza umana all'interno della natura. La penuria si presenta come la \enquote{negazione dell'uomo da parte della Terra}\footnote{\cite{sartre2006critica}, p.36.}: la natura non è a disposizione dell'uomo e l’uomo se ne serve sottraendo alla natura stessa i mezzi di sussistenza e di produzione delle sue condizioni di vita.\par
La penuria, da chiave di lettura di tanti conflitti umani, assume progressivamente il ruolo di condizione ontologica fondamentale per il sorgere della conflittualità umana, come una sorta di grande rimosso, un evento in grado di ripresentarsi come trasfigurato. L'individuo, soggetto portatore di un'interiorità, si rapporta alla materia disumana e inorganica attraverso la \textit{praxis}: la materia si rivela all'esperienza come governata dalle leggi dell'esteriorità, a differenza dell'interiorità del soggetto, l'\textit{in sé}. La penuria è la scarsezza delle risorse messe a disposizione dell'individuo dall'ambiente, motivo per cui è legittimo definire la penuria \enquote{il motore passivo della Storia}.\par
Nonostante l'intenzione di Sartre sia ritrovare un principio di intelligibilità nella Storia, un principio necessario e immutabile, che si possa applicare a qualunque tipo di organismo, egli non riesce ad immaginare un altro tipo di rapporto con l'ambiente, pertanto ammette che\par
\begin{quote}
    [...]malgrado la sua contingenza, la penuria è una relazione umana fondamentale (con la Natura e con gli uomini). In tal senso, bisogna dire che è la penuria a fare di noi \textit{questi} individui producenti \textit{questa} Storia e autodefinentisi come uomini.\footnote{\cite{sartre1963critica}, p.249, corsivi originali.}\par
\end{quote}\par
La penuria, in altri termini, è il motivo per cui il contatto con l'Altro diventa necessario: è la ragione del commercio con i propri simili e con l'ambiente. Ciò che muove il contatto con l'altro e la materia inerte è il bisogno, sinteticamente definito da Tertulian nel proprio saggio dedicato alla \textit{Critique}: \enquote{Il bisogno è certo un rapporto «univoco e di interiorità», ma non esprime meno la profonda eteronomia della condizione umana, poiché, per colmare la mancanza interiormente avvertita, occorre immergersi nell’esteriorità e trovare in essa i mezzi per il suo appagamento.}\footnote{\cite{tertulian2007ragione}, p.241.}
Le modalità con cui gli uomini si aggregano per rispondere al problema materiale della penuria (intesa in senso sia esistenziale sia in senso materiale) è definita in rapporto al bisogno, motivo di profonda ispirazione marxista, dal momento che vede nell'economia la struttura fondamentale della società.\par
%-----------------------------------------------------------------
\paragraph{La \textit{praxis}}
Una volta definita la necessità del soggetto di integrarsi nel collettivo e di rapportarsi alla materia, Sartre procede a riconoscere la dimensione storica come la dimensione fondamentale in cui l'uomo \textit{si fa}, in cui il soggetto ha la sua possibilità di oggettivarsi attraverso la \textit{praxis} e in cui riceve dall'ambiente esterno gli stimoli che lo portano a compiere determinate scelte.\par
La \textit{praxis}, infatti, è il termine con cui si descrive il passaggio dall'interiorità del soggetto teorizzata nella fase maggiormente \textit{esistenzialista} di Sartre, all'oggettività esterna della Storia, tema centrale negli scritti marxisti di Sartre: la \textit{praxis} si configura come \enquote{un passaggio dall'oggettivo all'oggettivo mediante l'interiorizzazione}. È l'azione di \textit{sentire} qualcosa che permette alla soggettività di volgersi contro se stessa raggiungendo la possibilità di una trasformazione oggettiva. La \textit{praxis} consiste proprio nell'oggettivare la propria soggettività nell'azione, dando vita ad una dialettica con la soggettività altrui:
\begin{quote}
    La \textit{praxis}, infatti, è un passaggio dall'oggettivo all'oggettivo mediante l'interiorizzazione; il progetto, come superamento soggettivo dell'oggettività verso l'oggettività, teso tra le condizioni oggettive dell'ambiente e le strutture oggettive del campo dei possibili, rappresenta \textit{in se stesso} l'unità mobile di soggettività ed oggettività, le due determinazioni cardinali dell'attività. Il soggettivo appare allora come un momento necessario del processo oggettivo.\footnote{\cite{sartre1963critica}, p.81; corsivi originali.}
\end{quote}
Dal passo sopracitato si può vedere come Sartre sia disposto a confrontare l'\enquote{assoluta libertà} dell'individuo con \enquote{le strutture oggettive del campo dei possibili}. Questo, però, non deve far pensare che Sartre stia concedendo alcunché ad una posizione di tipo determinista: riconoscere che esistono \enquote{strutture oggettive del campo dei possibili} non implica che il soggetto sia irrimediabilmente attratto nel campo di tali strutture.\par
Nel suo saggio \textit{Sartrean Structuralism?}, Peter Caws\footnote{\cite{howells1992cambridge}, pp.293-317.} insiste sulla consapevolezza dell'individuo come strumento con cui sottrarsi al determinismo delle strutture:
\begin{quote}
    What Sartre emphasized, in contrast, was the complete lucidity of the conscious subject as free to enter or not into relationships, and the responsibility of the agent for the constitution and maintenance in practice of the group structures to which he or she might belong.\footnote{\cite{howells1992cambridge}, p.293.}
\end{quote}
Ciò che accomuna gli strutturalisti, sostiene Caws, è il fatto di riconoscere all'interno delle relazioni la presenza di strutture - i casi più evidenti sono la linguistica, la produzione letteraria e mitologica, i movimenti della storia - le quali, operando a livello inconscio, condizionano la possibilità d'azione dell'individuo. La concezione del soggetto alla base di questa tesi è l'idea di un soggetto \textit{attraversato} dalle strutture; non \textit{governato} ma \textit{percorso}. Per Sartre quest'idea è inaccettabile: per l'esistenzialista il soggetto è in definitiva l'unico responsabile delle sue scelte: si è detto che non esiste una morale \textit{a priori}, valida \textit{in generale}, ciò significa che è il soggetto, in ultima analisi, a dover decidere delle sue azioni.\par
e esiste una morale condivisa tra gli uomini, sta in definitiva all'individuo la possibilità di scegliere se applicare tale morale, o ergersi, agendo altrimenti, a creatore di una nuova norma morale.\par
Il concetto di \enquote{totalizzazione}, concetto centrale nella \textit{Critique}, nasce proprio dall'idea per cui un soggetto, nel compiere un'azione, si deve confrontare con la totalità dei possibili relativi a quell'azione che sono già occorsi.\par
Nel secondo tomo della \textit{Critique} si fa specificamente riferimento ad un episodio: un incontro di pugilato.
\begin{quote}
    [...] l'intera \textit{boxe} è presente in ogni momento del combattimento come sport e come tecnica, con tutte le qualità umane e tutto il condizionamento materiale (allenamento, stato di salute, ecc.) che essa esige. [...] \enquote{\textit{Del buon pugilato}} significa infatti una \underline{pratica} di combattimento che (in ciascuno degli avversari) va al di là della pura tecnica appresa, pur realizzandola per interno in ogni momento. Anche il minimo movimento dovrà infatti essere un'\textit{invenzione}: \underline{scelta} di colpire di sinistro un avversario che si è scoperto, magari facendo una finta (rischi che si corrono per ignoranza), ecc.\footnote{\cite{sartre2006critica}, p.44.}\par
\end{quote}
Il passo chiarisce come il soggetto, nella sua \textit{praxis} (che sia essa banale e quotidiana o straordinaria), debba confrontarsi con la serie degli eventi a lui precedenti, implicati dall'esistenza di una \enquote{pratica} e di una \enquote{tecnica}, per poi \enquote{inventare} nel momento dell'azione. 


\paragraph{Il metodo passivo-regressivo}


%------------------------------------------------------------------------------------
\section{La ragion dialettica}
Concludendo, in che cosa consiste la ragion dialettica? È evidente che essa non consiste \textit{unicamente} nel riconoscere la presenza della dialettica alla base di ogni manifestazione umana, ma anche nella consapevolezza della presenza della dialettica. 

\begin{quote}
    La dialettica come logica vivente dell'azione non può apparire a una ragione contemplativa; essa si rivela in corso di \textit{praxis} e come momento necessario di questa o, se si preferisce, si crea di nuovo in tutte le azioni (benché queste ci appaiano solo sulla base di un mondo interamente costituito dalla \textit{praxis} dialettica del passato) e diventa metodo teorico e pratico quando l’azione in corso di svolgimento si dà i propri lumi. Nel corso dell'azione, l'individuo scopre la dialettica come trasparenza razionale in quanto la fa e come necessità assoluta in quanto gli sfugge, ossia semplicemente in quanto gli altri la fanno; per concludere, proprio nella misura in cui si riconosce nel superamento dei suoi bisogni, l'individuo riconosce la legge che gli altri gli impongono superando i loro bisogni (la riconosce: ciò non vuol dire che vi si sottometta), riconosce la propria autonomia (in quanto può venir utilizzata dall'altro, e lo viene di continuo, mediante finte, manovre, ecc.) come potenza estranea e l’autonomia degli altri come la legge inesorabile che permette di costringerli. Ma, appunto per la reciprocità delle costrizioni e delle autonomie, la legge finisce per sfuggire a tutti e solo nel movimento vorticoso della totalizzazione appare come Ragione dialettica, cioè esterna a tutti perché interna a ciascuno e totalizzazione in corso ma senza totalizzatore di tutte le totalizzazioni totalizzate e di tutte le totalità detotalizzate.\footnote{\cite{sartre1963critica}, p.164.}
\end{quote}


%-----------------------------------------------------------------
\section{Claude Lévi-Strauss: l'integrazione della storia nella struttura}
\subsection{Il marxismo in antropologia}
A questo punto del percorso non è solo opportuno ma necessario specificare il ruolo che il marxismo svolge nel pensiero di Lévi-Strauss. Confrontando quest'ultimo con Sartre, infatti, ci si accorge che il marxismo sartriano, nonostante l'importante apporto dell'esistenzialismo, è assai più aderente alla tradizione marxista, inoltre il \textit{philosophe} parigino è intenzionato a impegnarsi politicamente e attivamente secondo una prospettiva marxista, mentre in Lévi-Strauss il marxismo ha un apporto tutto sommato tangenziale, tanto che Sergio Moravia ha dubitato della genuinità del marxismo levistraussiano\footnote{\cite{moravia1969ragionenascosta}; per un'analisi più dettagliata vd. anche \cite{mckeon1981marxism}.}.\par
All'inizio di \textit{Tristi Tropici} Lévi-Strauss dichiara da quali discipline abbia tratto ispirazione: la geologia, la psicanalisi e il marxismo\footnote{\cite{levi1960tristi}, pp.55-58.}. Secondo l'antropologo, sia in geologia che in psicanalisi lo studioso si trova \enquote{davanti a fenomeni in apparenza impenetrabili}, per analizzare i quali deve applicare \enquote{qualità raffinate: intuito, sensibilità e gusto}. La storia dello storico, scrive Lévi-Strauss, differisce assai da quella del geologo e dello psicanalista: questi ultimi traggono le loro osservazioni proiettando sul passato modelli validi per la natura del soggetto cui sono applicati, \enquote{certe proprietà fondamentali dell'universo fisico o psichico} validi in quanto \enquote{l'ordine che si stabilisce in un insieme [...] non è né contingente né arbitrario}\footnote{\cite{levi1960tristi}, pp.55.}.\par
Il modello epistemologico che si prospetta differisce profondamente dalle scienze filosofiche per come erano state concepite fino ad allora: contrariamente allo storicismo tradizionale, che vede il filosofo e la ragione storica apparire al termine degli eventi, per lo strutturalismo si tratta di individuare ciò che costituisce il \textit{proprium} dell'uomo, che in quanto tale è rintracciabile in altri fenomeni umani, sia collettivi che individuali, ossia le strutture fondamentali che presiedono ai processi cognitivi. L'ambizione scientifica di fondo, come sottolineato da Francesco Remotti\footnote{\cite{simposio2013}.}, è restituire alle scienze umane lo statuto di scienza tra le altre scienze: in grado, cioè, di elaborare un modello replicabile valido anche per l'uomo.\par
\textit{Elementi di autocritica} di Louis Althusser\footnote{\cite{althusser1975elementi}, pp.23-43.} riconosce alla base di gran parte degli \textit{strutturalisti} una tendenza spinozista. La cifra filosofica comune allo spinozismo consiste proprio nel materialismo radicale con cui si riconosce la natura dell'uomo come \textit{cosa}. È evidente che la parentela tra questo materialismo e il marxismo levistraussiano sia nei fatti superficiale: Lévi-Strauss condivide con il marxismo la forte critica verso il colonialismo, celebra la decolonizzazione come una manifestazione della lotta di classe, ma non prende posizione pubblicamente in favore della classe operaia. Nelle opere della maturità, prima tra tutte \textit{Tristi Tropici}, vi è una forte critica nei confronti della civiltà occidentale, delle sue contraddizioni e delle sue profonde ingiustizie, causa profonda della nascita della disciplina etnologica: 
\begin{quote}
    [...] se l'Occidente ha prodotto degli etnografi è perché un cocente rimorso doveva tormentarlo, obbligandolo a confrontare la sua immagine con quella delle società differenti, nella speranza di vedervi riflesse le stesse tare, o di averne un aiuto per spiegarsi come le proprie si fossero sviluppate.\footnote{\cite{levi1960tristi}, p.377.}
\end{quote}
Per utilizzare una schematizzazione, il debito di Lévi-Strauss nei confronti del marxismo è maggiore del lascito a quest'ultimo: Lévi-Strauss utilizza concetti marxiani senza però curarsi troppo di rimanere all'interno del tracciato di Marx ed Engels. Nel passo che segue, ad esempio, viene rivendicata una certa indipendenza nei confronti del lascito di Marx:
\begin{quote}
    Il marxismo - se non proprio Marx - ha ragionato troppo spesso come se le pratiche dipendessero immediatamente dalla \textit{praxis}. Senza mettere in causa l'incontestabile primato delle infrastrutture, noi crediamo che tra \textit{praxis} e pratiche si inserisca sempre un mediatore, che è lo schema concettuale per opera del quale una materia e una forma, prive entrambe di esistenza indipendente, si adempiono come strutture, ossia come esseri al tempo stesso empirici e intelligibili. Ciò che noi desideriamo è proprio dare il nostro contributo a quella teoria delle sovrastrutture, appena abbozzata da Marx, che riserva alla storia [...] la cura di sviluppare lo studio delle infrastrutture propriamente dette, che non può essere il nostro in modo specifico, dato che l'etnologia è prima di tutto psicologia.\footnote{\cite{levi2010pensiero}, pp.142-143.}
\end{quote}
Lévi-Strauss nutre un profondo rispetto nei confronti dell'opera di Marx, ma la sua intenzione è elaborare il proprio pensiero autonomamente, libero dal peso dalla tradizione da cui, nonostante tutto, trae i suoi strumenti e il suo metodo. Sempre riguardo al passo sopracitato, ad esempio, vi si può vedere come Lévi-Strauss faccia uso della distinzione marxiana tra struttura e infrastruttura per circoscrivere il proprio campo d'indagine a quest'ultima, intesa come mediatrice tra il piano della struttura e le sovrastrutture. Ciò, nonostante l'esplicito richiamo terminologico al pensiero di Marx, non è però sufficiente a vedere nel marxismo una delle grandi fonti d'ispirazione di Lévi-Strauss, tanto che alcuni studiosi hanno rilevato un'ispirazione profondamente idealista nella \enquote{logica del sensibile} levistraussiana.\par
Sergio Moravia, uno dei più attenti studiosi italiani contemporanei a Lévi-Strauss, nella monografia \textit{La ragione nascosta}\footnote{\cite{moravia1969ragionenascosta}.} sottolinea l'ispirazione fondamentalmente a-marxiana della psico-logica di Lévi-Strauss:
\begin{quote}
    Tra le strutture e gli eventi, tra le categorie e i fatti Lévi-Strauss privilegia le strutture e le categorie. Non gli interessano gli elementi del reale, così variabili e (come tali) così insignificanti; gli interessano invece le relazioni, che sono costanti. La sua attenzione, aggiunge altrove, si rivolge alle forme, che sono universali. Ciò che gli preme, sempre, non sono i contenuti, bensì i  principi logico.formali che regolano le opposizioni, gli scarti differenziali esistenti nel tessuto della realtà. Tra le logiche particolari e la logica generale che governa sul piano formale quelle stese logiche particolari, egli preferisce quest'ultima.\footnote{\cite{moravia1969ragionenascosta}, p.404.}
\end{quote}
Del resto non è un caso che sia proprio uno studioso italiano a rilevare la profonda divergenza tra l'opera di Lévi-Strauss e l'opera di Marx: in Italia, infatti, l'antropologia del secondo dopoguerra ha avuto tra i suoi protagonisti Ernesto De Martino, studioso di formazione sostanzialmente storicista. Che fosse uno storicismo idealista (e quindi crociano) prima o uno storicismo marxista in seguito, l'impostazione storicista dell'antropologia italiana ha impedito una corretta ricezione dello strutturalismo levistraussiano, come del resto ricorda Salvatore D'Onofrio\footnote{\cite{donofrio2019leviestrutt}.}, il che potrebbe essere una prima ragione per cui sia proprio italiano uno studioso contemporaneo a Lévi-Strauss a rilevarne la profonda ispirazione non marxista.\par %\vspace{0.5cm}
Uno studioso contemporaneo di formazione francese, invece, Wiktor Stoczkowski, nel suo \textit{Anthropologies rédemptrices}\footnote{\cite{stoczkowski2008anthropologies}.} rileva la distanza tra Lévi-Strauss e il marxismo anche sul versante biografico: l'etnologo, scrive l'autore del saggio, ha alle spalle un passato di militante tra gli intellettuali di sinistra, sulle pagine di riviste quali \textit{La Nouvelle Revue Socialiste} e \textit{L'Ètudiant Socialiste}, ricoprendo inoltre il ruolo di segretatio generale della Fédération nationale des étudiants socialistes. La conoscenza levistraussiana di Marx da parte di Lévi-Strauss, e l'impegno di quest'ultimo all'interno della sinistra intellettuale, pertanto, non si limita alla \enquote{fascinazione} sviluppata per l'opera di Marx a sedici anni, accennata in \textit{Tristi Tropici}, ma giunge a fornire un contributo attivo nel movimento socialista. Questo contributo, però, come rileva Stoczkowski, non è citato nei lavori successivi, e gli articoli d'ispirazione socialista stesi durante il periodo tra il 1925 e il 1945 rivelano un marxismo tutto sommato superficiale:
\begin{quote}
    Il suffit pourtant de s'intéresser de près aux milieux que Lévi-Strauss fréquentait dans les années 1920 et 1930, ou encore de relire les articles qu'il a publiés durant cette période, pour se rendre compte que ses références à Marx étaient alors étonnamment rares, en flagrante contradiction avec les déclarations dont la première date de 1955, c'est-à-dire d'une époque où l'évocation de Marx n'avait plus le même sens qu'un quart de siècle auparavant.\footnote{\cite{stoczkowski2008anthropologies}, p.119.}
\end{quote}
Il motivo di ciò, secondo Stoczkowski, è da cercare nella presenza ricorrente, all'interno degli scritti del periodo 1925-1945, del nome di Henri De Man, politico belga, militante socialista e professore di psicologia sociale all'Università di Francoforte fino al 1933. È a questo intellettuale, accusato in seguito di essere \enquote{l'homme qui assassina Karl Marx}\footnote{\cite{stoczkowski2008anthropologies}, pp.111-137.}, che Lévi-Strauss dedica la sua attenzione e i suoi contributi nel periodo di viva militanza socialista. Ciò arricchisce ulteriormente il ritratto dell'etnologo, fornendo l'immagine di un rapporto con una versione del marxismo tutto sommato superficiale, tra l'adozione eclettica di strumenti concettuali e termini quali \enquote{struttura}, \enquote{infrastruttura} e \enquote{sovrastruttura}, e il rapporto di attivismo intellettuale (in seguito omesso nei resoconti autobiografici)\footnote{\cite{stoczkowski2008anthropologies}, pp.115-119.} con un marxismo più dettato dalla temperie culturale e per questo contingente, ormai superato negli anni della maturità filosofica.\par
%----------------------------------------------------------------
\subsection{\normalfont{Storia e dialettica}}
Come ricorda Pierre Guenancia\footnote{\cite{guenancia2013fourmis}}, nell'opera di Lévi-Strauss l'ultimo capitolo de \textit{Il pensiero selvaggio} merita un posto a parte. É evidente la distanza concettuale che lo separa dal resto del libro, dedicato ad un'analisi delle forme di pensiero delle popolazioni indigene. Questo scarto è da un lato dovuto al fatto che si tratta delle conclusioni, dall'altro l'autore sta tirando le somme non solo sul volume ma su una buona parte della sua produzione: \textit{Il pensiero selvaggio} è un'opera della maturità, e l'ultimo capitolo corona una lunga e dettagliata analisi in corso da diversi anni, che solo ora può esprimersi esplicitamente su che cosa sia il concetto di ragione emergente dall'elenco sterminato di modi del pensiero.\par
Il capitolo si apre con l'esplicito rimando a Jean-Paul Sartre, per segnalare che la scelta terminologica dei capitoli precedenti, soprattutto per quanto riguarda l'uso del termine dialettica, rappresenta una sorta d'\textit{invasione di campo}. Scrive infatti Lévi-Strauss:
\begin{quote}
    [...] in quale misura un pensiero, che sa e che vuole e essere ad un tempo aneddotico e geometrico, può essere ancora chiamato dialettico? Il pensiero selvaggio è totalizzante: in realtà, esso pretende di andar molto più lontano in questo senso di quanto Sartre non conceda alla ragione dialettica, poiché, da un lato, quest'ultima si lascia sfuggire la serialità pura (mentre abbiamo visto come i sistemi classificatori riescano ad integrarla), e, d'altro lato, esclude lo schematismo, nel quale questi sistemi medesimi trovano il loro coronamento. Noi pensiamo che, nell'intransigente rifiuto proprio del pensiero selvaggio - per cui niente di umano (e neppure di vivente) gli può rimanere estraneo - la ragione dialettica scopra il suo vero principio.\footnote{\cite{levi2010pensiero}, p.255.}
\end{quote}
In questo passo, proprio all'inizio dell'ultimo capitolo, si può vedere come lo strutturalismo di Lévi-Strauss si rapporti al pensiero dialettico di Sartre: il primo afferma di sorpassare ed inglobare il secondo in quanto quest'ultimo si lascia sfuggire la serialità, la dimensione sincronica che sta all'interno della dimensione diacronica. Il pensiero allo stato selvaggio di Lévi-Strauss si propone invece di svincolarsi dalla storicità trovando nelle classificazioni ciò che si frappone e media il passaggio da struttura a sovrastruttura - l'infrastruttura - e ritrovando proprio nell'infrastruttura il principio fondamentale alla base del pensiero dialettico studiato da Sartre.\par
Sartre, inoltre, è accusato di aver fatto largo uso nella sua \textit{Critique} di una \enquote{peculiare ragione analitica:  egli definisce, distingue, classifica e contrappone.}\footnote{\cite{levi2010pensiero}, p.256.} Proprio a partire dall'uso fatto da Sartre della ragione analitica Lévi-Strauss rileva un'incompatibilità di fondo tra la ragione dialettica e i suoi stessi frutti, arrivando a sottolinearne il significato - addirittura - equivoco: \enquote{talora [Sartre] contrappone ragione analitica e ragione dialettica come errore e verità, se non addirittura come diavolo e buon Dio; talaltra le due ragioni appaiono complementari: vie diverse conducenti alla stessa verità}\footnote{\cite{levi2010pensiero}, p.255.}. Del resto diversi commentatori sottolineano la poca chiarezza e talvolta l'oscurità della \textit{Critique} sartriana, respingendone in blocco la validità in virtù del suo carattere oscuro.\par
Lévi-Strauss si configura come un lettore della \textit{Critique} accorto e sensibile, tuttavia 1) troppo impegnato a difendere lo statuto di scienza tra le scienze dell'antropologia per riconoscerne gli eventuali limiti e debiti rispetto alla filosofia \textit{tradizionale}; 2) più interessato a evidenziare il carattere di parzialità della ragione dialettica in favore del carattere di \enquote{totalità} della ragione analitica che a trovare un campo condiviso tra ragione analitica e dialettica, in cui queste si possano dire reciprocamente complementari.\par
Questo perché, come già accennato, l'incontro-scontro tra Sartre e Lévi-Strauss è, prima ancora che lo scontro tra due pensieri opposti, lo scontro tra due protagonisti della cultura del loro tempo, impegnati in una battaglia filosofica come nelle loro vicende biografiche: come si vedrà nella sezione dedicata a Jean Pouillon, entrambi questi intellettuali condividono - pertanto si contendono - l'attenzione dell'opinione pubblica, degli intellettuali emergenti negli anni 60, mentre Sartre ha ormai passato il suo periodo di massimo fulgore e Lévi-Strauss è ormai affermato a livello internazionale. Una delle biografe di Sartre, infatti, racconta di una notevole differenza tra il largo seguito con cui il pubblico aveva seguito le lezioni di Jean Pouillon presso il seminario di Lévi-Strauss, mentre invece le \textit{lectures} sartriane precedenti la pubblicazione della \textit{Critique} avevano lasciato notevole perplessità:
\begin{quote}
    Cohen-Solal: "The lecture took place at Rue 44 de Rennes, exactly opposite his own apartment. [Sartre] entered the packed room around six in the evening, carrying a huge folder under his arm. 'I am going to tell you what I am doing now,' he started in a mechanical, hurried tone of voice. And he continued to speak without ever raising his eyes from the text, as if still absorbed in his writing. 'He spoke for three quarters of an hour,' Jean Pouillon remembers, 'one hour, an hour and a quarter, an hour and a half, an hour and three quarters, without ever raising his head. All those who were standing, half of the audience, were exhausted. Some had already crumpled to the floor. [...] It was as if Sartre had completely forgotten about time.' At last Jean Wahl signaled him to stop, and the philosopher picked up his papers and walked to his study as abruptly as he had come".\footnote{\cite{sartre1985annie}, pp.388-389.}
\end{quote}
Sarebbe ingenuo appiattire la lettura dell'opera di questi due grandi pensatori alle vicende biografiche che li vedono direttamente coinvolti, ma sarebbe altrettanto ingenuo tralasciare episodi come questo: irrilevanti forse per la filosofia, ma utili sicuramente per comprendere a che punto si spingesse la rivalità tra i due intellettuali, come le divergenze filosofiche avessero riscontro nella vita quotidiana, negli eventi e negli incontri.\par
Questo aneddoto, pertanto, contribuisce a spiegare l'opposizione radicale tra questi due pensatori, nonché l'ansioso sforzo di inglobare un sistema in un altro: il motivo per cui Lévi-Strauss cita esplicitamente Sartre è proprio il tentativo di rendere la \textit{propria} ragione analitica prima opposta alla ragion dialettica sartriana, poi per rivendicarne il ruolo tutto sommato complementare, ed infine per proclamare il primato della ragione analitica, capace di dare conto \textit{anche} della ragion dialettica, situata al suo interno.
\begin{quote}
    Il termine ragione dialettica cela quindi il continuo sforzo che la ragione analitica deve fare per riformarsi, se pretende di render conto del linguaggio, della società e del pensiero; e la distinzione fra le due ragioni, a nostro modo di vedere, è fondata soltanto sul temporaneo scarto che separa la ragione analitica dall'intelligenza della vita.\footnote{\cite{levi2010pensiero}, p.256.}
\end{quote}
La ragion dialettica, per Lévi-Strauss, è in definitiva la ragione analitica \textit{nel suo farsi}, ossia nel suo continuo aggiornarsi grazie ad un rapporto dialettico che intrattiene con la realtà, motivo per cui \enquote{la ragione dialettica non è, per noi, \textit{qualcosa d'altro rispetto alla} ragione analitica - su cui si fonderebbe l'originalità assoluta di un ordine umano -  bensì \textit{qualcosa di più nella} ragione analitica: la condizione richiesta affinché essa osi affrontare la risoluzione dell'umano in non umano.}\footnote{\cite{levi2010pensiero}, p.256; corsivi originali.}\par
Prima di proseguire il presente lavoro mi preme sottolineare come l'intreccio tra vita privata e pensiero filosofico, tra filosofia e scienze umane, tra esistenzialismo e strutturalismo, conduca da una parte ad interpretare l'opera filosofica degli autori avvalendosi delle loro vicende biografiche, dall'altra a scoprire la ricchezza di un dibattito che, nonostante prenda le vicende umane degli autori come punto di riferimento, si articola come un incontro-scontro di grande rilevanza filosofica e più che mai attuale, che non può essere appiattito in alcun modo all'aspetto aneddotico.\par
La comparsa della \enquote{risoluzione dell'umano in non umano}, infatti, segna l'emergere del profondo antiumanesimo levistraussiano: l'autore accoglie polemicamente l'accusa sartriana di essere un \enquote{esteta} e di \enquote{studiare gli uomini come se fossero formiche}, dal momento che ritiene questo atteggiamento \enquote{essere quello di ogni uomo di scienza}. Il fine delle scienze, e con loro le scienze umane, è infatti per Lévi-Strauss non \textit{costituire} l'uomo ma \textit{dissolverlo}. Coerentemente con il progetto maussiano, che si è visto mirante a riassorbire l'etnologia con le altre scienze umane nella fisica e nella chimica, \enquote{l'analisi etnografica vuole raggiungere delle invarianti, di cui il presente lavoro mostra che si situano talvolta nei punti più imprevisti}\footnote{\cite{levi2010pensiero}, p.257.}.\par
La ricerca di queste \enquote{invarianti} è sufficiente a qualificare la posizione di Lévi-Strauss, come lui stesso ammette, alla stregua di uno scienziato, e non di un etnologo \textit{strictu sensu}. Ciò che permette a Lévi-Strauss di \textit{fare scienza}, rimanendo nell'accezione di questa più aderente alla terminologia levistraussiana, è una concezione dell'uomo diametralmente opposta a quella di Jean-Paul Sartre: per Lévi-Strauss l'uomo è attraversato da strutture, non le crea, non si pone nell'esistenza come una singolarità della natura, come un \textit{imperium in imperio}, un soggetto consapevole del proprio libero arbitrio, in una condizione fondamentalmente di totale ed assoluta libertà.\par
Nello strutturalismo in generale (particolarmente in quello di Lacan, Althusser e Foucault), nonché nella versione elaborata da Lévi-Strauss, si verifica la cosiddetta \enquote{morte dell'uomo}, inteso nell'accezione di soggetto con le caratteristiche già descritte:
\begin{quote}
    [...] non dimentichiamo che il verbo \textit{dissolvere} non implica affatto, (anzi esclude) la distruzione delle parti costitutive del corpo sottoposto all'azione di un altro corpo. La soluzione di un solido in un liquido modifica la disposizione delle molecole del primo; offre anche, spesso, una maniera efficace di metterle in riserva, per recuperarle all'occorrenza e per meglio studiarne le proprietà. [...]
    E il giorno in cui si riuscirà a capire la vita come una funzione della materia inerte, sarà per scoprire che quest'ultima possiede proprietà ben diverse da quelle che le si attribuivano anteriormente.\footnote{\cite{levi2010pensiero}, pp.257-258.}
\end{quote}
Il paragone dell'uomo ad una parte di un composto in soluzione dà idea della profonda separazione tra l'opera di Sartre e l'opera di Lévi-Strauss: se il primo vede l'uomo come un essere fondamentalmente libero, irriducibile ad alcunché, il secondo non solo non teme di \enquote{dissolvere} l'uomo (sempre a condizione di mantenerne l'unicità e le peculiarità singolari), ma il cui obiettivo ultimo è \enquote{capire la vita come una funzione della materia inerte}, dove quest'ultimo aggettivo per Sartre indica il confrontarsi del soggetto con la serialità degli altri individui, prima di tutto, e soltanto in seguito con la materia inerte vera e propria.\par
%Nel suo commentario alla \textit{Critique de la raison dialectique}, Joseph Catalano utilizz
La ragione analitica, sotto questo aspetto, si può vedere come una diretta derivazione della materia inerte, dal momento che studia le strutture elementari della realtà, sociale e non. Come ribadisce anche Lévi-Strauss
\begin{quote}
    Nella nostra prospettiva, di conseguenza, l'io non si oppone all'altro come l'uomo non si oppone al mondo: le \textit{verità apprese attraverso l'uomo} sono \enquote{del mondo}, e per questo sono importanti. Si capisce dunque che noi troviamo nell'etnologia il principio di ogni ricerca, mentre per Sartre essa solleva un problema, nella forma di disagio da superare o di resistenza da ridurre. E infatti, che cosa si può dire dei popoli \enquote{senza storia}, quando si è definito l'uomo in base alla dialettica e la dialettica in base alla storia?\footnote{\cite{levi2010pensiero}, p.258; corsivo mio.}
\end{quote}
Colpisce in questo passo la visione dell'uomo non più al centro della natura, ma di intermediario, utile ad apprendere verità dal carattere trascendente (nonostante non sia esplicitamente dichiarato), indifferenti rispetto alla presenza umana. Come è già stato rilevato da Sergio Moravia, del resto, lo strutturalismo levistraussiano rivela un carattere ibrido, tra idealismo e spinozismo: tenta di ridurre tutte le manifestazioni umane, sociali e non, a una serie di strutture (collocate, seguendo la lezione marxiana, tra le \textit{infrastrutture}); ciò persegue lo scopo di rintracciare una razionalità \enquote{nascosta} dietro alle molteplici e differenti manifestazioni umane scoperte ed analizzate dall'etnologia. Questo breve \textit{excursus} serve a spiegare la ricchezza del passo sopracitato: l'analisi strutturale operata da Lévi-Strauss nel \textit{Pensiero selvaggio} mira a superare la posizione sartriana (e con lui dello storicismo marxista) secondo cui la società occidentale è una fortuita singolarità nella variegata esperienza umana; l'unica in cui si sia manifestata compiutamente la ragione storica, la razionalità intrinseca dell'uomo, che ricalca le medesime strutture di tipo dialettico sia nell'esperienza individuale che collettiva.\par
La celebre suddivisione levistraussiana tra società \enquote{calde} e società \enquote{fredde} è funzionale a ricondurre i due tipi di società, apparentemente opposti, ad un unico insieme del quale rappresentano due poli complementari. Il passo sopracitato, infatti, accusa dichiaratamente Sartre di etnocentrismo, di segnare uno iato tra la società occidentale e le altre società, in quanto nella prima l'evento, il fluire storico degli avvenimenti è stato accolto con favore, mentre le altre società - definite dall'antropologia precedente \enquote{primitive} - si sono opposte al mutamento intrinseco alla storia. Continua infatti il passo:
\begin{quote}
    Talvolta Sartre sembra distinguere due dialettiche: la \enquote{vera}, che sarebbe quella delle società storiche, e una dialettica ripetitiva e a breve scadenza, che egli concede alle società cosiddette primitive pur collocandola molto vicino alla biologia; egli così mette in pericolo tutto il suo sistema, poiché, con il sotterfugio dell'etnografia, che è incontestabilmente una scienza umana e che si dedica allo studio di tali società, il ponte tra uomo e natura, demolito con tanto accanimento, verrebbe a essere surrettiziamente ristabilito.\footnote{\cite{levi2010pensiero}, p.258.}
\end{quote}
La critica si fa tanto radicale da accusare Sartre di riconoscere all'umanità non storica uno statuto dignitario minore: \enquote{[l'umanità] non le appartiene in proprio e che anzi è in funzione del fatto che l'umanità storica la prende a carico}. È da notare come sia acuta la critica mossa da Lévi-Strauss: se non è del tutto scorretto accusare quest'ultimo di aver subordinato l'individuo, e con lui l'uomo, alle strutture, non è nemmeno scorretto dire che Sartre, nel difendere la posizione opposta, finisce per subordinare l'uomo alla storia. Ciò, sia chiaro, non può in alcun modo assomigliare alle istanze anti-umaniste portate avanti dagli strutturalisti.\par
In \textit{Tristi Tropici} il viaggio etnologico si configura prima di tutto come un viaggio all'interno di sé, per scoprire la propria soggettività e la relatività delle proprie convinzioni private attraverso il contatto con ciò che è infinitamente altro. Inoltre, l'antropologia ha goduto della fortunata definizione di \enquote{filosofia dal giro lungo}, intesa come disciplina che trova nel confronto con il diverso la sua risorsa migliore. Ciò non potrebbe essere più calzante nel richiamare il seguito della critica levistraussiana a Sartre: l'errore di quest'ultimo è stato fare uso del Cogito cartesiano, nella convinzione che non solo fosse sufficiente la propria percezione e considerazione di sé a fare della filosofia, ma anche nella convinzione che fosse proprio l'Io l'unico ente adatto a giudicare se stesso.\par
Si noti che, per la terminologia senza dubbio, qui Lévi-Strauss si spoglia dai panni dell'etnologo e dedica a Sartre una critica propriamente filosofia: il \textit{philosophe}, colpevole di una certa \enquote{angustia} di sguardo, ha errato ponendo la sua soggettività come analoga alla soggettività degli altri individui. Il Cogito cartesiano, infatti, è in grado \enquote{accedere all'universale, ma a condizione di restare psicologico e individuale}\footnote{\cite{levi2010pensiero}, p.259. È evidente che non si tratta dell'originale progetto cartesiano, ma della rilettura fornita da Lévi-Strauss.}; l'errore di Sartre è stato proprio partire da premesse cartesiane, caratterizzate dal massimo grado di soggettività, e pretendere di estendere le proprie riflessioni a tutto il genere umano, basandosi unicamente sulla propria esistenza storica.
\begin{quote}
    Chi pretende di installarsi nelle presunte evidenze dell'io non ne viene più fuori. La conoscenza degli uomini sembra talvolta più facile a chi si lasci intrappolare dall'identità personale.\footnote{\cite{levi2010pensiero}, p.259.}
\end{quote}
Ciò che Lévi-Strauss sta affermando, senza utilizzare questa precisa terminologia, è che Sartre da una parte pretende che la sua opera sia \textit{sovra}scientifica (nella misura in cui ha il compito di sovrintendere i fini generali delle scienze umane particolari), dall'altra parte cade nel banale errore di proiettare sulle civiltà esotiche il tipo di coscienza storica (o storicizzata) che si incontra prevalentemente nella società occidentale. L'errore del filosofo, sembra dire Lévi-Strauss, è tentare una goffa invasione di campo che finisce per disperdersi \enquote{nei vicoli ciechi della psicologia sociale} e per cadere nel banale errore dell'etnocentrismo.\par
\begin{quote}
    Cartesio, che voleva fondare una fisica, tagliava fuori l'Uomo dalla Società. Sartre, che pretende di fondare un'antropologia, taglia fuori la sua società dalle altre società.\footnote{\cite{levi2010pensiero}, p.259.}
\end{quote}
L'altro grande errore di cui Sartre cade vittima è ignorare i progressi della recente antropologia: il celebre \enquote{metodo passivo-regressivo} teorizzato dal \textit{philosophe} non è altro, stando a Lévi-Strauss, che il metodo utilizzato dagli etnologi da decenni. Si tratta di un'accusa grave, forse dettata dal fastidio avvertito nei confronti di un'invasione di campo: la pretesa di Sartre di autoproclamarsi con la propria filosofia al di sopra delle altre scienze sociali e di sovrintenderne i fini è avvertita come una mancanza di rispetto, tale da suscitare una reazione così aggressiva.\par




Ma la tesi fondamentale del volume di Lévi-Strauss è riconoscere alla base della mente umana una \enquote{coscienza atemporale} composta dalle relazioni simultanee tra gli elementi di un sistema, non necessariamente una coscienza dialettica nelle caratteristiche delineate da Sartre. La ragione dialettica per come la vede 


Ragione analitica che vede l'oggetto dall'interno



\subsection{\normalfont{Struttura e dialettica}}
Il primo saggio di Antropologia strutturale
vd. anche i saggi in \textit{Antropologia strutturale}, ce ne sono di esplicitamente dedicati a struttura e dialettica
Vediamo di fare notevole riferimento all'opera di F. Remotti\footnote{\cite{remotti1971levi}.}


Definizione levistraussiana di uomo: vd. \cite{wiseman2009cambridge}, p.20, 23, 42 et segg., 153. vd. Douglas., 
Recupera le conclusioni delle Questioni di metodo: si spiega il rapporto tra antropologia e esistenzialismo.

\subsection{La ragion dialettica}

\section{Filosofia ed etnologia}
L'ultimo capitolo de \textit{Il pensiero selvaggio} rappresenta un singolare tentativo di appropriazione del pensiero di un altro intellettuale: sia Sartre che Lévi-Strauss, campioni indiscussi nei propri campi d'indagine, tentano di invadere lo \textit{spazio intellettuale} dell'interlocutore polemico. Nonostante all'inizio del presente lavoro si sia osservato come la sociologia francese - e con lei anche le altre scienze umane - non abbia mai rispettato la divisione tra discipline, diffusa negli altri paesi, nelle parole di Lévi-Strauss si può intravedere un tentativo esplicito di fare filosofia, di estendere l'etnologia all'antropologia filosofica.\par 
Nella sezione precedente, infatti, si è visto come Lévi-Strauss, rispondendo a un filosofo in un libro dall'argomento etnologico, lasci intendere quale sia la sua personale concezione dell'uomo. Non è errato, in base a quanto si è visto, interpretare l'ultimo capitolo del \textit{Pensiero selvaggio} come un'antropologia filosofica, i cui caratteri sono tratti dal progetto antropologico levistraussiano elaborato in particolare ne \textit{Il pensiero selvaggio}. L'umanità descritta in questo 


Alla fin fine il pensiero di Lévi-Strauss fagocita anche il suo avversario polemico più agguerrito.