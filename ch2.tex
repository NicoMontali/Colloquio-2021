\section{Jean-Paul Sartre e l'antropologia filosofica}
\subsection{Alcune questioni editoriali}Prima di avvicinarsi alla \textit{Critica della ragion dialettica} occorre chiarire una questione editoriale di non minore importanza: l'opera, estremamente corposa, si articola in due tomi sia nell'edizione francese che in quella italiana. Il primo tomo, \textit{Teoria degli insiemi pratici}, viene edito in Francia da Gallimard nel 1960\footnote{\cite{sartre1960critique}} e in Italia da Il Saggiatore\footnote{\cite{sartre1963critica}} in due volumi separati nel 1963, in un'edizione che vede anteposte alla \textit{Critica} vera e propria le \textit{Questioni di metodo}. Il secondo tomo, \textit{L'intelligibilità della storia}, viene pubblicato nel 1985 a Parigi sempre da Gallimard\footnote{\cite{sartre1985critique}}, mentre in Italia, per la traduzione pubblicata da Marinotti occorre attendere fino al 2006.
Il primo volume, edito mentre Sartre è ancora vivente, è un lavoro compiuto nonostante la sua complessità e le difficoltà interpretative che ancora oggi suscita; il secondo volume, invece, è postumo, e come tale non ha potuto ricevere la revisione che gli sarebbe spettata.
\subsection{Un'accoglienza problematica}
La \textit{Critique} è accolta con recensioni assai sfavorevoli in terra francese, critiche e aperte prese di posizione polemiche nei confronti del linguaggio e dei contenuti, giudicati oscuri, ridondanti, sospettati di misticismo. Una simile ostilità può essere spiegata a partire dal fatto che il libro di Sartre aveva come bersaglio polemico sia gli intellettuali marxisti, colpevoli di dogmatismo (cfr. \textit{infra}) e mancanza di \textit{engagement}, sia gli intellettuali non marxisti, al servizio di un sistema capitalistico colpevole di trasformare gli uomini in \textit{cose} attraverso un processo di alienazione.
Lo scritto viene accolto con freddezza sia a destra, si pensi a Raymond Aron\footnote{Per un'analisi più attenta della risposta di Aron cfr. \textit{infra}.}, che a sinistra, nell'ambiente marxista.
Dopo un'iniziale diffidenza, però, e soprattutto dopo la morte dell'autore e l'edizione della seconda parte dell'opera, la \textit{Critique} ha ricevuto maggior interesse e credito: oltreoceano Joseph Catalano ha dedicato un intero volume al commento dell'opera\footnote{\cite{catalano1986commentary}}, nonché altri saggi sul pensiero del \textit{philosophe}\footnote{\cite{catalano1985commentary}; \cite{catalano2010body}; \cite{catalano2010reading}; \cite{catalano2021saint}.} parigino; e anche in Europa si sono moltiplicati i contributi dedicati al marxismo sartriano, come si vedrà in seguito\footnote{Per una lista aggiornata si veda la rassegna bibliografica curata da Gabriella Farina in \cite{wormser2005sartre}, pp.XXXXX}.
%---------------------------------------------------------
\subsection{Rinnovare il marxismo.} La \textit{Critica} e le \textit{Questioni di metodo}, come sottolinea nella sua \textit{Introduzione a Sartre} Sergio Moravia\footnote{Nonostante il lavoro di Moravia sia datato, il suo contributo è prezioso poiché, come si può notare dalla bibliografia, questo studioso si è occupato approfonditamente sia di Sartre che di Lévi-Strauss. Per una bibliografia esaustiva del recentemente scomparso storico della filosofia non è ancora disponibile online una lista completa.}, si pone l'obiettivo di \enquote{ridestare} il marxismo contemporaneo dal sonno dogmatico in cui era sopito. L'originale obiettivo di Marx e del marxismo era recuperare una connessione diretta con il concreto, connessione che l'idealismo tedesco aveva perso riducendo il reale alla sua componente idealistica, e portando avanti il ragionamento sulla realtà ipostatizzata come idea, e non sulla realtà conosciuta attraverso un processo dialettico.
L'obiettivo di Sartre è rivitalizzare il marxismo attraverso una attenta disamina critica, svolta applicando il metodo critico agli stessi strumenti critici di cui fa uso il marxismo.
Al centro di tale disamina, com'è facile intuire, sta la nozione di dialettica, nozione controversa, guardata con sospetto dai filosofi di corrente analitica, e vissuta come problematica anche dalla tradizione continentale. 
Per affrontare al meglio lo sviluppo dei temi che si vengono a presentare è bene dedicare qualche parola a descrivere il concetto di \enquote{ragione dialettica}, dal momento che, come si vedrà, Lévi-Strauss utilizza e critica questa nozione.

\paragraph{La dialettica}
Hervé Vautrelle, all'interno del commentario edito da Ellipses\footnote{\cite{vautrelle2001critique}}, chiarisce perché sia necessario distinguere tra ragione analitica e ragion dialettica.
\begin{quote}
    La ragion dialettica rischiara il passato e il presente alla luce dell'avvenire, attraverso \enquote{l'intelligibilità assoluta di una novità irriducibile}, mentre la ragione analitica rapporta il presente e il futuro al passato, dissolvendo ciò che è sconosciuto in ciò che è conosciuto. La ragione. La ragione analitica scopre i legami di esteriorità riguardo ad elementi che sono giustapposti gli uni di fianco agli altri, senza compenetrazione [reciproca]. Per la ragione dialettica, tutti i momenti di un processo prendono senso in rapporto al primo momento, e quest'ultimo si comprende a partire dai momenti che seguono.
\end{quote}

Un ricco volume antologico curato da Alberto Burgio ne ricostruisce alcune declinazioni in autori di capitale importanza\footnote{\cite{burgio2007dialettica}}; tra questi spicca il contributo di Nicolas Tertulian, dedicato alla nozione di dialettica in Sartre.
É da notare che 


Ma cosa c'entra con LS? Per Sartre il marxismo è l'unica antropologia in quanto scienza dell'uomo possibile. L'unica che riconosca all'uomo il suo statuto di individuo libero e che cerchi di restituirgli le sue possibilità di emancipazione dall'alienazione
%-----------------------------------------------------------
\section{Claude Lévi-Strauss: struttura e storia}
Come ricorda l'articolo di Pierre Guenancia\footnote{\cite{guenancia2013fourmis}}, nell'opera di Lévi-Strauss un particolare 
Oggetto di questo paragrafo sarà l'opera \textit{Il pensiero selvaggio}\footnote{\cite{levi2010pensiero}.}, e in particolare l'ultimo capitolo 
Vediamo di fare notevole riferimento all'opera di F. Remotti\footnote{\cite{remotti1971levi}.}
\subsection{Antropologia e marxismo}
Vd. il libro di Wiktor stockowski