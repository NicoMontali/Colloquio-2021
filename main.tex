\documentclass[a4paper, 12pt, twoside,openany]{book}
\usepackage[T1]{fontenc} %come sotto
\usepackage[utf8]{inputenc} %aggiunge caratteri accentati
\usepackage[italian]{babel} %lingua
\usepackage[osf]{libertinus} %font generale del documento
\usepackage{indentfirst} %rientro paragrafo
\pagestyle{plain} %nessun heading o foot particolare
\usepackage[a4paper,top=3cm,bottom=3cm,left=3cm,right=3cm]{geometry} %impaginazione e margini documento
%\usepackage{layaureo} %alternative layout, NECESSARIO A4PAPER IN DOCCLASS
%GESTIONE IMMAGINI E GRAFICHE
\usepackage{graphicx, wrapfig} %gestione immagini e grafiche
\graphicspath{  {./images/}  } %cartella delle immagini
%GESTIONE TITOLI NEL SOMMARIO
\usepackage{tocloft} %rimuove il grassetto dall'indice
\renewcommand{\cfttoctitlefont}{\huge\centerline}
\renewcommand{\cftchapfont}{\large}
%GESTIONE TITOLI PARAGRAFI E SEZIONI
\usepackage{titlesec} %le impostazioni default del package titlesec
%NB no punto finale nei titoli delle sezioni del corpo del testo

\titleformat{\chapter}[display]
{\normalfont\huge\bfseries\scshape\filcenter}{\thechapter}{20pt}{\Huge} %ho rimosso 
\titleformat{\section}          %bfseries dalle voci contrassegnate
{\normalfont\Large\bfseries}{\thesection}{1em}{}         %%%qui
\titleformat{\subsection}[runin]
{\normalfont\large\itshape}{\normalfont\thesubsection}{1em}{}      %%%qui
\titleformat{\subsubsection}[runin]
{\normalfont\normalsize\bfseries}{\thesubsubsection}{1em}{}
\titleformat{\paragraph}[runin]
{\normalfont\normalsize\bfseries}{\theparagraph}{1em}{}
\titleformat{\subparagraph}[runin]
{\normalfont\normalsize\bfseries}{\thesubparagraph}{1em}{}
\titlespacing*{\chapter}{0pt}{50pt}{40pt}
\titlespacing*{\section}{0pt}{3.5ex plus 1ex minus .2ex}{2.3ex plus .2ex}
\titlespacing*{\subsection}{0pt}{3.25ex plus 1ex minus .2ex}{1.5ex plus .2ex}
\titlespacing*{\subsubsection}{0pt}{3.25ex plus 1ex minus .2ex}{1.5ex plus .2ex}
\titlespacing*{\paragraph}{0pt}{3.25ex plus 1ex minus .2ex}{1em}
\titlespacing*{\subparagraph}{\parindent}{3.25ex plus 1ex minus .2ex}{1em}
%GESTIONE ASPETTO DELLE CITAZIONI ESTESE
\usepackage[nottoc]{tocbibind} %include la voce bibliografia nell'indice
\usepackage[autostyle,italian=guillemets]{csquotes} %rende più semplice la gestione di bibtex e permette inoltre di citare cose estese
%SAME
\usepackage{etoolbox} %setta il carattere delle citazioni estese come più piccolo e rimuove il separatore verticale
\AtBeginEnvironment{quote}{\vspace{-\topsep}\small}
\AtEndEnvironment{quote}{\vspace{-\topsep}}
%GESTIONE BIBLIOGRAFIA
%\usepackage{biblatex-chicago} %ottimo pacchetto, molto ricco ma anche complicato da settare
\usepackage[backend=biber,bibstyle=authortitle, firstinits=true,citestyle=verbose-trad1]{biblatex}
\addbibresource{bibliography.bib} %specifica il file della bibliografia
\renewcommand*{\newunitpunct}{\addcomma\space} %separa i campi con virgole e non con punti, come di default
\renewcommand*{\revsdnamepunct}{} %rimuove la virgola tra nome e cognome dell'autore
\renewcommand\mkbibnamefamily[1]{\textsc{#1}} %autori in maiuscoletto
%LINK: INTERNI ED ESTERNI AL DOCUMENTO
%\usepackage{hyperref} %NB da caricare per ultimo, nel caso usare \phantomsection per la bibliografia
%%%%%%%%%%%%%%%%%%%%%%%%%%%%%%%%%%%%%%%%%%%%%%%%%%%%%%%%%%%%%%%%%%%%%%%%
\begin{document}
\begin{titlepage}
    \begin{figure}
            \flushleft\includegraphics[width=0.37\textwidth]{images/orizzontale colore.jpg} %logosns.png è più armonizzato
    \end{figure}
    
    \vspace{0.5cm}
    \begin{center}
        \Huge{Il dibattito Sartre - Lévi-Strauss}
        
        \vspace{0.1cm}
        
        \LARGE{Il concetto levistraussiano di Ragione in un confronto tra filosofia e scienze umane}
        
        \vspace{1.5cm}

        \Large{Nicolò Montali}
        
        \vspace{1.5cm}
        
        \large\textsc{Classe di Lettere e Filosofia}\\
        \textsc{Scuola Normale Superiore}\\
        \textsc{Pisa, a.a. 2020/2021}\\
        \vfill
    \end{center}    
    \flushright\Large
    Rel.: Prof. Lorenzo Bartalesi\\
    Colloquio di passaggio d'anno
\end{titlepage} %inserisce il frontespizio
%\thispagestyle{empty} %sospende la numerazione della pagina
\phantomsection %inserisce l'indice nell'indice
\addcontentsline{toc}{chapter}{Indice}
\tableofcontents %SOMMARIO
%\thispagestyle{empty}

\chapter*{Introduzione} %sospende la numerazione dei capitoli per la sezione
\addcontentsline{toc}{chapter}{\textsc{Introduzione}}
Per capire appieno la profondità del dibattito che si svolse a metà del XX secolo tra Claude Levi-Strauss e Jean-Paul Sartre occorrerebbe molto più di un lavoro di colloquio, dal momento che si tratta di due delle più influenti e originali personalità intellettuali francesi del loro tempo. Per operare nel modo più chiaro possibile una ricostruzione storico-filosofica di tale dibattito, pertanto, è mia intenzione trattare le opere centrali dello scambio tra i due intellettuali - per fare uso di un paragone botanico - come il tronco di un albero, dal quale discendere a ritroso nel tempo fino alle radici e, parimenti, procedere cronologicamente nella storia della filosofia fino a concentrarsi su alcuni rami. Ovviamente, per circoscrivere il lavoro da svolgere, l’attenzione si concentrerà sulle opere e sugli intellettuali maggiormente influenti ed influenzati da Sartre e Lévi-Strauss, ovvero, per tornare a fare uso del paragone botanico, sui rami principali e più prossimi al tronco, e sulle radici appena sotto la superficie. Il lavoro, pertanto, si articolerà in tre capitoli, seguendo una scansione cronologica, anche se il primo e il terzo capitolo, di fatto, costituiscono due metà di un unico discorso.\par
Seguendo l'esempio di Bruno Karsenti\footnote{\cite{proprietà2013karsenti}}, il compito di Jean-Paul Sartre e della sua opera, in particolare i due volumi componenti la monumentale \textit{Critica della Ragion dialettica} sarà quello di fungere da specchio deformante, ossia evidenziare le peculiarità della posizione di Lévi-Strauss attraverso l'adozione di una concezione della ragione diametralmente differente, ma paradossalmente complementare. \par \vspace{0.25cm}
Claude Lévi-Strauss, come si vedrà più approfonditamente in seguito, si può collocare tra quei rari quanto grandi pensatori situati all'intersezione tra due o più discipline, in particolare la filosofia, l'etnologia, l'antropologia e le scienze umane in generale.\par
Questi eclettici, di cui la Francia può vantare un discreto numero, si pensi a Denis Diderot, Pierre Bourdieu, Émile Durkheim\footnote{Non si cada nell'errore di considerare Durkheim esclusivamente un sociologo, per quanto pioniere: animato dallo stesso spirito di Auguste Comte, nel suo lavoro egli intrattiene con le altre scienze umane un rapporto di reciproco arricchimento, riunisce attorno a sé una rete di collaboratori dalle competenze varie per tentare di svolgere un vasto e completo studio della realtà sociale nelle sue molteplici dimensioni (morali, religiosi, giuridici, economici).}, Henri Bergson, ma anche a Henri Poincaré e Gaston Bachelard sul versante scientifico, hanno non solo il grande merito di rinnovare e rivitalizzare la loro disciplina, ma anche l'oneroso compito di ridefinire il lessico e gli strumenti di tale disciplina, aprendo un dibattito di carattere epistemologico che va a ricollocare le scienze umane all'interno del paradigma scientifico dominante.\par \vspace{0.25cm}
Il ruolo di Lévi-Strauss, com'é stato ampiamente sottolineato\footnote{L'opera di Claude Lévi-Strauss e la bibliografia che lo riguarda è sterminata; per una lista aggiornata si rimanda a \cite{levibibliographie}; \cite{abeles2004bibliographie}; \cite{levibibliografia}} è stato capitale nel ridefinire le possibilità interpretative dell'antropologia attraverso l'uso rigoroso del concetto di \enquote{struttura}. Mentre alcuni suoi contemporanei, aderenti anche loro al movimento \textit{strutturalista}, ossia facenti uso della nozione di \enquote{struttura} nel loro campo d'indagine, sono stati guardati con crescente sospetto con il passare degli anni, il lavoro di Lévi-Strauss è stato sicuramente il più foriero di innovazioni nel suo campo di applicazione\footnote{Per una ricostruzione storica e un contributo esegetico dettagliato delle declinazioni che la nozione di \enquote{struttura} ha ricevuto nel corso dei decenni si rinvia a: \cite{moravia1975strutturalismo}, \cite{boudon2020strutturalismo} e ancora \cite{jean1968strutturalismo}.}.\par 
Il presente lavoro, tuttavia, non si concentrerà tanto sulla nozione di \textit{struttura}, che rimane imprescindibile per comprendere l'opera dell'antropologo, quanto invece sulla concezione levistraussiana di ragione, in particolare quanto elaborato ne \textit{La pensée sauvage}. Come ricorda giustamente Pierre Guenancia, l'ultimo capitolo di quest'opera merita un posto a sé, in gran parte svincolato dai capitoli precedenti, ma non per questo privo di continuità con il pensiero levistraussiano.\par
\textit{Il pensiero selvaggio} si può a buon diritto considerare come il prodotto maturo di un pensatore che già si è confrontato con un'umanità altra rispetto alla civiltà occidentale, ha fatto suo un enorme bagaglio culturale etnografico osservandolo in prima persona, conosce il pensiero, il percorso degli antropologi ed etnologi che prima di lui hanno percorso la sua stessa strada, e solo ora, finalmente, si concede una sintesi originale sul sistema di pensiero che costituisce l'unità minima dell'attività intellettuale umana, desunta dalle ricerche esposte nel monumentale \textit{Le strutture elementari della parentela}, il pensiero allo stato selvaggio.\par
D'altronde, già Lucien Lévy-Bruhl, eminente filosofo e antropologo all'Università Sorbona di Parigi, si era concentrato sulla definizione di una mentalità \enquote{primitiva}, profondamente differente da quella occidentale a causa dell'assenza del principio di non contraddizione, ma dominata invece dal principio di \enquote{partecipazione}; ed anche Émile Durkheim, nel suo ultimo grande lavoro, \textit{Le forme elementari della vita religiosa}, si dedica da una parte allo studio del totemismo, da lui identificato come la forma più primitiva di istituzione religiosa, ma tenta anche, partendo da questa specie di rappresentazioni collettive, di pervenire ad una mentalità primitiva che non solo permette, ma è sottesa alle rappresentazioni collettive di carattere religioso.\par \vspace{0.25cm}
Ciò che lega le opere dell'ultimo Durkheim, di Lévy-Bruhl e di Lévi-Strauss è l'intenzione di indagare le istituzioni sociali per andare a rinvenire un sostrato celato sotto il velo dei contenuti: uno stadio del pensiero non ancora addomesticato, non ancora ammansito attraverso le leggi della logica, il pensiero \enquote{allo stato selvaggio}. Questo è esattamente l'ambizioso progetto di Claude Lévi-Strauss, per quanto viene sviluppato soprattutto ne \textit{Il pensiero selvaggio}: partire dall'antropologia per giungere ad una teoria della mente dell'uomo.\par
Nella Francia del secondo XX secolo, tuttavia, Claude Lévi-Strauss e la sua antropologia debbono contendersi la scena intellettuale con le filosofie \enquote{tradizionali}, meno legate alle scienze umane e alle riflessioni recenti di queste; tra gli esponenti di tali filosofie, un posto di rilievo spetta sicuramente al poliedrico Jean-Paul Sartre.\par
Quest'ultimo, all'epoca dell'uscita delle opere più influenti di Lévi-Strauss, era un intellettuale di spicco, dichiaratamente di sinistra, militante ma critico verso la sinistra internazionale, in quel periodo sconvolta dalla rivelazione dei gesti di Josif Stalin in Unione Sovietica.\par \vspace{0.25cm}
Sartre, scrittore poligrafo estremamente prolifico, si può definire a tutti gli effetti un filosofo (nonostante la sua nota avversione i \textit{philosophes} allievi dell'École Normale): assai celebre e celebrato durante gli anni della Seconda Guerra Mondiale in particolare per \textit{L'Essere e il Nulla}, opera di carattere esistenzialista, nella quale confluivano le idee già elaborate nei romanzi, e maturata in seguito al fecondo incontro con i tedeschi Husserl e Heidegger, dai quali aveva mutuato un'impostazione fenomenologica\footnote{Per un'analisi della fenomenologia sartriana si vedano le sezioni dedicate nei testi introduttivi: \cite{costa2002franzini}, pp.246-250, e anche \cite{macann1993four}, pp.111-158.}.\par
Dopo il capolavoro del '43, celebrato ma anche criticato per il suo carattere pessimista, Sartre si dedica all'elaborazione teorica di una filosofia aperta all'incontro con l'Altro e alla dimensione sociale. L'intenzione sfocia nella pubblicazione della conferenza \textit{L'esistenzialismo è un umanismo} (1943), edito in seguito come volume autonomo, in cui sono ripresentate le tesi de \textit{L'Essere e il Nulla} in una chiave più ottimista e propositiva; e nei \textit{Quaderni per una morale} (annotazioni stese nel 1947-48 ed edite postume), in cui si riconosce l'esigenza di affiancare un'antropologia all'ontologia già descritta in precedenza. Nella \textit{Critica della ragion dialettica} queste intenzioni vanno a concretizzarsi in un'antropologia marxista, e come tale \textit{éngagée}: impegnata nella liberazione dell'individuo dal processo di alienazione messo in atto dai processi di produzione; ma non solo: il progetto di Sartre è teorizzare una fenomenologia della storia, individuandone le strutture fondamentali.\par
Il pensiero sartriano, per utilizzare una categoria schematica e per questo imprecisa, si può far rientrare tra le \enquote{filosofie della riflessione}, dove con questo termine si intende un modo di fare filosofia che predilige il Sé come soggetto indagatore ed oggetto dell'indagine. Allontanandosi dai dati sensibili, limitati per quanto utili, la ragione adotta categorie concettuali all'interno delle quali vigono leggi e forze che possono essere studiate solo attraverso la ragione, per cui questa si erge a soggetto giudice ed oggetto della sua indagine. All'interno della \textit{Critica}, la ragione dialettica, che si manifesta nella Storia, analizza il suo oggetto, la Storia.\par \vspace{0.25cm}
Compito di questo lavoro è confrontare la concezione sartriana di ragione dialettica, concezione radicalmente marxista, con il concetto di ragione analitica elaborato da Lévi-Strauss nella sua antropologia. Come si avrà modo di vedere, entrambi gli autori, campioni a loro modo della loro disciplina in terra francese, si dedicano alla descrizione e all'analisi critica della razionalità dell'uomo, specialmente per come essa si manifesta all'interno della storia e nei gruppi sociali.\par
All'interno dell'opera di Sartre confluiscono diverse componenti: il marxismo, la sua particolare filosofia della storia, l'esistenzialismo, per indicarne alcune; mentre in Lévi-Strauss vi è un marxismo assai differente e caratterizzato, un'avversione ostentata nei confronti della filosofia accademica, le scienze umane \textit{tout court}, la formazione filosofica \textit{ripudiata}.\par
Ciò è sufficiente a suggerire quanto la discussione tra i due intellettuali sia densa di rilievi e zone d'ombra, e quanto, ancora oggi, possa dire sull'intersecarsi tra filosofia e scienze umane. L'opposizione sartriana, che in sé manifesta problematiche ampiamente avvertite nel marxismo \textit{tradizionale}, mette alla prova il paradigma scientifico elaborato dall'antropologia strutturale levistraussiana, e in questo modo le permette di affinare i suoi strumenti e risolvere il rapporto problematico che essa intrattiene con la Storia.\par
Parafrasando Eduardo Viveiros de Castro, e facendo uso di un'immagine suggerita da Michel Izard, le filosofie/antropologie di Sartre e Lévi-Strauss sono \enquote{cannibali}: si contendono lo stesso campo d'indagine, intendendo l'una inglobare l'altra. \par %input permette il nesting, \include lo impedisce

\chapter{Le radici del dibattito nelle scienze umane in Francia}
\section{Claude Lévi-Strauss lettore di Marcel Mauss}
Potrebbe sembrare un torto verso la storia delle scienze umane anteporre Marcel Mauss allo zio Émile Durkheim, al quale il nipote è debitore sotto diversi aspetti, ma si vedrà presto il motivo: nella produzione levistraussiana l'interpretazione dell'opera di Mauss si può a buon diritto considerare come un eccezionale punto d'accesso al pensiero di Lévi-Strauss; il pensiero dell'etnologo, esposto dalle parole dell'antropologo funge, per fare uso di una metafora geologica, da marcatore, utile ad individuare gli aspetti originali del pensiero di Claude Lévi-Strauss per come vengono evidenziati da lui stesso nel lavoro di altro. In altri termini, è proprio il commento di Lévi-Strauss che estrapola dall'opera di Mauss i punti sui quali i due intellettuali convergono.
Secondariamente, l'\textit{Introduzione} all'opera di Marcel Mauss da parte di Lévi-Strauss è pubblicata nel 1950, mentre il volume \textit{Il totemismo oggi}, in cui Durkheim figura (tra gli altri) come interlocutore virtuale vede la luce solo nel 1962. Si procederà quindi con questa scansione.
%----------------------------------------------------------------
\subsection{Le forme di classificazione: tra pensiero ed istituzioni sociali.}
Una tappa preliminare ma necessaria all'elaborazione delle teorie esposte da Mauss nel giustamente celebre \textit{Essai sur le don}, consiste nel saggio scritto a quattro mani con Émile Durkheim \textit{De quelques formes primitives de classification}\footnote{\cite{durkheimmaussformes}}. Il saggio, come dice il titolo, analizza alcune forme di classificazione presenti nelle civiltà arcaiche: le fratrie, i totem e i clan. Ma perché le classificazione, un argomento così spiccatamente teoretico, suscita l'interesse dei due antropologi? La tesi alla base del saggio, nonché l'intuizione più feconda dell'antropologia maussiana è il carattere totale di fatti sociali quali le forme di classificazione. Durkheim e Mauss sostengono che vi sia uno stretto rapporto di parentela, un vero e proprio parallelismo, tra le forme di classificazione e le istituzioni sociali.
Ma la loro analisi parte da presupposti da specificare: per gli autori la classificazione si forma non spontaneamente da uno stato del pensiero di compenetrazione e indistinguibilità tra gli elementi; il pensiero dei primitivi, non diversamente che per Lévy-Bruhl, è caratterizzato dal \enquote{principio di partecipazione}, uno stato di non distinzione tra un elemento e un altro, in cui tutti gli oggetti si assomigliano e fanno parte di un unico tutto\footnote{\enquote{In quel punto la coscienza non è che un flusso continuo di rappresentazioni che si perdono le une nelle altre, e quando le differenziazioni cominciano ad apparire, sono tutt'affatto frammentarie}. \cite{durkheimmaussformes}, p.23.}. In questa prospettiva la storia della logica, almeno per quanto riguarda i concetti di generi e specie, è da rifondare: tali concetti sono soggetti ad uno sviluppo storico, non si trovano in natura e non sono dati nella loro interezza; sono invece prodotti di uno sviluppo che coinvolge più elementi eterogenei, innestandosi in un processo che se fosse lasciato alla sua spontaneità non porterebbe alla nascita di concetti quali generi e specie. Ma si veda ciò attraverso le parole di Durkheim e Mauss:
\begin{quote}
    Non soltanto la nostra nozione attuale della classificazione ha una storia, ma questa stessa storia presuppone una considerevole preistoria. In realtà, lo spirito umano ha preso le mosse da uno stato di massima indistinzione; ancor oggi c'è tutta una parte della nostra letteratura popolare, dei nostri miti, delle religioni che si basa su una fondamentale confusione di ogni immagine e idea. Si potrebbe affermare che immagini o idee separate le une dalle alte con una certa chiarezza, non ce ne siano.\footnote{\cite{durkheimmaussformes}, p.21.}[...]
    \\Non è vero, dunque, che l'uomo classifichi spontaneamente e per una sorta di necessità naturale: agli inizi, fanno difetto all'umanità anche le condizioni indispensabili alla funziona classificatrice. E poi basta analizzare l'idea di classificazione per comprendere che l'uomo non poteva trovare in se stesso gli elementi essenziali. Una classe è un gruppo di cose; orbene, le cose non si offrono all'osservazione di per se stesse raggruppate.\footnote{\cite{durkheimmaussformes}, p.23.}
\end{quote}
Dare il giusto peso a queste parole significa riconoscere che la classificazione non è un fatto innato, appartenente all'ordine naturale delle cose, ma un dato culturale, che si va a manifestare unicamente ove vi sia la civiltà e le sue istituzioni. Ciò che prima poteva sembrare una facoltà che spontaneamente nata dallo sviluppo autonomo della ragione, si scopre ora essere il risultato di un processo storico, appreso ed insegnato, ed in quanto storico frutto di uno sviluppo diacronico.
Ora, se si prosegue nella lettura sarà facile capire le intenzioni degli antropologi: \enquote{L'importanza di questa classificazione è tale che si estende a tutti i fatti della vita e se ne ritrova la traccia in tutti i riti principali}\footnote{\cite{durkheimmaussformes}, p.28.}. Quest'affermazione significa che tra i fatti sociali e l'individuo vi è una reciproca influenza e determinazione. Per riutilizzare le parole di Bruno Karsenti, \enquote{si tratta di cogliere l' "uomo tutto intero", o l' "uomo totale", e al tempo stesso di inscrivere la sociologia "nell'antropologia". Su un piano strettamente sociologico, si tratta sempre di cogliere l'essenza dei rapporti sociali, di raggiungere il punto in cui questi si intrecciano concretamente e si esprimono in una totalità; o ancora, secondo un concetto cui Mauss deve gran parte della sua celebrità, si tratta di cogliere il "fatto sociale totale".}\footnote{\cite{karsenti1997uomo}, p.33.}
Il motivo di questo rapido \textit{excursus} è presto spiegato: una volta evidenziata la natura \enquote{totale} di alcuni fatti, tra i quali le forme di classificazione, è possibile osservare i fatti sociali da un'altra prospettiva, suggestiva e illuminante. Del resto, è noto che Durkheim nell'ultima parte della sua produzione si sia dedicato alla sociologia religiosa, ma è bene sottolineare che il fine di questa svolta fosse lo sviluppo di una teoria della conoscenza dal punto di vista sociologico.
Si prenda un estratto da \textit{Le forme elementari della religione} di Émile Durkheim (1912), dichiaratamente debitore all'articolo scritto a quattro mani con il nipote Marcel Mauss:
\begin{quote}
   [...] queste classificazioni sistematiche sono le prime che incontriamo nella storia; ora, si è visto che esse si sono foggiate sull’organizzazione sociale, o piuttosto che hanno preso come schemi i quadri stessi della società. Le fratrie hanno servito da generi, e i clan da specie. Poiché erano raggruppati gli uomini, essi hanno potuto raggruppare le cose; per classificare queste ultime si sono limitati a fare loro posto nei gruppi che formavano essi stessi. E se queste diverse classi di cose non sono state semplicemente giustapposte le une alle altre, ma sono state ordinate secondo un piano unitario, ciò è accaduto perché i gruppi sociali con cui esse si confondono sono anch’essi solidali e formano con la loro unione un tutto organico, la tribù. L’unità di questi primi sistemi logici non fa che riprodurre l’unità della società.\footnote{\cite{durkheim2013forme}, p.201.}
\end{quote}
Il passo illustra con chiarezza il rapporto di influenza che il fattore sociale esercita sull'individuo. Ma Durkheim tiene a rimarcare le possibilità dell'individuo rispetto all'istituzione sociale: se non vi fosse tale possibilità, del resto, non si spiegherebbe l'originaria formazione di generi e specie.
\begin{quote}
    Ma una cosa è il sentimento delle somiglianze, un’altra è la nozione di genere. Il genere è lo schema esteriore di cui gli oggetti percepiti come simili formano, in parte, il contenuto. E il contenuto non può fornire esso stesso lo schema sotto il quale si dispone. Esso è fatto di immagini vaghe e ondeggianti, dovute alla sovrapposizione e alla fusione parziale di un determinato numero di immagini individuali, che si trovano dotate di elementi comuni; lo schema, al contrario, è una forma definita, dai contorni stabili,ma suscettibile di applicazione a un numero determinato di cose, percepite o meno, attuali o possibili. [...] Ecco perché tutta una scuola di studiosi si rifiuta, non senza ragione, di identificare l’idea di genere e quella di immagine generica. L’immagine generica non è che la rappresentazione residua, dai limiti incerti, che lasciano in noi rappresentazioni simili, quando sono simultaneamente presenti nella coscienza; il genere è invece un simbolo logico in virtù del quale pensiamo distintamente queste affinità e altre analoghe.\footnote{\cite{durkheim2013forme}, pp.203-204.}
\end{quote}
Il discorso di Durkheim, in continuità con quanto già sostenuto insieme al nipote nell'articolo precedente, intende rintracciare le funzioni elementari del pensiero, partendo dalla gnoseologia religiosa o propriamente dall'analisi comparata delle forme di classificazione, per rinvenire, al di sotto di queste manifestazioni, gli elementi intellettuali che andranno poi a costituire il pensiero scientifico come oggi lo conosciamo. L'importanza dell'argomento è presto detta: questo tema, come si avrà modo di vedere in seguito, si rivelerà centrale all'interno dell'opera di Lévi-Strauss.
Tentando di fare un breve bilancio, appare chiaro che l'articolo di Durkheim e Mauss contiene diversi snodi concettuali difficoltosi, che la rendono una lettura tanto foriera di idee quanto delicata: si inserisce, come già evidenziato, in un dibattito che coinvolge tutte le discipline umanistiche in Francia ad inizio XX secolo, ognuna determinata a rivendicare un proprio statuto scientifico così come il proprio campo d'indagine; la discussione sulla natura della mentalità dei primitivi, così differente dalla mentalità dei moderni, tuttavia simile nelle sue manifestazioni più semplici; l'analisi delle forme di classificazione nel loro rapporto di reciproca influenza con le istituzioni sociali.
Senza avere la pretesa di risolvere tali questioni, si inizia ora a comprendere quanto il pensiero di Lévi-Strauss sia radicato nella storia delle scienze umane in Francia.
%-------------------------------------------------------------
\subsection{\normalfont{Introduzione} all'opera di Marcel Mauss}
È il 1950 quando una raccolta antologica di saggi di Marcel Mauss viene pubblicata dalla Presse Universitarie de France di Parigi. L'opera, composta da saggi pubblicati autonomamente sulle pagine di riviste quali \textit{L'Année Sociologique}, può considerarsi a buon diritto come una \textit{summa} del pensiero di Mauss. Il volume, originariamente intitolato \textit{Sociologie et anthropologie}, è edito in Italia da Einaudi ben quindici anni dopo, nel 1965, con il titolo \textit{Teoria della magia e altri saggi}. 
L'edizione italiana, con una prefazione di Ernesto De Martino, vede mantenuta e tradotta integralmente l'introduzione di Claude Levi-Strauss\footnote{\cite{mauss1965teoria} [pp. \textsc{xv-liv}]}, introduzione accolta con entusiasmo a livello internazionale, tanto da portare la casa editrice londinese Routledge a pubblicarla come volume autonomo in seguito\footnote{\cite{levi1987introduction}}. Percorrendo le parole di Levi-Strauss, tuttavia, presto ci si accorge che quella che si ha tra le mani è ben più di un'introduzione all'opera di Mauss: il lavoro di quest'ultimo, infatti, diventa quasi un pretesto, la materia prima di una rielaborazione estremamente originale. Si tratta di qualcosa di più che un'esegesi: è un omaggio, un contributo originale, una lettura estremamente personale che molto si allontana dallo scopo di introdurre, come avverte Georges Gurvitch \footnote{Si tenga presente che il sociologo russo, naturalizzato francese nel 1928, aveva inizialmente accolto Lévi-Strauss sotto la sua ala protettrice, per poi progressivamente prendere le distanze da questi una volta che l'impostazione strutturalista di quest'ultimo si era manifestata completamente.}, definendola una \enquote{interpretazione molto personale} nella sua prefazione\footnote{\cite{mauss1965teoria}, p.\textsc{xiv}.}.
Non importa in questa sede l'aspetto \textit{esteriore} del rapporto tra Marcel Mauss e Claude Lévi-Strauss\footnote{Rapporto testimoniato dalle parole dell'antropologo in \cite{levi1960tristi}, p.233.}, quanto invece la rappresentazione che lo stesso Lévi-Strauss fornisce del suo rapporto con Mauss. Inoltre lo scopo del presente lavoro non è ricostruire l'integrità concettuale dei testi di Marcel Mauss per poi comprendere le possibili distorsioni o forzature da parte di Claude Lévi-Strauss, ma la ricchezza del pensiero di quest'ultimo, nonché gli snodi concettuali propedeutici alla formazione del suo peculiare concetto di ragione, pertanto è molto più funzionale a questo scopo concentrarsi sulla rappresentazione del pensiero di Mauss e del proprio fornita da Lévi-Strauss stesso.
In tal senso Lévi-Strauss tiene più volte sottolinea il rapporto di continuità con il lavoro Marcel Mauss, di cui il volume rappresenta un campione rappresentativo: l'opera del defunto etnologo viene celebrata per la sua genialità, per la portata delle sue intuizioni, ma anche, e non a caso, per il suo carattere frammentario. Il corpus maussiano è ammirato per le sue \enquote{immense possibilità}, è definito addirittura \enquote{il \textit{novum organum} delle scienze sociali del xx secolo}; ebbene, non è improprio concludere che ponendosi in forte continuità con Mauss, e quindi celebrando la novità delle sue teorie, Lévi-Strauss intende marcare il carattere rivoluzionario del \textit{suo} modo di fare antropologia.
Il testo di Mauss che per ammissione dello stesso Lévi-Strauss ha influenzato più profondamente il suo pensiero è, com'è prevedibile vista la ricchezza del suo contenuto, l'\textit{Essai sur le don}\footnote{\cite{mauss1923essai}; tradotto in italiano come \cite{mauss2002saggio}}. In questo scritto Lévi-Strauss individua l'applicazione più efficace e feconda, per quanto non esaustiva, di un'osservazione attenta e penetrante, volta a rintracciare \textit{non} la \enquote{funzione} sociale del dono e dei rituali ad esso collegati, \textit{ma} la \enquote{struttura}, invisibile e onnipresente nell'organizzazione sociale, della quale il dono è una manifestazione.
Occorre rimarcare che la nozione di struttura non viene sviluppata nella sua interezza da Mauss: questo è invece merito di Lévi-Strauss.
\begin{quote}
    Mauss vi appare [nell'argomentazione seguita nell\textit{Essai sur le don}] dominato con ragione da una certezza di ordine logico, e cioè che lo \textit{scambio} è il comune denominatore di un grande numero di attività sociali apparentemente eterogenee. Ma egli non giunge a vedere questo scambio nei fatti. L'osservazione empirica non gli fornisce lo scambio, ma soltanto - come dice lui stesso - \enquote{tre obblighi: dare, ricevere, ricambiare}. Tutta la teoria esige così l'esistenza di una struttura, di cui l'esperienza non offre che frammenti, i membri sparsi, o piuttosto gli elementi.\footnote{\cite{mauss1965teoria}, p.\textsc{xli}}
\end{quote}


Nell'introduzione, inoltre, viene approfondito il rapporto tra etnologia e psicanalisi, rapporto 

È inutile soffermarsi sulla lettura del \textit{Saggio sul dono} di Mauss stesso, dal momento che le parole di Lévi-Strauss sono assai più esplicite e rivelatrici.

Intento di LS è porsi in continuità con Mauss, riprendendo e sviluppando le speculazioni di quest'ultimo da dove egli le ha interrotte.

\textit{novum organum} delle scienze sociali del XX secolo: LS sta in realtà definendo così la propria opera, consapevole della portata rivoluzionaria delle sue intuizioni.

Dicotomia tra Mauss e Malinowski in materia del concetto di \textit{funzione} e del fatto che quanto il primo era bravo a speculare il secondo lo era a osservare.

\section{Claude Lévi-Strauss lettore di Émile Durkheim}
La conoscenza enciclopedica di Lévi-Strauss non si limita alle culture indigene di popolazioni \textit{selvagge}, agli usi e costumi di popoli di paesi esotici, ma anche la sociologia francese \textit{tout court}. Marcel Mauss, come si è visto, è sicuramente oggetto di una particolare attenzione e predilezione, ma l'interesse di Lévi-Strauss, nonché il suo debito intellettuale, è rivolto a tutta la tradizione sociologica francese.
%------------------------------------------------------------
\subsection{\normalfont{La sociologie française}}
Nel 1945 Georges Gurvitch richiede a Claude Lévi-Strauss un articolo per il volume, curato dal sociologo russo, \textit{Les études sociologiques dans les différents pays} (1947)\footnote{Originariamente edito oltreoceano come \cite{gurvitch1945twentieth}}; 
%------------------------------------------------------------
\subsection{\normalfont{Le totémisme aujourd'houi}}
Il volume pubblicato nel 1962, \textit{Il totemismo oggi}\footnote{\cite{levi2020totemismo}.}, costituisce la prima parte di un progetto più ambizioso, che vedrà il suo compimento nel volume \textit{Il pensiero selvaggio}\footnote{\cite{levi2010pensiero}}, il quale sarà oggetto di osservazione approfondita nel capitolo successivo.
\textit{La pensée sauvage}, titolo che in lingua francese rimanda esplicitamente al fiore \textit{Viola tricolor} (Viola del pensiero in lingua italiana), va a comporre con \textit{Le totémisme aujourd'houi} un unico progetto: ritrovare all'interno delle istituzioni religiose e sociali le caratteristiche di un sistema di classificazione.
Il volume \textit{Il totemismo oggi}, più breve del suo \enquote{seguito}, è composto con uno stile argomentativo rigoroso, pieno di dimostrazioni e riferimenti accademici eruditi, e ha il compito di mostrare la consapevolezza storica con cui Lévi Strauss si inserisce nel dibattito sul totemismo. Il volumetto di Lévi-Strauss, come viene definito da Claudo Stroppa\footnote{\cite{stroppa1965totemismo}}, si può considerare come il suo personalissimo ed originale modo di compiere uno \textit{status quaestionis} in materia di totemismo. 
I principali interlocutori accademici con cui Lévi-Strauss deve confrontarsi all'interno dell'opera sono altri antropologi che all'epoca già avevano tentato di spiegare il fenomeno del totemismo senza successo: nonostante il problema fosse già stato sondato più volte all'epoca, nessuno era riuscito fino a quel momento a produrre un'interpretazione esaustiva, in grado di spiegare il fenomeno totemico in tutti i suoi aspetti. Del resto, è da sottolineare, il fenomeno del totemismo aveva attirato l'attenzione di intellettuali di diversa formazione, tra i quali spicca Henri Bergson, \textit{maitre à penser} della Francia tra i due secoli, filosofo estremamente prolifico e influente sulle generazioni successive.
Nonostante la polemica presa di distanza dalle discipline filosofiche e dal modo di insegnarle in Francia che Lévi-Strauss conduce nella prima parte del celeberrimo \textit{Tristi Tropici}\footnote{\cite{levi1960tristi}, pp.49-52.}, la formazione in materia che ricevette influenzò profondamente il modo di condurre la ricerca etnologica. Si tengano presenti due fatti: da una parte Lévi-Strauss aveva preparato il concorso per diventare professore di filosofia\footnote{Vd. in \cite{levi1960tristi}, p.52.}, dall'altro lato la sua formazione in filosofia, che lo vede lettore particolarmente attento di pensatori della tradizione francese quali Montaigne, Rousseau e Bergson lo situa in una posizione privilegiata per applicare e sfruttare al meglio procedimenti tradizionalmente utilizzati in filosofia all'interno dell'etnologia.
%------------------------------------------------------------
\subsection{\textit{Il totemismo dal di dentro}}
Il quinto capitolo, l'ultimo de \textit{Il totemismo oggi}, può essere letto come un confronto tra due metodi apparentemente contrari, entrambi tesi a spiegare il totemismo, seppur partendo da prospettive diametralmente opposte.
Il metodo etnologico, come se fosse possibile definirne propriamente uno, si può approssimativamente riassumere come la raccolta di un numero indefinitamente ampio di testimonianze circa usi e costumi di popoli (si tenga presente che prima di Lévi-Strauss erano assai rari gli antropologi e gli etnologi con l'uso di compiere le loro indagini sul campo, si pensi a Malinowski, essendo molto più diffusa l'usanza di affidarsi alla vasta produzione di memorie e resoconti da parte di etnografi) seguito dal tentativo di elaborare una spiegazione che sia applicabile a tutti gli elementi oggetto dell'indagine. Ma tale spiegazione si può anche riassumere come il tentativo di individuare un elemento comune, un'unità minima rintracciabile alla base di ogni rituale caratteristico di un gruppo sociale


[insert here] la lettura levistraussiana della sociologia francese

Una graffiante pagina dell'opera sopracitata accusa ferocemente la filosofia di essersi ridotta a un vuoto gioco di retorica, costituito da elementi evocati con la parola e dissolti da questa stessa\footnote{Ho cominciato allora a capire che tutti i problemi, gravi o futili, possono essere liquidati applicando un metodo sempre identico, che consiste nel contrapporre due punti di vista tradizionali sulla questione,; introdurre cioè il primo con le giustificazioni del senso comune, per distruggerlo poi con il secondo; infine rigettarli uno da una parte e uno dall'altra, adottando invece un terzo punto di vista che riveli il carattere ugualmente parziale dei due altri, ricondotti con artifici di vocabolario agli aspetti complementari di una stessa realtà: forma e sostanza, contenente e contenuto, essere e parere, continuo e discontinuo, essenza ed esistenza ecc. [\cite{levi1960tristi}, p.49]}, ma nonostante queste aperte dichiarazioni di ostilità nei confronti

\chapter{Ragione analitica e ragion dialettica}
\section{Jean-Paul Sartre e l'antropologia filosofica}
\subsection{Alcune questioni editoriali}Prima di avvicinarsi alla \textit{Critica della ragion dialettica} occorre chiarire una questione editoriale di non minore importanza: l'opera, estremamente corposa, si articola in due tomi sia nell'edizione francese che in quella italiana. Il primo tomo, \textit{Teoria degli insiemi pratici}, viene edito in Francia da Gallimard nel 1960\footnote{\cite{sartre1960critique}} e in Italia da Il Saggiatore\footnote{\cite{sartre1963critica}} in due volumi separati nel 1963, in un'edizione che vede anteposte alla \textit{Critica} vera e propria le \textit{Questioni di metodo}. Il secondo tomo, \textit{L'intelligibilità della storia}, viene pubblicato nel 1985 a Parigi sempre da Gallimard\footnote{\cite{sartre1985critique}}, mentre in Italia, per la traduzione pubblicata da Marinotti occorre attendere fino al 2006.
Il primo volume, edito mentre Sartre è ancora vivente, è un lavoro compiuto nonostante la sua complessità e le difficoltà interpretative che ancora oggi suscita; il secondo volume, invece, è postumo, e come tale non ha potuto ricevere la revisione che gli sarebbe spettata.
\subsection{Un'accoglienza problematica}
La \textit{Critique} è accolta con recensioni assai sfavorevoli in terra francese, critiche e aperte prese di posizione polemiche nei confronti del linguaggio e dei contenuti, giudicati oscuri, ridondanti, sospettati di misticismo. Una simile ostilità può essere spiegata a partire dal fatto che il libro di Sartre aveva come bersaglio polemico sia gli intellettuali marxisti, colpevoli di dogmatismo (cfr. \textit{infra}) e mancanza di \textit{engagement}, sia gli intellettuali non marxisti, al servizio di un sistema capitalistico colpevole di trasformare gli uomini in \textit{cose} attraverso un processo di alienazione.
Lo scritto viene accolto con freddezza sia a destra, si pensi a Raymond Aron\footnote{Per un'analisi più attenta della risposta di Aron cfr. \textit{infra}.}, che a sinistra, nell'ambiente marxista.
Dopo un'iniziale diffidenza, però, e soprattutto dopo la morte dell'autore e l'edizione della seconda parte dell'opera, la \textit{Critique} ha ricevuto maggior interesse e credito: oltreoceano Joseph Catalano ha dedicato un intero volume al commento dell'opera\footnote{\cite{catalano1986commentary}}, nonché altri saggi sul pensiero del \textit{philosophe}\footnote{\cite{catalano1985commentary}; \cite{catalano2010body}; \cite{catalano2010reading}; \cite{catalano2021saint}.} parigino; e anche in Europa si sono moltiplicati i contributi dedicati al marxismo sartriano, come si vedrà in seguito\footnote{Per una lista aggiornata si veda la rassegna bibliografica curata da Gabriella Farina in \cite{wormser2005sartre}, pp.XXXXX}.
%---------------------------------------------------------
\subsection{Rinnovare il marxismo.} La \textit{Critica} e le \textit{Questioni di metodo}, come sottolinea nella sua \textit{Introduzione a Sartre} Sergio Moravia\footnote{Nonostante il lavoro di Moravia sia datato, il suo contributo è prezioso poiché, come si può notare dalla bibliografia, questo studioso si è occupato approfonditamente sia di Sartre che di Lévi-Strauss. Per una bibliografia esaustiva del recentemente scomparso storico della filosofia non è ancora disponibile online una lista completa.}, si pone l'obiettivo di \enquote{ridestare} il marxismo contemporaneo dal sonno dogmatico in cui era sopito. L'originale obiettivo di Marx e del marxismo era recuperare una connessione diretta con il concreto, connessione che l'idealismo tedesco aveva perso riducendo il reale alla sua componente idealistica, e portando avanti il ragionamento sulla realtà ipostatizzata come idea, e non sulla realtà conosciuta attraverso un processo dialettico.
L'obiettivo di Sartre è rivitalizzare il marxismo attraverso una attenta disamina critica, svolta applicando il metodo critico agli stessi strumenti critici di cui fa uso il marxismo.
Al centro di tale disamina, com'è facile intuire, sta la nozione di dialettica, nozione controversa, guardata con sospetto dai filosofi di corrente analitica, e vissuta come problematica anche dalla tradizione continentale. 
Per affrontare al meglio lo sviluppo dei temi che si vengono a presentare è bene dedicare qualche parola a descrivere il concetto di \enquote{ragione dialettica}, dal momento che, come si vedrà, Lévi-Strauss utilizza e critica questa nozione.

\paragraph{La dialettica}
Hervé Vautrelle, all'interno del commentario edito da Ellipses\footnote{\cite{vautrelle2001critique}}, chiarisce perché sia necessario distinguere tra ragione analitica e ragion dialettica.
\begin{quote}
    La ragion dialettica rischiara il passato e il presente alla luce dell'avvenire, attraverso \enquote{l'intelligibilità assoluta di una novità irriducibile}, mentre la ragione analitica rapporta il presente e il futuro al passato, dissolvendo ciò che è sconosciuto in ciò che è conosciuto. La ragione. La ragione analitica scopre i legami di esteriorità riguardo ad elementi che sono giustapposti gli uni di fianco agli altri, senza compenetrazione [reciproca]. Per la ragione dialettica, tutti i momenti di un processo prendono senso in rapporto al primo momento, e quest'ultimo si comprende a partire dai momenti che seguono.
\end{quote}

Un ricco volume antologico curato da Alberto Burgio ne ricostruisce alcune declinazioni in autori di capitale importanza\footnote{\cite{burgio2007dialettica}}; tra questi spicca il contributo di Nicolas Tertulian, dedicato alla nozione di dialettica in Sartre.
É da notare che 


Ma cosa c'entra con LS? Per Sartre il marxismo è l'unica antropologia in quanto scienza dell'uomo possibile. L'unica che riconosca all'uomo il suo statuto di individuo libero e che cerchi di restituirgli le sue possibilità di emancipazione dall'alienazione
%-----------------------------------------------------------
\section{Claude Lévi-Strauss: struttura e storia}
Come ricorda l'articolo di Pierre Guenancia\footnote{\cite{guenancia2013fourmis}}, nell'opera di Lévi-Strauss un particolare 
Oggetto di questo paragrafo sarà l'opera \textit{Il pensiero selvaggio}\footnote{\cite{levi2010pensiero}.}, e in particolare l'ultimo capitolo 
Vediamo di fare notevole riferimento all'opera di F. Remotti\footnote{\cite{remotti1971levi}.}
\subsection{Antropologia e marxismo}
Vd. il libro di Wiktor stockowski

\chapter{Alcuni eredi}
\section{Raymond Aron}


\section{Jean Pouillon}
nel suo Fetiches sans fetichisme c'è un capitolo dedicato al dibattito


\section{Lucien Sebag}

%\bibliographystyle{siam} %questo set di comandi esclude e rimpiazza DEL TUTTO i comandi definiti come da preambolo. In caso di modifica per tornare ai comandi standard, verificare di sostituire correttamente TUTTI i componenti, sia nel preambolo che nella coda del documento.
%\bibliography{bibliography}
%il comando \nocite necessita di specificare quale voce non citata riportare
\printbibliography %inserisce le voci citate
\end{document}