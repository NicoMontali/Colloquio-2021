\section{Claude Lévi-Strauss lettore di Marcel Mauss}
Potrebbe sembrare un torto verso la storia delle scienze umane anteporre Marcel Mauss allo zio Émile Durkheim, al quale il nipote è debitore sotto diversi aspetti, ma si vedrà presto il motivo: nella produzione levistraussiana l'interpretazione dell'opera di Mauss si può a buon diritto considerare come un eccezionale, nonché originale, accesso al pensiero di Lévi-Strauss; il pensiero dell'etnologo, esposto dalle parole dell'antropologo funge, per fare uso di una metafora geologica, da marcatore, utile ad individuare gli aspetti originali del pensiero di Claude Lévi-Strauss per come vengono evidenziati da lui stesso nel lavoro di altro. In altri termini, è proprio il commento di Lévi-Strauss che estrapola dall'opera di Mauss i tratti in cui i due intellettuali convergono.
Secondariamente, la pubblicazione dell'\textit{Introduzione} alle opere di Marcel Mauss da parte di Lévi-Strauss stesso avviene nel 1950, mentre le tesi Émile Durkheim esposte ne \textit{Il totemismo oggi}, facenti diretto riferimento a Durkheim (tra gli altri), pervengono all'attenzione del grande pubblico solo nel 1962. Si proceda quindi con questa scansione.
\\È il 1950 quando una raccolta antologica di saggi di Marcel Mauss viene pubblicata dalla Presse Universitarie de France di Parigi. L'opera, composta da saggi pubblicati autonomamente sulle pagine di riviste avvalendosi della collaborazione dell'amico e collega Henri Hubert, può considerarsi a buon diritto come una \textit{summa} del pensiero di Mauss. Il volume, originariamente intitolato \textit{Sociologie et anthropologie}, è edito in Italia da Einaudi ben quindici anni dopo, nel 1965, con il titolo \textit{Teoria della magia e altri saggi}. 
L'edizione italiana, con una prefazione di Ernesto De Martino, vede mantenuta e tradotta integralmente l'introduzione di Claude Levi-Strauss\footnote{\cite{mauss1965teoria} [pp. \textsc{xv-liv}]}, introduzione che peraltro ha ricevuto approvazione e notorietà in terra americana, portando, nel 1987, la casa editrice Taylor \& Francis a pubblicarla come volume autonomo\footnote{\cite{levi1987introduction}}. Percorrendo le parole di Levi-Strauss, tuttavia, presto ci si accorge che quella che si ha tra le mani è ben più di un'introduzione all'opera di Mauss: il lavoro di quest'ultimo, infatti, diventa quasi un pretesto, la materia prima di una rielaborazione estremamente originale.
\subsection{Le forme di classificazione: tra pensiero ed istituzioni sociali.}
Una tappa preliminare ma necessaria all'elaborazione delle teorie esposte da Mauss nel giustamente celebre \textit{Essai sur le don}, consiste nel saggio scritto a quattro mani con Émile Durkheim \textit{De quelques formes primitives de classification}. Il saggio, come dice il titolo, analizza alcune forme di classificazione presenti nelle civiltà arcaiche: le fratrie, i totem e i clan. Ma perché le classificazione, un argomento così spiccatamente teoretico, suscita l'interesse 
ma il motivo per cui l'argomento suscita interesse negli antropologi è lo stretto rapporto che i generi e le specie intrattengono con le istituzioni sociali. L'intuizione lungimirante presente nel saggio, che Mauss svilupperà nelle sue opere più mature, è che vi siano dei fatti sociali in grado di permeare ed influenzare tutti i livelli della vita di un individuo. L'analisi delle forme di classificazione, tipicamente un argomento di teoria della conoscenza, è un punto d'accesso privilegiato per lo studio delle istituzioni sociali. La tesi sostenuta da Mauss e Durkheim, infatti, è che vi sia un parallelismo tra la ripartizione delle persone in fratrie e clan o totem, e la divisione in una gerarchia di generi e specie.


RICORDATI DI AGGIUNGERE LE COSE DA \cite{karsenti1997uomo}
\section{Claude Lévi-Strauss lettore di Émile Durkheim}
Il volume pubblicato nel 1962, \textit{Il totemismo oggi}\footnote{\cite{levi2020totemismo}.}, costituisce la prima parte di un progetto più ambizioso, che vedrà il suo compimento nel volume \textit{Il pensiero selvaggio}\footnote{\cite{levi2010pensiero}}, il quale sarà oggetto di osservazione approfondita nel capitolo successivo.
\textit{La pensée sauvage}, titolo che in lingua francese rimanda esplicitamente al fiore \textit{Viola tricolor} (Viola del pensiero in lingua italiana), va a comporre con \textit{Le totémisme aujourd'houi} un unico progetto: ritrovare all'interno delle istituzioni religiose e sociali le caratteristiche di un sistema di classificazione.
Il volume \textit{Il totemismo oggi}, più breve del suo "seguito", è composto con uno stile argomentativo rigoroso, pieno di dimostrazioni e riferimenti accademici eruditi, e ha il compito di mostrare la consapevolezza storica con cui Lévi Strauss si inserisce nel dibattito sul totemismo. Il volumetto di Lévi-Strauss, come viene definito da Claudo Stroppa\footnote{\cite{stroppa1965totemismo}}, si può considerare come il suo personalissimo ed originale modo di compiere uno \textit{status quaestionis} in materia di totemismo. 
I principali interlocutori accademici con cui Lévi-Strauss deve confrontarsi all'interno dell'opera sono altri antropologi che all'epoca già avevano tentato di spiegare il fenomeno del totemismo senza successo: nonostante il problema fosse già stato sondato più volte all'epoca, nessuno era riuscito fino a quel momento a produrre un'interpretazione esaustiva, in grado di spiegare il fenomeno totemico in tutti i suoi aspetti. Del resto, è da sottolineare, il fenomeno del totemismo aveva attirato l'attenzione di intellettuali di diversa formazione, tra i quali spicca Henri Bergson, \textit{maitre à penser} della Francia tra i due secoli, filosofo estremamente prolifico e influente sulle generazioni successive.
Nonostante la polemica presa di distanza dalle discipline filosofiche e dal modo di insegnarle in Francia che Lévi-Strauss conduce nella prima parte del celeberrimo \textit{Tristi Tropici}\footnote{\cite{levi1960tristi}, pp.49-52.}, la formazione in materia che ricevette influenzò profondamente il modo di condurre la ricerca etnologica. Si tengano presenti due fatti: da una parte Lévi-Strauss aveva preparato il concorso per diventare professore di filosofia\footnote{Vd. in \cite{levi1960tristi}, p.52.}, dall'altro lato la sua formazione in filosofia, che lo vede lettore particolarmente attento di pensatori della tradizione francese quali Montaigne, Rousseau e Bergson lo situa in una posizione privilegiata per applicare e sfruttare al meglio procedimenti tradizionalmente utilizzati in filosofia all'interno dell'etnologia.
\subsection{\textit{Il totemismo dal di dentro}}
Il quinto capitolo, l'ultimo de \textit{Il totemismo oggi}, può essere letto come un confronto tra due metodi apparentemente contrari, entrambi tesi a spiegare il totemismo, seppur partendo da prospettive diametralmente opposte.
Il metodo etnologico, come se fosse possibile definirne propriamente uno, si può approssimativamente riassumere come la raccolta di un numero indefinitamente ampio di testimonianze circa usi e costumi di popoli (si tenga presente che prima di Lévi-Strauss erano assai rari gli antropologi e gli etnologi con l'uso di compiere le loro indagini sul campo, si pensi a Malinowski, essendo molto più diffusa l'usanza di affidarsi alla vasta produzione di memorie e resoconti da parte di etnografi) seguito dal tentativo di elaborare una spiegazione che sia applicabile a tutti gli elementi oggetto dell'indagine. Ma tale spiegazione si può anche riassumere come il tentativo di individuare un elemento comune, un'unità minima rintracciabile alla base di ogni rituale caratteristico di un gruppo sociale


Una graffiante pagina dell'opera sopracitata accusa ferocemente la filosofia di essersi ridotta a un vuoto gioco di retorica, costituito da elementi evocati con la parola e dissolti da questa stessa\footnote{Ho cominciato allora a capire che tutti i problemi, gravi o futili, possono essere liquidati applicando un metodo sempre identico, che consiste nel contrapporre due punti di vista tradizionali sulla questione,; introdurre cioè il primo con le giustificazioni del senso comune, per distruggerlo poi con il secondo; infine rigettarli uno da una parte e uno dall'altra, adottando invece un terzo punto di vista che riveli il carattere ugualmente parziale dei due altri, ricondotti con artifici di vocabolario agli aspetti complementari di una stessa realtà: forma e sostanza, contenente e contenuto, essere e parere, continuo e discontinuo, essenza ed esistenza ecc. [\cite{levi1960tristi}, p.49]}