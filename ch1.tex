\section{Claude Lévi-Strauss lettore di Marcel Mauss}
Potrebbe sembrare un torto verso la storia delle scienze umane anteporre Marcel Mauss allo zio Émile Durkheim, al quale il nipote è debitore sotto diversi aspetti, ma si vedrà presto il motivo: nella produzione levistraussiana l'interpretazione dell'opera di Mauss si può a buon diritto considerare come un eccezionale punto d'accesso al pensiero di Lévi-Strauss; il pensiero dell'etnologo, esposto dalle parole dell'antropologo funge, per fare uso di una metafora geologica, da marcatore, utile ad individuare gli aspetti originali del pensiero di Claude Lévi-Strauss per come vengono evidenziati da lui stesso nel lavoro di altro. In altri termini, è proprio il commento di Lévi-Strauss che estrapola dall'opera di Mauss i punti sui quali i due intellettuali convergono.
Secondariamente, l'\textit{Introduzione} all'opera di Marcel Mauss da parte di Lévi-Strauss è pubblicata nel 1950, mentre il volume \textit{Il totemismo oggi}, in cui Durkheim figura (tra gli altri) come interlocutore virtuale vede la luce solo nel 1962. Si procederà quindi con questa scansione.
%----------------------------------------------------------------
\subsection{Le forme di classificazione: tra pensiero ed istituzioni sociali.}
Una tappa preliminare ma necessaria all'elaborazione delle teorie esposte da Mauss nel giustamente celebre \textit{Essai sur le don}, consiste nel saggio scritto a quattro mani con Émile Durkheim \textit{De quelques formes primitives de classification}\footnote{\cite{durkheimmaussformes}}. Il saggio, come dice il titolo, analizza alcune forme di classificazione presenti nelle civiltà arcaiche: le fratrie, i totem e i clan. Ma perché le classificazione, un argomento così spiccatamente teoretico, suscita l'interesse dei due antropologi? La tesi alla base del saggio, nonché l'intuizione più feconda dell'antropologia maussiana è il carattere totale di fatti sociali quali le forme di classificazione. Durkheim e Mauss sostengono che vi sia uno stretto rapporto di parentela, un vero e proprio parallelismo, tra le forme di classificazione e le istituzioni sociali.
Ma la loro analisi parte da presupposti da specificare: per gli autori la classificazione si forma non spontaneamente da uno stato del pensiero di compenetrazione e indistinguibilità tra gli elementi; il pensiero dei primitivi, non diversamente che per Lévy-Bruhl, è caratterizzato dal \enquote{principio di partecipazione}, uno stato di non distinzione tra un elemento e un altro, in cui tutti gli oggetti si assomigliano e fanno parte di un unico tutto\footnote{\enquote{In quel punto la coscienza non è che un flusso continuo di rappresentazioni che si perdono le une nelle altre, e quando le differenziazioni cominciano ad apparire, sono tutt'affatto frammentarie}. \cite{durkheimmaussformes}, p.23.}. In questa prospettiva la storia della logica, almeno per quanto riguarda i concetti di generi e specie, è da rifondare: tali concetti sono soggetti ad uno sviluppo storico, non si trovano in natura e non sono dati nella loro interezza; sono invece prodotti di uno sviluppo che coinvolge più elementi eterogenei, innestandosi in un processo che se fosse lasciato alla sua spontaneità non porterebbe alla nascita di concetti quali generi e specie. Ma si veda ciò attraverso le parole di Durkheim e Mauss:
\begin{quote}
    Non soltanto la nostra nozione attuale della classificazione ha una storia, ma questa stessa storia presuppone una considerevole preistoria. In realtà, lo spirito umano ha preso le mosse da uno stato di massima indistinzione; ancor oggi c'è tutta una parte della nostra letteratura popolare, dei nostri miti, delle religioni che si basa su una fondamentale confusione di ogni immagine e idea. Si potrebbe affermare che immagini o idee separate le une dalle alte con una certa chiarezza, non ce ne siano.\footnote{\cite{durkheimmaussformes}, p.21.}[...]
    \\Non è vero, dunque, che l'uomo classifichi spontaneamente e per una sorta di necessità naturale: agli inizi, fanno difetto all'umanità anche le condizioni indispensabili alla funziona classificatrice. E poi basta analizzare l'idea di classificazione per comprendere che l'uomo non poteva trovare in se stesso gli elementi essenziali. Una classe è un gruppo di cose; orbene, le cose non si offrono all'osservazione di per se stesse raggruppate.\footnote{\cite{durkheimmaussformes}, p.23.}
\end{quote}
Dare il giusto peso a queste parole significa riconoscere che la classificazione non è un fatto innato, appartenente all'ordine naturale delle cose, ma un dato culturale, che si va a manifestare unicamente ove vi sia la civiltà e le sue istituzioni. Ciò che prima poteva sembrare una facoltà che spontaneamente nata dallo sviluppo autonomo della ragione, si scopre ora essere il risultato di un processo storico, appreso ed insegnato, ed in quanto storico frutto di uno sviluppo diacronico.
Ora, se si prosegue nella lettura sarà facile capire le intenzioni degli antropologi: \enquote{L'importanza di questa classificazione è tale che si estende a tutti i fatti della vita e se ne ritrova la traccia in tutti i riti principali}\footnote{\cite{durkheimmaussformes}, p.28.}. Quest'affermazione significa che tra i fatti sociali e l'individuo vi è una reciproca influenza e determinazione. Per riutilizzare le parole di Bruno Karsenti, \enquote{si tratta di cogliere l' "uomo tutto intero", o l' "uomo totale", e al tempo stesso di inscrivere la sociologia "nell'antropologia". Su un piano strettamente sociologico, si tratta sempre di cogliere l'essenza dei rapporti sociali, di raggiungere il punto in cui questi si intrecciano concretamente e si esprimono in una totalità; o ancora, secondo un concetto cui Mauss deve gran parte della sua celebrità, si tratta di cogliere il "fatto sociale totale".}\footnote{\cite{karsenti1997uomo}, p.33.}
Il motivo di questo rapido \textit{excursus} è presto spiegato: una volta evidenziata la natura \enquote{totale} di alcuni fatti, tra i quali le forme di classificazione, è possibile osservare i fatti sociali da un'altra prospettiva, suggestiva e illuminante. Del resto, è noto che Durkheim nell'ultima parte della sua produzione si sia dedicato alla sociologia religiosa, ma è bene sottolineare che il fine di questa svolta fosse lo sviluppo di una teoria della conoscenza dal punto di vista sociologico.
Si prenda un estratto da \textit{Le forme elementari della religione} di Émile Durkheim (1912), dichiaratamente debitore all'articolo scritto a quattro mani con il nipote Marcel Mauss:
\begin{quote}
   [...] queste classificazioni sistematiche sono le prime che incontriamo nella storia; ora, si è visto che esse si sono foggiate sull’organizzazione sociale, o piuttosto che hanno preso come schemi i quadri stessi della società. Le fratrie hanno servito da generi, e i clan da specie. Poiché erano raggruppati gli uomini, essi hanno potuto raggruppare le cose; per classificare queste ultime si sono limitati a fare loro posto nei gruppi che formavano essi stessi. E se queste diverse classi di cose non sono state semplicemente giustapposte le une alle altre, ma sono state ordinate secondo un piano unitario, ciò è accaduto perché i gruppi sociali con cui esse si confondono sono anch’essi solidali e formano con la loro unione un tutto organico, la tribù. L’unità di questi primi sistemi logici non fa che riprodurre l’unità della società.\footnote{\cite{durkheim2013forme}, p.201.}
\end{quote}
Il passo illustra con chiarezza il rapporto di influenza che il fattore sociale esercita sull'individuo. Ma Durkheim tiene a rimarcare le possibilità dell'individuo rispetto all'istituzione sociale: se non vi fosse tale possibilità, del resto, non si spiegherebbe l'originaria formazione di generi e specie.
\begin{quote}
    Ma una cosa è il sentimento delle somiglianze, un’altra è la nozione di genere. Il genere è lo schema esteriore di cui gli oggetti percepiti come simili formano, in parte, il contenuto. E il contenuto non può fornire esso stesso lo schema sotto il quale si dispone. Esso è fatto di immagini vaghe e ondeggianti, dovute alla sovrapposizione e alla fusione parziale di un determinato numero di immagini individuali, che si trovano dotate di elementi comuni; lo schema, al contrario, è una forma definita, dai contorni stabili,ma suscettibile di applicazione a un numero determinato di cose, percepite o meno, attuali o possibili. [...] Ecco perché tutta una scuola di studiosi si rifiuta, non senza ragione, di identificare l’idea di genere e quella di immagine generica. L’immagine generica non è che la rappresentazione residua, dai limiti incerti, che lasciano in noi rappresentazioni simili, quando sono simultaneamente presenti nella coscienza; il genere è invece un simbolo logico in virtù del quale pensiamo distintamente queste affinità e altre analoghe.\footnote{\cite{durkheim2013forme}, pp.203-204.}
\end{quote}
Il discorso di Durkheim, in continuità con quanto già sostenuto insieme al nipote nell'articolo precedente, intende rintracciare le funzioni elementari del pensiero, partendo dalla gnoseologia religiosa o propriamente dall'analisi comparata delle forme di classificazione, per rinvenire, al di sotto di queste manifestazioni, gli elementi intellettuali che andranno poi a costituire il pensiero scientifico come oggi lo conosciamo. L'importanza dell'argomento è presto detta: questo tema, come si avrà modo di vedere in seguito, si rivelerà centrale all'interno dell'opera di Lévi-Strauss.
Tentando di fare un breve bilancio, appare chiaro che l'articolo di Durkheim e Mauss contiene diversi snodi concettuali difficoltosi, che la rendono una lettura tanto foriera di idee quanto delicata: si inserisce, come già evidenziato, in un dibattito che coinvolge tutte le discipline umanistiche in Francia ad inizio XX secolo, ognuna determinata a rivendicare un proprio statuto scientifico così come il proprio campo d'indagine; la discussione sulla natura della mentalità dei primitivi, così differente dalla mentalità dei moderni, tuttavia simile nelle sue manifestazioni più semplici; l'analisi delle forme di classificazione nel loro rapporto di reciproca influenza con le istituzioni sociali.
Senza avere la pretesa di risolvere tali questioni, si inizia ora a comprendere quanto il pensiero di Lévi-Strauss sia radicato nella storia delle scienze umane in Francia.
%-------------------------------------------------------------
\subsection{\normalfont{Introduzione} all'opera di Marcel Mauss}
È il 1950 quando una raccolta antologica di saggi di Marcel Mauss viene pubblicata dalla Presse Universitarie de France di Parigi. L'opera, composta da saggi pubblicati autonomamente sulle pagine di riviste quali \textit{L'Année Sociologique}, può considerarsi a buon diritto come una \textit{summa} del pensiero di Mauss. Il volume, originariamente intitolato \textit{Sociologie et anthropologie}, è edito in Italia da Einaudi ben quindici anni dopo, nel 1965, con il titolo \textit{Teoria della magia e altri saggi}. 
L'edizione italiana, con una prefazione di Ernesto De Martino, vede mantenuta e tradotta integralmente l'introduzione di Claude Levi-Strauss\footnote{\cite{mauss1965teoria} [pp. \textsc{xv-liv}]}, introduzione accolta con entusiasmo a livello internazionale, tanto da portare la casa editrice londinese Routledge a pubblicarla come volume autonomo in seguito\footnote{\cite{levi1987introduction}}. Percorrendo le parole di Levi-Strauss, tuttavia, presto ci si accorge che quella che si ha tra le mani è ben più di un'introduzione all'opera di Mauss: il lavoro di quest'ultimo, infatti, diventa quasi un pretesto, la materia prima di una rielaborazione estremamente originale. Si tratta di qualcosa di più che un'esegesi: è un omaggio, un contributo originale, una lettura estremamente personale che molto si allontana dallo scopo di introdurre, come avverte Georges Gurvitch \footnote{Si tenga presente che il sociologo russo, naturalizzato francese nel 1928, aveva inizialmente accolto Lévi-Strauss sotto la sua ala protettrice, per poi progressivamente prendere le distanze da questi una volta che l'impostazione strutturalista di quest'ultimo si era manifestata completamente.}, definendola una \enquote{interpretazione molto personale} nella sua prefazione\footnote{\cite{mauss1965teoria}, p.\textsc{xiv}.}.
Non importa in questa sede l'aspetto \textit{esteriore} del rapporto tra Marcel Mauss e Claude Lévi-Strauss\footnote{Rapporto testimoniato dalle parole dell'antropologo in \cite{levi1960tristi}, p.233.}, quanto invece la rappresentazione che lo stesso Lévi-Strauss fornisce del suo rapporto con Mauss. Inoltre lo scopo del presente lavoro non è ricostruire l'integrità concettuale dei testi di Marcel Mauss per poi comprendere le possibili distorsioni o forzature da parte di Claude Lévi-Strauss, ma la ricchezza del pensiero di quest'ultimo, nonché gli snodi concettuali propedeutici alla formazione del suo peculiare concetto di ragione, pertanto è molto più funzionale a questo scopo concentrarsi sulla rappresentazione del pensiero di Mauss e del proprio fornita da Lévi-Strauss stesso.
In tal senso Lévi-Strauss tiene più volte sottolinea il rapporto di continuità con il lavoro Marcel Mauss, di cui il volume rappresenta un campione rappresentativo: l'opera del defunto etnologo viene celebrata per la sua genialità, per la portata delle sue intuizioni, ma anche, e non a caso, per il suo carattere frammentario. Il corpus maussiano è ammirato per le sue \enquote{immense possibilità}, è definito addirittura \enquote{il \textit{novum organum} delle scienze sociali del xx secolo}; ebbene, non è improprio concludere che ponendosi in forte continuità con Mauss, e quindi celebrando la novità delle sue teorie, Lévi-Strauss intende marcare il carattere rivoluzionario del \textit{suo} modo di fare antropologia.
Il testo di Mauss che per ammissione dello stesso Lévi-Strauss ha influenzato più profondamente il suo pensiero è, com'è prevedibile vista la ricchezza del suo contenuto, l'\textit{Essai sur le don}\footnote{\cite{mauss1923essai}; tradotto in italiano come \cite{mauss2002saggio}}. In questo scritto Lévi-Strauss individua l'applicazione più efficace e feconda, per quanto non esaustiva, di un'osservazione attenta e penetrante, volta a rintracciare \textit{non} la \enquote{funzione} sociale del dono e dei rituali ad esso collegati, \textit{ma} la \enquote{struttura}, invisibile e onnipresente nell'organizzazione sociale, della quale il dono è una manifestazione.
Occorre rimarcare che la nozione di struttura non viene sviluppata nella sua interezza da Mauss: questo è invece merito di Lévi-Strauss.
\begin{quote}
    Mauss vi appare [nell'argomentazione seguita nell\textit{Essai sur le don}] dominato con ragione da una certezza di ordine logico, e cioè che lo \textit{scambio} è il comune denominatore di un grande numero di attività sociali apparentemente eterogenee. Ma egli non giunge a vedere questo scambio nei fatti. L'osservazione empirica non gli fornisce lo scambio, ma soltanto - come dice lui stesso - \enquote{tre obblighi: dare, ricevere, ricambiare}. Tutta la teoria esige così l'esistenza di una struttura, di cui l'esperienza non offre che frammenti, i membri sparsi, o piuttosto gli elementi.\footnote{\cite{mauss1965teoria}, p.\textsc{xli}}
\end{quote}


Nell'introduzione, inoltre, viene approfondito il rapporto tra etnologia e psicanalisi, rapporto 

È inutile soffermarsi sulla lettura del \textit{Saggio sul dono} di Mauss stesso, dal momento che le parole di Lévi-Strauss sono assai più esplicite e rivelatrici.

Intento di LS è porsi in continuità con Mauss, riprendendo e sviluppando le speculazioni di quest'ultimo da dove egli le ha interrotte.

\textit{novum organum} delle scienze sociali del XX secolo: LS sta in realtà definendo così la propria opera, consapevole della portata rivoluzionaria delle sue intuizioni.

Dicotomia tra Mauss e Malinowski in materia del concetto di \textit{funzione} e del fatto che quanto il primo era bravo a speculare il secondo lo era a osservare.

\section{Claude Lévi-Strauss lettore di Émile Durkheim}
La conoscenza enciclopedica di Lévi-Strauss non si limita alle culture indigene di popolazioni \textit{selvagge}, agli usi e costumi di popoli di paesi esotici, ma anche la sociologia francese \textit{tout court}. Marcel Mauss, come si è visto, è sicuramente oggetto di una particolare attenzione e predilezione, ma l'interesse di Lévi-Strauss, nonché il suo debito intellettuale, è rivolto a tutta la tradizione sociologica francese.
%------------------------------------------------------------
\subsection{\normalfont{La sociologie française}}
Nel 1945 Georges Gurvitch richiede a Claude Lévi-Strauss un articolo per il volume, curato dal sociologo russo, \textit{Les études sociologiques dans les différents pays} (1947)\footnote{Originariamente edito oltreoceano come \cite{gurvitch1945twentieth}}; 
%------------------------------------------------------------
\subsection{\normalfont{Le totémisme aujourd'houi}}
Il volume pubblicato nel 1962, \textit{Il totemismo oggi}\footnote{\cite{levi2020totemismo}.}, costituisce la prima parte di un progetto più ambizioso, che vedrà il suo compimento nel volume \textit{Il pensiero selvaggio}\footnote{\cite{levi2010pensiero}}, il quale sarà oggetto di osservazione approfondita nel capitolo successivo.
\textit{La pensée sauvage}, titolo che in lingua francese rimanda esplicitamente al fiore \textit{Viola tricolor} (Viola del pensiero in lingua italiana), va a comporre con \textit{Le totémisme aujourd'houi} un unico progetto: ritrovare all'interno delle istituzioni religiose e sociali le caratteristiche di un sistema di classificazione.
Il volume \textit{Il totemismo oggi}, più breve del suo \enquote{seguito}, è composto con uno stile argomentativo rigoroso, pieno di dimostrazioni e riferimenti accademici eruditi, e ha il compito di mostrare la consapevolezza storica con cui Lévi Strauss si inserisce nel dibattito sul totemismo. Il volumetto di Lévi-Strauss, come viene definito da Claudo Stroppa\footnote{\cite{stroppa1965totemismo}}, si può considerare come il suo personalissimo ed originale modo di compiere uno \textit{status quaestionis} in materia di totemismo. 
I principali interlocutori accademici con cui Lévi-Strauss deve confrontarsi all'interno dell'opera sono altri antropologi che all'epoca già avevano tentato di spiegare il fenomeno del totemismo senza successo: nonostante il problema fosse già stato sondato più volte all'epoca, nessuno era riuscito fino a quel momento a produrre un'interpretazione esaustiva, in grado di spiegare il fenomeno totemico in tutti i suoi aspetti. Del resto, è da sottolineare, il fenomeno del totemismo aveva attirato l'attenzione di intellettuali di diversa formazione, tra i quali spicca Henri Bergson, \textit{maitre à penser} della Francia tra i due secoli, filosofo estremamente prolifico e influente sulle generazioni successive.
Nonostante la polemica presa di distanza dalle discipline filosofiche e dal modo di insegnarle in Francia che Lévi-Strauss conduce nella prima parte del celeberrimo \textit{Tristi Tropici}\footnote{\cite{levi1960tristi}, pp.49-52.}, la formazione in materia che ricevette influenzò profondamente il modo di condurre la ricerca etnologica. Si tengano presenti due fatti: da una parte Lévi-Strauss aveva preparato il concorso per diventare professore di filosofia\footnote{Vd. in \cite{levi1960tristi}, p.52.}, dall'altro lato la sua formazione in filosofia, che lo vede lettore particolarmente attento di pensatori della tradizione francese quali Montaigne, Rousseau e Bergson lo situa in una posizione privilegiata per applicare e sfruttare al meglio procedimenti tradizionalmente utilizzati in filosofia all'interno dell'etnologia.
%------------------------------------------------------------
\subsection{\textit{Il totemismo dal di dentro}}
Il quinto capitolo, l'ultimo de \textit{Il totemismo oggi}, può essere letto come un confronto tra due metodi apparentemente contrari, entrambi tesi a spiegare il totemismo, seppur partendo da prospettive diametralmente opposte.
Il metodo etnologico, come se fosse possibile definirne propriamente uno, si può approssimativamente riassumere come la raccolta di un numero indefinitamente ampio di testimonianze circa usi e costumi di popoli (si tenga presente che prima di Lévi-Strauss erano assai rari gli antropologi e gli etnologi con l'uso di compiere le loro indagini sul campo, si pensi a Malinowski, essendo molto più diffusa l'usanza di affidarsi alla vasta produzione di memorie e resoconti da parte di etnografi) seguito dal tentativo di elaborare una spiegazione che sia applicabile a tutti gli elementi oggetto dell'indagine. Ma tale spiegazione si può anche riassumere come il tentativo di individuare un elemento comune, un'unità minima rintracciabile alla base di ogni rituale caratteristico di un gruppo sociale


[insert here] la lettura levistraussiana della sociologia francese

Una graffiante pagina dell'opera sopracitata accusa ferocemente la filosofia di essersi ridotta a un vuoto gioco di retorica, costituito da elementi evocati con la parola e dissolti da questa stessa\footnote{Ho cominciato allora a capire che tutti i problemi, gravi o futili, possono essere liquidati applicando un metodo sempre identico, che consiste nel contrapporre due punti di vista tradizionali sulla questione,; introdurre cioè il primo con le giustificazioni del senso comune, per distruggerlo poi con il secondo; infine rigettarli uno da una parte e uno dall'altra, adottando invece un terzo punto di vista che riveli il carattere ugualmente parziale dei due altri, ricondotti con artifici di vocabolario agli aspetti complementari di una stessa realtà: forma e sostanza, contenente e contenuto, essere e parere, continuo e discontinuo, essenza ed esistenza ecc. [\cite{levi1960tristi}, p.49]}, ma nonostante queste aperte dichiarazioni di ostilità nei confronti