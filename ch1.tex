\chapter{Le radici del dibattito nelle\\ scienze umane in Francia}
%---------------------------------------------------------------------
\epigraph{L'origine filosofica della sociologia francese le ha giocato, in passato, qualche scherzo; potrebbe essere la sua migliore risorsa per il futuro.}{\textsc{Claude Lévi-Strauss}, \textit{La sociologia francese}, 1945 [tr. it. 2013], p.79.}

Da un punto di vista puramente storico, la filosofia ha posto i fondamenti per lo sviluppo di quegli strumenti concettuali che sono poi andati a formare le scienze umane. Queste ultime si sono distaccate dalla loro origine, rivendicando la loro autonomia, non mancando talvolta di appropriarsi dei concetti già elaborati in sede filosofica, come nel caso della scuola durkheimiana. Per comprendere appieno lo sviluppo delle scienze umane e dei loro strumenti occorre rintracciare il lascito filosofico di cui queste si sono appropriate, la loro genesi remota, le modalità d'indagine proprie delle scienze umane e quelle della filosofia, i concetti, a volte adoperati indiscriminatamente in una disciplina, talvolta  guardati con sospetto in un'altra.\par
Uno scienziato sociale accorto come Lévi-Strauss si caratterizza non solo per essere consapevole dell'origine dei concetti della propria disciplina, ma anche per il gusto con cui applica tali concetti, facendo un uso sapiente dei metodi di una disciplina per superare le difficoltà riscontrate nell'altra. Ciò che infatti emergerà da questo capitolo come da quelli successivi è l'eclettismo delle scienze umane in Francia.\par
%--------------------------------------------------------------------------------
\section{\normalfont{La sociologie française}}
Nel 1945 Georges Gurvitch commissiona a Claude Lévi-Strauss un articolo per il volume, curato dal sociologo russo, \textit{Les études sociologiques dans les différents pays} (1947)\footnote{Originariamente edito oltreoceano come \cite{gurvitch1945twentieth}. Per la genesi del saggio si veda l'\textit{Introduzione} di Fabrizio Denunzio in \cite{levistrauss2013sociologia}, pp.9-33.}: l'esito è un'opera di sintesi di ampio respiro, che descrive la sociologia francese nella sua peculiare genesi, erede del grande progetto di Auguste Comte, e nel suo particolarissimo rapporto con le altre scienze umane.\par
Nella prima parte (di tre) del saggio, nel fare un bilancio della sociologia francese al momento della stesura, Lévi-Strauss ne fornisce una definizione sintetica che rimarca la differenza tra questa e la sociologia in altri paesi del mondo.
\begin{quote}
    La sociologia ha lungamente sofferto in altri paesi, la Gran Bretagna e gli Stati Uniti in particolare, dell'esistenza di pareti stagne tra sé e l'etnologia. [...]
    [La sociologia] non si considera mai come una disciplina isolata che lavora nel suo campo, ma piuttosto come un metodo o un'attitudine di fronte ai fenomeni umani. Non c'è bisogno, allora, di essere sociologi per fare della sociologia. [...]
    L'universalismo della sociologia francese le ha permesso di contribuire al rinnovamento di molte scienze umane. [...]
    Questa stretta collaborazione tra la sociologia e tutte le tendenze o correnti di pensiero che hanno per oggetto l'Uomo e lo studio dell'Uomo, è uno dei tratti più caratteristici della scuola francese.\footnote{\cite{levistrauss2013sociologia}, pp.40-43.}
\end{quote} \par
Il saggio prosegue concentrandosi sulla vocazione universalistica delle scienze umane francesi, sul forte legame che corre tra sociologia, psicologia, etnologia e antropologia, fino a rilevare una tendenza in Francia: chiunque, anche un \enquote{non sociologo} può fare sociologia. Questo eclettismo ha ripercussioni da entrambi i lati: da un lato un filosofo ha diritto di esprimersi in materia sociologica, d'altra parte chi fa sociologia può estendere le sue analisi a ciò che sociologia non è.\par
Quest'ultimo punto getta nuova luce sulle ambizioni (non esplicitamente dichiarate, forse ancora inconsapevoli) di Lévi-Strauss: se la sociologia si è permessa di mutuare idee e concetti dalle altre discipline, prime tra tutte l'etnologia e la filosofia, allora è lecito cogliere l'indagine socio-antropologica come un modo alternativo di fare filosofia. Come la penetrante analisi della filosofa Claude Imbert mostra, infatti, nonostante non sia sbagliato definire Lévi-Strauss un antropologo, è improprio confinare alla sola antropologia il suo pensiero: le intuizioni e le scoperte di quest'ultimo, infatti, hanno notevolissime ricadute in campo filosofico, tanto da spingere la filosofa, studiosa di logica e teoria della conoscenza, a dedicare un saggio alla sua epistemologia nel \textit{Cambridge Companion to Lévi-Strauss}\footnote{\cite{wiseman2009cambridge}, pp.118-138. Anche in Italia il pensiero di Lévi-Strauss è stato oggetto d'interesse da parte di studiosi di filosofia e di storia della filosofia. Ecco alcuni contributi: \textsc{Sandro Nannini}, \textit{Il pensiero simbolico. Saggio su Lévi-Strauss}, Bologna, Il Mulino, 1981; \textsc{Sergio Moravia}, \textit{La ragione nascosta. Scienza e filosofia nel pensiero di Claude Lévi-Strauss}, G.C. Sansoni, Firenze, 1969; \textit{Lévi-Strauss e l'antropologia strutturale}, G.C. Sansoni, Firenze, 1973; \textit{Ragione strutturale e universi di senso. Saggio sul pensiero di Claude Lévi-Strauss}, Le Lettere, Firenze, 2004.}\par
Ciò significa che nel dibattito sul concetto di Ragione che si andrà a svolgere tra Sartre e Lévi-Strauss, non vi è ciò che uno studioso straniero potrebbe avvertire come un'invasione di campo: l'universalismo e l'eclettismo della sociologia francese permettono uno scambio intenso e arricchente proprio perché non si fermano dove in altre tradizioni sono innalzate \enquote{pareti stagne} tra le discipline.\par 
\vspace{0.5cm}
Chi è il padre della sociologia francese? La risposta non è scontata: nonostante Auguste Comte sia celebrato con ammirazione, la scelta di Lévi-Strauss vede contendersi il ruolo di padre della sociologia da Émile Durkheim e Marcel Mauss. Perché due pensatori così recenti? Solitamente, per nobilitare la genesi di una disciplina, si situa la sua origine indietro nel tempo.\par
Leggendo il saggio appare chiaro che la paternità di cui si sta parlando non è tanto della sociologia francese \textit{in generale}, quanto, invece, della \textit{futura} sociologia francese: la tragedia della Seconda Guerra Mondiale, sul finire della quale è scritto il saggio, getta inquietanti ombre sul futuro della sociologia in Francia, già colpita nel corso della Prima Guerra Mondiale\footnote{\enquote{[...] la prima generazione di discepoli di Durkheim è stata decimata durante la Grande Guerra e i vuoti non sono ancora stati colmati.}, \cite{levistrauss2013sociologia}, p.60.}.\par
L'interrogativo su \textit{chi sia il padre della sociologia francese} si può, alla luce di queste precisazioni, riformulare come \textit{chi sia il pensatore dal quale la sociologia francese possa ripartire dopo la tragedia Seconda Guerra Mondiale}, pertanto il padre della \textit{futura} sociologia francese.\par
Si tratta di Marcel Mauss, al quale il saggio è infatti dedicato.\par
Tra le debolezze e le incongruenze che Lévi-Strauss rileva nella teoria di Durkheim, c'è un punto sul quale l'autore e il grande sociologo divergono: la teoria del simbolismo.\par
Nell'ultima grande opera di Durkheim, \textit{Le forme elementari della vta religiosa}, si esplicitano, stando a Lévi-Strauss, le risorse ma anche i limiti delle teorie del suo autore: Durkheim fa precedere la società al simbolismo, cercando di spiegare il secondo fenomeno con il primo.\par
Ma nel 1945 Lévi-Strauss si limita a tratteggiare, abbozzando appena, ciò che sarà in grado di confutare meglio in seguito: il rapporto di causalità tra simbolismo e istituzione sociale, secondo Lévi-Strauss, è da Durkheim erroneamente invertito. Anche se non è ora il luogo adatto per sollevare la questione, vi è una notevole differenza tra la denuncia dei limiti della teoria durkheimiana presente ne \textit{La sociologie française} e la confutazione della stessa teoria presente in \textit{Le totémisme aujourd'houi} in concomitanza con la proposta di una soluzione alternativa al problema del totemismo. Ciò permette di sottolineare la differenza tra le posizioni e gli argomenti dell'autore nel 1945, quando il problema era intuito e rimarcato in uno scritto non incentrato sul problema del totemismo, e le posizioni dello stesso autore quasi vent'anni dopo, nel 1962 in un volume precisamente focalizzato sul totemismo e impegnato in un confronto critico tra i principali autori esperti in materia: nel saggio del 1945 si può già intravedere come Lévi-Strauss fosse per nulla soddisfatto della teoria durkheimiana circa il totemismo, che pure era stata oggetto della grande opera \textit{Le forme elementari della vita religiosa}; probabilmente in quel momento Lévi-Strauss non era ancora in grado di fornire una teoria alternativa a quella, seppur insoddisfacente, di Durkheim, tuttavia era in grado di criticare gli argomenti del sociologo attraverso la (parziale) confutazione qui presente.\par \smallskip
Una volta affrontata la \textit{pars destruens}, in cui sono forniti i motivi per cui Durkheim e la sua sociologia appartengono al passato della sociologia, rimane ora da discutere quali sono gli aspetti dell'opera di Mauss, analizzati non solo in questa sede ma anche nel paragrafo successivo, in special modo.

%DJNODIVNEOIJVNOEIJRNCOEIJFNVWEIONOWEITUNVDKJSNCOEWIJFNGOIWEJNVOWKJNVWEOITNU

Il saggio si conclude con un ultimo paragrafo particolarmente breve, se confrontato alle altre due sezioni, con le parole che fanno da epigrafe a questo paragrafo\footnote{\textit{Supra}}. Dal momento che si sono già descritti gli aspetti dell'opera maussiana ritenuti da Lévi-Strauss più fecondi, si può ora procedere a notare come, se si mette collegano l'inizio e la fine del saggio si può scorgere come l'autore stia suggerendo l'evoluzione della sociologia francese.\par
La sociologia durkheimiana è insoddisfacente nel suo metodo, ma un aiuto sembra poter provenire dalla Filosofia. Del resto, si è visto, la vocazione universalistica della sociologia francese le permette di trarre dalle altre discipline che studiano l'uomo ciò che le occorre per il suo scopo. E in questo caso ciò che le occorre sono gli strumenti concettuali per elaborare una teoria del simbolo esaustiva, strumenti che otterrà dall'etnologia maussiana e dalla linguistica strutturale della scuola di Praga.\par

%DJFNVSDKJVNSOVNITOVNOJFNVSKNVOEINVEF

Insomma, perché questo testo, breve e incentrato sul lavoro di altri e non sul lavoro dell'autore, è così importante? Ebbene, in questa sintesi ci sono già \textit{in nuce} tutti gli elementi che verranno poi faranno poi parte del pensiero \textit{maturo} di Lévi-Strauss: la teoria del simbolo come propedeutica e necessaria allo sviluppo delle istituzioni è in quest'opera solo abbozzata nei suoi contenuti, ma ne viene già riconosciuto il valore fondamentale.
%------------------------------------------------------------------
\section{Claude Lévi-Strauss lettore di Marcel Mauss}
Lévi-Strauss ha sempre nutrito stima e ammirazione per il lavoro di Marcel Mauss, del quale si sentiva direttamente discepolo. Sul finire degli anni '30, INFATTI, all'indomani del viaggio che lo avrebbe condotto intorno al mondo, ad esplorare culture e società lontane, Lévi-Strauss chiede a diversi etnologi: Lévy-Bruhl, Davy, e Mauss, la benedizione per l'impresa per cui sta partendo\footnote{\cite{levi1960tristi}, p.233.}.\par
Il rapporto tra i due etnologi, però, non si limita a questo episodio: vi è infatti un filo rosso tra la frammentaria ma rivoluzionaria opera di Mauss e l'ambizioso progetto levistraussiano. Tale rapporto d'elezione, rimarcato ed approfondito nell'\textit{Introduzione} all'opera di Marcel Mauss (1950) è provato anche dall'ispirazione del pensiero levistraussiano in tutta la sua evoluzione: il metodo, il predominio della ricerca etnologica su quella sociologica, l'attenzione per le strutture, sono aspetti consapevolmente ripresi dall'approccio maussiano. Ma vi è un altro motivo che potrebbe aver spinto Lévi-Strauss a scegliere il lavoro di Mauss come fonte di ispirazione a cui collegarsi: il carattere frammentario degli scritti del maestro, contrariamente all'opera di Durkheim, fornisce maggior libertà interpretativa, rotte inesplorate, per inserire il proprio contributo nella tradizione sociologica francese.\par
Il corpus maussiano, per utilizzare una metafora visiva, può essere assimilato a un affresco composto da elementi sparsi, quasi puntiformi, dai contorni sfumati appena accennati; mentre gli scritti di Durkheim potrebbero essere un affresco dai contorni precisi, le figure definite e statuarie. L'opera di Lévi-Strauss, per restare all'interno della metafora, si offre non come un completamento (ciò presupporrebbe infatti una incompiutezza di fondo), ma come un arricchimento, un'aggiunta appena sovrapposta al disegno originale di Mauss: una giustapposizione creata con l'obiettivo di fornire maggior chiarezza, senza preoccuparsi troppo filologicamente di mantenere di ciò che stava alla base del nuovo quadro, in questo caso l'affresco maussiano. Ciò che si vedrà più chiaramente nell'\textit{Introduzione}, infatti, è proprio il tentativo di \textit{riempire gli spazi} del lavoro di Mauss, ma non per questo esimersi dall'innestare il proprio originale contributo nel lavoro del maestro.\par
Quanto Lévi-Strauss fosse consapevole di questo processo di appropriazione passante per la sua personale reinterpretazione non è dato saperlo con precisione, ma si vedrà come ripercorrere il cammino tracciato nelle pagine dedicate a Mauss può chiarire la ricchezza e la profondità del pensiero di Lévi-Strauss, nonché il rispetto verso la tradizione nella quale intende iscriversi.\par
%----------------------------------------------------------------....................
\subsection{Le forme di classificazione: tra pensiero ed istituzioni sociali}
Il saggio \textit{De quelques formes primitives de classification}\footnote{\cite{durkheim1903formes}, in italiano come \cite{durkheimmaussformes}.}, scritto a quattro mani con lo zio Émile Durkheim, si può per certi aspetti considerare una tappa preliminare alla stesura del capolavoro di Mauss, l'\textit{Essai sur le don}\footnote{\cite{mauss1923essai}; in italiano come \cite{mauss2002saggio}.}.\par
Il saggio del 1923, come dice il titolo, analizza alcune forme di classificazione presenti nelle civiltà \textit{arcaiche}: le fratrie, i totem e i clan. La classificazione, argomento di carattere spiccatamente teoretico, permette di introdurre il problema di stabilire una teoria del \enquote{segno}, che spieghi il sorgere di questo in rapporto allo svilupparsi della socialità.\par
Il principale merito di questo saggio, in cui predomina l'influenza dell'etnologia maussiana, consiste nel riconoscere il valore sociologico del segno, e di ipotizzarne la causa nella socialità. Come si è visto in precedenza\footnote{vedi \textit{supra}, § 1.1.}, però, la soluzione ipotizzata da Durkheim è giudicata da Lévi-Strauss insoddisfacente: non è la socialità causa della comparsa del simbolo e del segno, ma il contrario: è il segno a costituire la causa, e quindi la chiave di volta per risolvere il problema del sorgere delle istituzioni sociali. Questo punto verrà ribadito con forza da Lévi-Strauss ne \textit{Il pensiero selvaggio} e \textit{Le totémisme aujourd'hui}.\par
Nell'articolo del 1903, però, il problema è solo sfiorato, e nemmeno in questi termini: non si parla ancora di teoria del simbolo, ma solo di \enquote{segni} e \enquote{forme di classificazione}.\par
È necessario, tuttavia, specificare i presupposti che costituiscono la base degli argomenti dell'articolo: per gli autori la classificazione si forma non spontaneamente da uno stato del pensiero di compenetrazione e indistinguibilità tra gli elementi; ciò che innesca la formazione di tali categoria è l'organizzazione del gruppo, ritenuta da Durkheim spontanea.\par
Il pensiero dei primitivi, non diversamente che per Lévy-Bruhl, è caratterizzato dal \enquote{principio di partecipazione}, uno stato di non distinzione tra un elemento e un altro, in cui tutti gli oggetti si assomigliano e fanno parte di un unico \enquote{flusso}\footnote{\enquote{In quel punto la coscienza non è che un flusso continuo di rappresentazioni che si perdono le une nelle altre, e quando le differenziazioni cominciano ad apparire, sono tutt'affatto frammentarie}. \cite{durkheimmaussformes}, p.23.}. In questa prospettiva la storia della logica, almeno per quanto riguarda i concetti di generi e specie, è da rifondare: tali concetti sono soggetti ad uno sviluppo storico, non si trovano in natura e non sono dati nella loro interezza; sono invece prodotti di uno sviluppo che coinvolge più elementi eterogenei, innestandosi in un processo che se fosse lasciato alla sua spontaneità non porterebbe alla nascita di concetti quali generi e specie. Ma si veda ciò attraverso le parole di Durkheim e Mauss:
\begin{quote}
    Non soltanto la nostra nozione attuale della classificazione ha una storia, ma questa stessa storia presuppone una considerevole preistoria. In realtà, lo spirito umano ha preso le mosse da uno stato di massima indistinzione; ancor oggi c'è tutta una parte della nostra letteratura popolare, dei nostri miti, delle religioni che si basa su una fondamentale confusione di ogni immagine e idea. Si potrebbe affermare che immagini o idee separate le une dalle alte con una certa chiarezza, non ce ne siano.\footnote{\cite{durkheimmaussformes}, p.21.}[...]\par \vspace{-0.2cm}
    Non è vero, dunque, che l'uomo classifichi spontaneamente e per una sorta di necessità naturale: agli inizi, fanno difetto all'umanità anche le condizioni indispensabili alla funziona classificatrice. E poi basta analizzare l'idea di classificazione per comprendere che l'uomo non poteva trovare in se stesso gli elementi essenziali. Una classe è un gruppo di cose; orbene, le cose non si offrono all'osservazione di per se stesse raggruppate.\footnote{\cite{durkheimmaussformes}, p.23.}
\end{quote} \par
Dare il giusto peso a queste parole significa riconoscere che la classificazione non è un fatto innato, appartenente all'ordine naturale delle cose, ma un dato culturale, che si va a manifestare unicamente ove vi sia la civiltà e le sue istituzioni (la cui generazione, dal momento che la classificazione non è un evento spontaneo ma causato dalle istituzioni, rimane da giustificare).\par
Ora, se si prosegue nella lettura sarà facile capire le intenzioni degli antropologi: \enquote{L'importanza di questa classificazione è tale che si estende a tutti i fatti della vita e se ne ritrova la traccia in tutti i riti principali}\footnote{\cite{durkheimmaussformes}, p.28.}.\par
Mentre Durkheim è solito concentrarsi sulle \enquote{rappresentazioni collettive}, una definizione che ancor oggi solleva qualche perplessità, il contributo di Mauss si scorge soprattutto nel modo \textit{a tutto tondo} in cui sono discussi i fatti sociali, in questo caso le classificazioni: le forme di classificazione sono osservate sotto molteplici aspetti inscindibili: religiosi, politici, economici, espressivi e mitopoietici.\par
Per riutilizzare le parole di Bruno Karsenti, \enquote{si tratta di cogliere l' "uomo tutto intero", o l' "uomo totale", e al tempo stesso di inscrivere la sociologia "nell'antropologia". Su un piano strettamente sociologico, si tratta sempre di cogliere l'essenza dei rapporti sociali, di raggiungere il punto in cui questi si intrecciano concretamente e si esprimono in una totalità; o ancora, secondo un concetto cui Mauss deve gran parte della sua celebrità, si tratta di cogliere il \enquote{fatto sociale totale}.}\footnote{\cite{karsenti1997uomo}, p.33.}\par
Si leggano alcune pagine dell'ultimo Durkheim, in cui l'apporto di Mauss si può dire trascurabile: è noto che il grande sociologo nell'ultima parte della sua produzione si sia dedicato alla sociologia religiosa, ma è bene ricordare che il fine di questo interesse fosse in ultima analisi l'elaborazione di una teoria della conoscenza attraverso la sociologia.\par
Si prenda un estratto da \textit{Le forme elementari della vita religiosa} di Émile Durkheim (1912), dichiaratamente debitore all'articolo scritto a quattro mani con Marcel Mauss:
\begin{quote}
   [...] queste classificazioni sistematiche sono le prime che incontriamo nella storia; ora, si è visto che esse \textit{si sono foggiate sull'organizzazione sociale}, o piuttosto che \textit{hanno preso come schemi i quadri stessi della società}. Le fratrie hanno servito da generi, e i clan da specie. Poiché erano raggruppati gli uomini, essi hanno potuto raggruppare le cose; per classificare queste ultime si sono limitati a fare loro posto nei gruppi che formavano essi stessi. [...] L’unità di questi primi sistemi logici non fa che riprodurre l’unità della società.\footnote{\cite{durkheim2013forme}, p.201; corsivo mio.}
\end{quote} \par
Il passo afferma esplicitamente l'influenza dell'organizzazione sociale sull'individuo. Ma dal momento che le forme di classificazione non precedono le istituzioni sociali, Durkheim tiene a specificare il motivo per cui le forme di classificazione, i generi e le specie, non possono precedere l'organizzazione politica di un gruppo, e quindi non possono essere generate spontaneamente dal presentarsi alla coscienza di immagini analoghe, oggetti diversi con alcuni elementi e caratteristiche comuni.\par
Una volta postulata l'esistenza di una \enquote{mentalità primitiva}, dominata da un principio di indistinzione, o partecipazione, l'unico modo per risolvere il problema è spiegare l'avvento della classificazione come l'imitazione di un fenomeno spontaneo, l'organizzazione dei gruppi.
\begin{quote}
    Ma una cosa è il sentimento delle somiglianze, un’altra è la nozione di genere. Il genere è lo schema esteriore di cui gli oggetti percepiti come simili formano, in parte, il contenuto. E il contenuto non può fornire esso stesso lo schema sotto il quale si dispone. Esso è fatto di immagini vaghe e ondeggianti, dovute alla sovrapposizione e alla fusione parziale di un determinato numero di immagini individuali, che si trovano dotate di elementi comuni; lo schema, al contrario, è una forma definita, dai contorni stabili,ma suscettibile di applicazione a un numero determinato di cose, percepite o meno, attuali o possibili. [...] Ecco perché tutta una scuola di studiosi si rifiuta, non senza ragione, di identificare l’idea di genere e quella di immagine generica. L’immagine generica non è che la rappresentazione residua, dai limiti incerti, che lasciano in noi rappresentazioni simili, quando sono simultaneamente presenti nella coscienza; il genere è invece un simbolo logico in virtù del quale pensiamo distintamente queste affinità e altre analoghe.\footnote{\cite{durkheim2013forme}, pp.203-204.}
\end{quote} \par
Le parole di Durkheim tradiscono una profonda influenza di Kant e il kantismo: l'importanza delle forme, la precedenza del contenuto e l'impossibilità di questo di condizionare gli schemi in cui è organizzato. Generi e specie non sono generati spontaneamente dall'analogia tra oggetti simili, quest'ultima è in grado di generare unicamente un \enquote{sentimento delle somiglianze}, mentre la nozione di genere richiede evidentemente l'intervento di un agente esterno, in questo caso l'organizzazione sociale, per mostrarsi alla coscienza con chiarezza. Ecco infatti cosa aggiunge Durkheim immediatamente dopo:
\begin{quote}
    L’idea di genere è uno strumento del pensiero evidentemente costruito dall'uomo. Ma per costruirlo, ci è occorso, quanto meno, un modello; infatti, come sarebbe potuta nascere questa idea se non ci fosse stato nulla in noi o fuori di noi in grado di suggerircela? Rispondere che essa ci è data a priori significa non rispondere; questa soluzione pigra è, come si è detto, la morte dell'analisi. Orbene, \textit{non si vede dove avremmo potuto trovare questo modello indispensabile, se non nello spettacolo della vita collettiva}.\footnote{\cite{durkheim2013forme}, p.204, corsivo mio.}
\end{quote}
Il limite maggiore dell'impostazione di ricerca durkheimiana consiste nella forte ispirazione positivista della sua indagine, che lo spinge a rimanere aderente all'osservazione empirica (per così dire: occorre attendere Malinowski perché l'etnologia diventi una disciplina sul campo): il funzionalismo durkheimiano, limitandosi ai dati osservati, ai fenomeni sociali nella loro concretezza, non risale oltre l'organizzazione dei gruppi, non riesce ad ipotizzare l'esistenza di un sistema, una \textit{struttura} di carattere cognitivo, alla base dei fenomeni collettivi. Eppure, bisogna riconoscere, che in alcuni passi del saggio appaiono intuizioni in seguito riprese e sviluppate da Lévi-Strauss: Durkheim e Mauss riconoscono il carattere speculativo delle forme primitive di classificazione, la loro parentela con il pensiero scientifico\footnote{[le classificazioni] sono opera di scienza e costituiscono una prima filosofia della natura.\cite{durkheimmaussformes}, pp.72-73.}. È bene rimarcare questo punto poiché \textit{Il pensiero selvaggio} e \textit{Il totemismo oggi} di Lévi-Strauss, opere di oltre cinquant'anni successive, partono proprio da concetti già presenti in questo articolo.\par
Nelle conclusioni, fino ad ora aderenti allo spirito e al contenuto dell'articolo, si può notare una svolta nelle posizioni di Mauss e Durkheim: dopo aver affermato lo statuto di \enquote{modello} della società rispetto ai sistemi di classificazione, le argomentazioni di carattere puramente sociologico cedono il posto ad altre di carattere psicologico-cognitivo: gli autori del saggio si accorgono che la loro proposta interpretativa ha solo rimandato il problema.
\begin{quote}
    Tuttavia, se quanto precede ci mette in condizioni di comprendere come abbia potuto costruirsi la nozione di classi collegate fra loro in un solo e identico sistema, \textit{noi tuttora ignoriamo quali siano le forze che hanno indotto gli uomini a ripartire le cose in classi secondo il metodo da essi adottato}. Dal fatti che sia la società a fornire il quadro esteriore della classificazione, non consegue necessariamente che il modo con cui il quadro è stato adoperato tragga la sua ragione dalle medesime origini.\footnote{\cite{durkheimmaussformes}, p.75, corsivo mio.}
\end{quote}
Le testimonianze etnografiche e le prove raccolte nelle società primitive, se interrogate con la stessa modalità utilizzata nelle  precedenti, non sono in grado di fornire una risposta soddisfacente. Gli autori del saggio sono consapevoli del limite della sociologia in tal senso: l'indagine socio-antropologica li ha portati alle soglie di una materia in cui l'osservazione empirica non può rivelarsi sufficiente.\par
Il passaggio dalla sociologia all'antropologia intesa nelle sue implicazioni più spiccatamente psicologiche è segnato da un importante cambiamento terminologico: la \enquote{nozione}, il \enquote{concetto} di classificazione, legato come tale alla sfera razionale-intellettuale, non si darebbe in maniera spontanea come si è visto in una mente \enquote{primitiva}. Nel saggio, infatti, viene rimarcata una netta differenza tra un concetto, o nozione, e un'emozione, un sentimento:
\begin{quote}
    Una classificazione logica è una classificazione di concetti. E il concetto è la nozione di un gruppo di esseri nettamente determinati, i cui limiti possono essere esattamente tracciati. L'emozione, viceversa, è cosa essenzialmente fluida e inconsistente.[...] D'altra parte per poter segnare i limiti di una classe, bisogna inoltre aver analizzato i caratteri mediante i quali si riconoscono e si distinguono gli esseri raccolti in detta classe. Ora l'emozione è di natura refrattaria all'analisi o per lo meno vi si presta a fatica giacché è troppo complessa. Soprattutto quando è di origine collettiva, sfida l'esame critico e ragionato.\footnote{\cite{durkheimmaussformes}, pp.77-78.}
\end{quote}
Da queste parole si può vedere come Durkheim e Mauss siano consapevoli del limite della loro analisi: la loro disamina critica può da una parte aver riconosciuto l'evoluzione del concetto di classificazione, ma dall'altra parte manca ancora un'analisi approfondita della forza con con la quale la classificazione, in tutte le sue forze, si è imposta all'uomo in ogni dove.\par
Il funzionalismo durkheimiano, d'altra parte, consegue il notevole risultato di svincolare l'analisi antropologica dal paradigma storico e diffusionista: come si è visto, non viene ipotizzato un modello originale di classificazione dal quale le altre forme di classificazione sono derivate secondo un processo storico, come nella crescita di un albero da unico tronco si generano più rami. La quantità di testimonianze riportate, e la varia locazione delle società osservate, inducono i due studiosi a riconoscere alle classificazioni l'essere proprie dell'individuo socializzato \textit{en général}, e alla forza 


Tentando di fare un breve bilancio, appare chiaro che l'articolo di Durkheim e Mauss contiene diversi snodi concettuali difficoltosi, che la rendono una lettura tanto foriera di idee quanto delicata: si inserisce, come già evidenziato, in un dibattito che coinvolge tutte le discipline umanistiche in Francia ad inizio XX secolo, ognuna determinata a rivendicare un proprio statuto scientifico così come il proprio campo d'indagine; la discussione sulla natura della mentalità dei primitivi, così differente dalla mentalità dei moderni, tuttavia simile nelle sue manifestazioni più semplici; l'analisi delle forme di classificazione nel loro rapporto di reciproca influenza con le istituzioni sociali.\par
La differenza fondamentale da rilevare tra Durkheim, e con lui Marcel Mauss, e Lévi-Strauss consiste non nell'ipotesi sulla formazione dei sistemi di classificazione, ma nel concetto di mentalità primitiva: una volta ipotizzata una mentalità primitiva costituita da un tutto indistinto, di esseri e cose che si compenetrano in un unico \textit{continuum}, Durkheim non riesce a spiegarsi il processo di formazione dei sistemi di classificazione se non introducendo un elemento esterno, ossia la vita sociale, spontaneamente organizzata in gruppi che fungono da modello alle nozioni di genere e specie; Lévi-Strauss, invece, si pone in una posizione diametralmente opposta, riconoscendo nella mentalità primitiva il germe del pensiero scientifico e l'origine delle classificazioni.\par
%-------------------------------------------------------------
\subsection{\normalfont{Introduzione} all'opera di Marcel Mauss}
È il 1950 quando una raccolta antologica di saggi di Marcel Mauss viene pubblicata dalla Presse Universitarie de France. L'opera, composta da saggi pubblicati autonomamente sulle pagine di riviste quali \textit{L'Année Sociologique}, può considerarsi a buon diritto come una \textit{summa} del pensiero di Mauss. Il volume, originariamente intitolato \textit{Sociologie et anthropologie}, è edito in Italia da Einaudi ben quindici anni dopo, nel 1965, con il titolo \textit{Teoria della magia e altri saggi}, con la prefazione di Ernesto de Martino.\par
Tale edizione vede mantenuta e tradotta integralmente l'introduzione di Claude Levi-Strauss\footnote{\cite{mauss1965teoria} [pp.\textsc{xv-liv}]}, introduzione accolta con entusiasmo a livello internazionale, tanto da portare la casa editrice londinese Routledge a pubblicarla come volume autonomo in seguito\footnote{\cite{levi1987introduction}}. Percorrendo le parole di Levi-Strauss, tuttavia, presto ci si accorge che quella che si ha tra le mani è ben più di un'introduzione all'opera di Mauss: il lavoro di quest'ultimo, infatti, diventa quasi un pretesto, la materia prima di una rielaborazione estremamente originale. Si tratta di qualcosa di più che un'esegesi: è un omaggio, un contributo originale, una lettura estremamente personale che molto si allontana dallo scopo di introdurre, come avverte Georges Gurvitch, definendola una \enquote{interpretazione molto personale} nella sua prefazione\footnote{\cite{mauss1965teoria}, p.\textsc{xiv}.}.\par
Non importa in questa sede l'aspetto \textit{esteriore} del rapporto tra Marcel Mauss e Claude Lévi-Strauss, già accennato nelle pagine precedenti, quanto invece la rappresentazione che lo stesso Lévi-Strauss fornisce del suo rapporto con Mauss. Inoltre lo scopo del presente lavoro non è ricostruire l'integrità concettuale dei testi di Marcel Mauss per poi comprendere le possibili distorsioni o forzature da parte di Claude Lévi-Strauss, ma la ricchezza del pensiero di quest'ultimo, nonché gli snodi concettuali propedeutici alla formazione del suo peculiare concetto di ragione; per questo motivo è molto più funzionale a tale scopo concentrarsi sulla rappresentazione del pensiero di Mauss e del proprio fornita da Lévi-Strauss stesso.\par \vspace{0.5cm}
Senza dedicarsi a una disamina testuale dell'\textit{Introduzione}, i concetti funzionali allo scopo del presente lavoro sono: il legame delle scienze umane con la psicologia e la psicopatologia, la metodologia maussiana, assai differente dal metodo di Durkheim, e l'intuizione dell'esistenza di una struttura, un sistema non rilevabile empiricamente, sotteso ai fenomeni sociali.\par
Si inizi da quest'ultimo punto: nel giustamente famoso {Essai sur le don}\footnote{\cite{mauss1923essai}; tradotto in italiano come \cite{mauss2002saggio}.} Lévi-Strauss individua l'applicazione più efficace e feconda, per quanto non esaustiva, di un'osservazione attenta e penetrante, volta a rintracciare \textit{non} la \enquote{funzione} sociale del dono e dei rituali ad esso collegati, \textit{ma} la \enquote{struttura}, invisibile e onnipresente nell'organizzazione sociale, della quale il dono è una manifestazione.\par
Occorre rimarcare che la nozione di struttura non viene sviluppata nella sua interezza da Mauss: questo è invece merito di Lévi-Strauss.
\begin{quote}
    Mauss vi appare [nell'argomentazione seguita nell'\textit{Essai sur le don}] dominato con ragione da una certezza di ordine logico, e cioè che lo \textit{scambio} è il comune denominatore di un grande numero di attività sociali apparentemente eterogenee. Ma egli non giunge a vedere questo scambio nei fatti. L'osservazione empirica non gli fornisce lo scambio, ma soltanto - come dice lui stesso - \enquote{obblighi: dare, ricevere, ricambiare}. Tutta la teoria esige così l'esistenza di una \underline{struttura}, di cui l'esperienza non offre che frammenti, i membri sparsi, o piuttosto gli elementi.\footnote{\cite{mauss1965teoria}, p.\textsc{xli}; corsivi originali, sottolineature mie.}
\end{quote}
Ciò che Lévi-Strauss sta cercando di fare, servendosi dell'opera di Mauss, è riconoscere l'esistenza di una forma di comunicazione alla base dell'istituzione del dono; in altri termini concepire il dono come una manifestazione particolare di più generici processi di scambio. Per rilevare tale teoria della comunicazione come fondamentale alla pratica del dono, però, Mauss non ricorre a un numero impressionante di testimonianze, come invece facevano antropologi quali James Frazer, e nemmeno a un numero minore di casi focalizzandosi sulla comparazione per trovare correlazioni concomitanti, come invece era proprio di Durkheim: la lezione maussiana consiste proprio nell'applicare un metodo composto da ipotesi e osservazioni empiriche, che però non scoraggino l'etnologo dall'esercitare il suo \textit{gusto}. Scrive infatti Lévi-Strauss di Mauss, utilizzando Malinowski (\enquote{migliore
come osservatore che come teorico}\footnote{\cite{mauss1965teoria}, p.\textsc{xxxix}.}) per compiere un paragone: 
\begin{quote}
    Là dove Mauss vedeva un \textit{rapporto costante} tra fenomeni, nel quale si trova la loro spiegazione, Malinowski si chiede soltanto \textit{a che cosa servano}, per cercarne una giustificazione. Questo modo di porre il problema annulla tutti i progressi precedenti, perché introduce di nuovo un apparato di postulati, privo di valore scientifico.\\
    Che Mauss abbia posto il problema nell’unico modo in cui doveva essere posto, è provato, al contrario, dagli sviluppi più recenti delle scienze sociali, i quali permettono di sperare in una loro matematizzazione progressiva. In taluni campi essenziali, come quello della parentela, \uline{l’analogia con il linguaggio, così decisamente affermata da Mauss, ha consentito di scoprire le regole precise secondo le quali si formano, in un tipo qualunque di società, cicli di reciprocità, le cui leggi meccaniche sono ormai conosciute, permettendo l’uso del ragionamento deduttivo in un campo che sembrava sottoposto all’arbitrio più completo}.\footnote{\cite{mauss1965teoria}, p.\textsc{xl}; corsivi originali, sottolineature mie.}
\end{quote}
Questo passaggio è particolarmente rivelatore dal momento che 1) viene mostrata la fertilità del metodo maussiano contrapposto alle osservazioni di Malinowski; 2) viene rivendicata esplicitamente una continuità tra l'opera di Mauss e quella di Lévi-Strauss tramite il tema del linguaggio e della parentela.\par
Malinowski, com'è infatti più volte ribadito nel corso del saggio, ha il suo maggior merito nell'osservare e nell'attenersi ai dati empirici così rilevati, mentre il merito di Mauss consiste proprio nel postulare ed intuire le strutture che stanno alla base dei fenomeni sociali e nell'uso di un metodo non funzionalista per la spiegazione dei fenomeni, che vincolerebbe indebitamente le osservazioni alla domanda \enquote{a cosa serve?}. In questo modo si svincola l'interpretazione dei fatti sociali dal funzionalismo in favore di un \enquote{metodo dei residui}\footnote{\cite{levistrauss2013sociologia}, p.75.}; ma si introduce anche la possibilità di utilizzare un metodo \enquote{deduttivo}, in quanto una volta rilevato il meccanismo sottostante ad un fenomeno collettivo si può risalire alle sue altre manifestazioni a partire dalla struttura postulata. In questo modo l'etnologia e la sociologia divengono parte del progetto scientifico maussiano, che ambiziosamente vuole mettere in comunicazione le scienze umane con le scienze naturali rimarcando la sostanziale continuità tra le macroaree disciplinari\footnote{\enquote{Innanzitutto, esistono società solo tra esseri viventi. I fenomeni sociologici attengono alla vita. Dunque, la sociologia, come la psicologia, non è che una parte della biologia; sia voi che noi, infatti, abbiamo da fare con uomini in carne e ossa, viventi o già vissuti.\\
La sociologia, poi, come la psicologia umana, è una parte di quella parte della biologia che è l’antropologia, cioè il complesso delle scienze che considerano l'uomo come essere vivente, cosciente e socievole.} in \cite{mauss1965teoria}, p.298.}
Oltre alla comunità d'intenti con Mauss, nel passo precedente Lévi-Strauss sottolinea anche la continuità tematica con il maestro: l'analisi dei sistemi di parentela, così ampiamente studiata nel monumentale volume \textit{Le structures élémentaires de la parenté}, vuole riprendere il progetto di Mauss dove è rimasto interrotto, ossia dall'analogia tra istituzioni sociali, la parentela, e il linguaggio.\par
Se si uniscono questi aspetti alle \enquote{immense possibilità} offerte dal corpus maussiano (già definito \enquote{il \textit{novum organum} delle scienze sociali del \textsc{xx} secolo}), non è improprio concludere che ponendosi in forte continuità con Mauss, e quindi celebrando la novità delle sue teorie, Lévi-Strauss intende marcare il carattere rivoluzionario del \textit{suo} modo di fare antropologia. 

%------------------------------------------------------------
\section{\normalfont{Le totémisme du dedans}}
Il volume pubblicato nel 1962, \textit{Il totemismo oggi}\footnote{\cite{levi2020totemismo}.}, costituisce la prima parte di un progetto più ambizioso, che vede il suo compimento nel volume \textit{Il pensiero selvaggio}\footnote{\cite{levi2010pensiero}}, il quale sarà oggetto di osservazione approfondita nel capitolo successivo. \textit{La pensée sauvage}\footnote{Titolo che in lingua francese evoca esplicitamente la teoria del simbolo attraverso l'ambiguo nome del fiore \textit{Viola tricolor} (Viola del pensiero in lingua italiana)}, va a comporre con \textit{Le totémisme aujourd'houi} un unico progetto: ritrovare all'interno delle istituzioni religiose e sociali le caratteristiche di un sistema di classificazione.\par
Il volume \textit{Il totemismo oggi}, più breve del suo \enquote{seguito}, è composto con rigoroso stile accademico, ricco di dimostrazioni, e ha il compito di mostrare la consapevolezza storica con cui Lévi Strauss si inserisce nel dibattito sul totemismo. È il suo personalissimo ed originale modo di compiere uno \textit{status quaestionis} in materia di totemismo: riprendere il discorso dove gli altri pensatori lo hanno interrotto.\par
Il quinto capitolo, l'ultimo de \textit{Il totemismo oggi}, appare ad un primo sguardo come assai differente dal resto dell'opera: mentre nei capitoli precedenti la comparazione avviene tra antropologi ed etnologi, nel quinto capitolo fa ingresso nel dibattito Henri Bergson. \par
Perché un filosofo, nonostante il suo noto eclettismo, è inserito in una tavola rotonda immaginaria presidiata da sociologi, etnologi ed antropologi?\par
Ebbene, Henri Bergson non è il solo filosofo in dialogo con l'antropologia, anche Jean-Jacques Rousseau compare tra gli autorevoli interlocutori, e assieme al filosofo fornisce spunti di riflessione determinanti per le conclusioni del volumetto.\par
Ciò che infatti differenzia, e avvantaggia, Bergson e Rousseau rispetto agli altri antropologi, è da una parte la loro solitudine e ignoranza dei resoconti etnografici, dall'altra il metodo \textit{riflessivo} utilizzato per spiegare il sorgere del pensare primitivo attraverso il totemismo.\par
Ne \textit{La pensée sauvage}, Lévi-Strauss si dichiara esplicitamente debitore a Auguste Comte per la sua intuizione 

\begin{quote}
    Ma ciò che c'importa, per la lezione che ne vogliamo trarre, è che Bergson e Rousseau siano riusciti a risalire fino ai fondamenti psicologici di istituzioni esotiche (nel caso di Rousseau senza supporne l'esistenza), attraverso un percorso di interiorità, cioè cercando su se stessi modi di pensiero, colti dapprima dal di fuori o semplicemente immaginati. Essi dimostrano così che ogni spirito d'uomo è un virtuale luogo d'esperienza per controllare quanto avviene in altri spiriti di uomini, quali che siano le distanze che li separano.\footnote{\cite{levi2020totemismo}, p.113.}
\end{quote}
Il passo, di per sé chiaro, non avrebbe bisogno di chiarimenti: ogni uomo è virtualmente in grado di sperimentare ciò che in qualsiasi altro uomo, esotico o \textit{selvaggio} che sia, in entrambi i casi sconosciuto, costituisce i fondamenti dell'esperienza. In altri termini, per il semplice fatto che Bergson, Rousseau e un componente della tribù Arunta condividono la natura umana, essi condividono anche le medesime modalità di conoscenza.\par
In cosa consistano tali modalità è oggetto dell'indagine svolta in \textit{La pensée sauvage}, ma già al termine de \textit{Le totémisme aujourd'houi} la soluzione è abbozzata: l'elemento fondamentale di ogni sistema di classificazione, tra cui quello totemico, è il principio di opposizione e di integrazione, ossia le funzioni elementari necessarie alla comparsa del simbolo.
\begin{quote}
    [...] Bergson ha potuto comprendere che cosa si nasconde dietro il totemismo, in quanto il suo pensiero era in simpatia, senza che lui lo sapesse, con il pensiero delle popolazioni totemiche.\footnote{\cite{levi2020totemismo}, p.XXX}
\end{quote}


Tenendo conto del fatto che \textit{Le totémisme autjourd'houi} compone con \textit{La pensée sauvage} un'opera unica, e che quest'ultimo volume è giustamente considerato di carattere strutturalista, si può comprendere come il rapporto dei pensatori \textit{strutturalisti} (termine che è stato lungamente discusso vista l'eterogeneità degli autori raggruppati sotto tale etichetta\footnote{Per un aggiornato, per quanto sintetico, \textit{status quaestionis} si rimanda a \cite{fornero2006strutturalismo}.}, o sicuramente di Lévi-Strauss, è più ricco e problematico di quanto si crede. Mentre alcuni riconoscono con ragione nello strutturalismo \enquote{la fine dell'umanesimo}, \enquote{la morte dell'individuo}, \enquote{l'anti-filosofia della riflessione}, è da rimarcare come Lévi-Strauss sfrutti i risultati di queste ultime a sostegno delle sue tesi. Ciò, ovviamente, non confuta la correttezza delle affermazioni sopra riportate, ma dimostra come 1)~non esista un \textit{unico} strutturalismo, ma solo analogie tra diverse modalità di condurre la riflessione sull'uomo e le istituzioni; 2)~il debito dello \textit{strutturalismo} verso la storia della filosofia precedente, e in particolare nei confronti di quelle correnti filosofiche delle quali si proponeva di essere il superamento.\par